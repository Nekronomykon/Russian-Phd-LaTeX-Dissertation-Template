% !TeX spellcheck = ru_RU
% !TeX encoding = UTF-8
\chapter{Исследование напряжений в производных бицикло[3.3.1]нонана}\label{ch:Strain:331}

\paragraph{Реакции разделения связей} Наименьшую погрешность обеспечивают \emph{гипергомодесмотические} реакции, сохраняющие не только состав, как гомодесмические... но и число связей различных типов. В частности, для 3,7,9-замещённых аналогов бицикло[3.3.1]нонана выводится уравнение
\begin{center}
  \chemfig{[:-30]Z*6(-(-[:180]-[:120]X-[:+60]-[:0]?)--Y---)}
  + \chemfig{H_3C-[:-30]X-[:+30]CH_3} + \chemfig{H_3C-[:-30]Y-[:+30]CH_3} + \chemfig{H_3C-[:-30]Z-[:+30]CH_3} + 4\(\cdot\) \chemfig{H_3C-[:-30]-[:+30]CH_3}
  + 4\(\cdot\)\chemfig{H_3C-[:+30](-[:+90]CH_3)-[:-30]CH_3}
  \(\longrightarrow\)
  
  \(\longrightarrow\) 
  2\(\cdot\)\chemfig{H_3C-[:+30]X-[:-30]-[:+30]CH_3} + 2\(\cdot\)\chemfig{H_3C-[:+30]Y-[:-30]-[:+30]CH_3} + 4\(\cdot\)\chemfig{H_3C-[:+30](-[:+90]CH_3)-[:-30]-[:+30]CH_3} + 
  2\(\cdot\)\chemfig{H_3C-[:+30](-[:+90]CH_3)-[:-30]Z-[:+30]CH_3}
\end{center}

Для 1,3,5,7-аналогов исследуемой бициклической системы
\begin{center}
  \chemfig{[:-30]*6(-A(-[:180]-[:120]X-[:+60]-[:0]\phantom{Z}?)--Y--Z-)}
  + \chemfig{H_3C-[:-30]X-[:+30]CH_3} + \chemfig{H_3C-[:-30]Y-[:+30]CH_3} + 5\(\cdot\) \chemfig{H_3C-[:-30]-[:+30]CH_3}
  + 2\(\cdot\)\chemfig{H_3C-[:+30]A(-[:+90]CH_3)-[:-30]CH_3}
  + 2\(\cdot\)\chemfig{H_3C-[:+30]Z(-[:+90]CH_3)-[:-30]CH_3}
  \(\longrightarrow\)
  
  \(\longrightarrow\) 
  2\(\cdot\)\chemfig{H_3C-[:+30]X-[:-30]-[:+30]CH_3} + 2\(\cdot\)\chemfig{H_3C-[:+30]Y-[:-30]-[:+30]CH_3} + 3\(\cdot\)\chemfig{H_3C-[:+30]A(-[:+90]CH_3)-[:-30]-[:+30]CH_3} + 3\(\cdot\)\chemfig{H_3C-[:+30]Z(-[:+90]CH_3)-[:-30]-[:+30]CH_3}
\end{center}

Энергетический эффект этих реакций, взятый с обратным знаком, оценивает общее напряжение молекулы относительно ациклических форм, считающихся свободными от напряжения. 

В частности, для 3,7-аналогов 1-азабицикло[3.3.1]нонана такое разделение связей описывается уравнением
\begin{center}
  \chemfig{[:-30]*6(-N(-[:180]-[:120]X-[:+60]-[:0]?)--Y---)} +
  + \chemfig{H_3C-[:-30]X-[:+30]CH_3} + \chemfig{H_3C-[:-30]Y-[:+30]CH_3}
  + 5\(\cdot\)\chemfig{H_3C-[:-30]-[:+30]CH_3}
  + 2\(\cdot\)\chemfig{H_3C-[:-30](-[:-90]CH_3)-[:+30]CH_3}
  + 2\(\cdot\)\chemfig{H_3C-[:-30]N(-[:-90]CH_3)-[:+30]CH_3}
  \(\longrightarrow\)
  
  \(\longrightarrow\) 
  2\(\cdot\)\chemfig{H_3C-[:+30]X-[:-30]-[:+30]CH_3} + 2\(\cdot\)\chemfig{H_3C-[:+30]Y-[:-30]-[:+30]CH_3} + 3\(\cdot\)\chemfig{H_3C-[:+30](-[:+90]CH_3)-[:-30]-[:+30]CH_3} + 
  3\(\cdot\)\chemfig{H_3C-[:+30]N(-[:+90]CH_3)-[:-30]-[:+30]CH_3}
\end{center}

Для аналогов 1,5-диаазабицикло[3.3.1]нонана
\begin{center}
  \chemfig{[:-30]*6(-N(-[:180]-[:120]X-[:+60]-[:0]\phantom{N}?)--Y--N-)}
  + \chemfig{H_3C-[:-30]X-[:+30]CH_3} + \chemfig{H_3C-[:-30]Y-[:+30]CH_3} + 5\(\cdot\) \chemfig{H_3C-[:-30]-[:+30]CH_3}
  + 2\(\cdot\)\chemfig{H_3C-[:+30](-[:+90]CH_3)-[:-30]CH_3}
  + 2\(\cdot\)\chemfig{H_3C-[:+30]N(-[:+90]CH_3)-[:-30]CH_3}
  \(\longrightarrow\)
  
  \(\longrightarrow\) 
  2\(\cdot\)\chemfig{H_3C-[:+30]X-[:-30]-[:+30]CH_3} + 2\(\cdot\)\chemfig{H_3C-[:+30]Y-[:-30]-[:+30]CH_3} + 
  6\(\cdot\)\chemfig{H_3C-[:+30]N(-[:+90]CH_3)-[:-30]-[:+30]CH_3}
\end{center}

Для более подробного рассмотрения отдельных взаимодействий необходимо комбинировать эти уравнения с другими.

Для расчётов энергий индивидуальных реагентов применяются неэмпирические (\emph{ab~initio}) методы. Они включают на первом этапе оптимизацию геометрии молекул. Далее расчёт собственных частот колебаний для конформеров. 

Часто несколько методов расчёта комбинируются в виде так называемых \emph{аддитивных схем}\dots Полная энергия $\Energytot$ и энергия нулевых колебаний $\EnergyZPE$

\begin{equation}
E_\varpi = \underbrace{{\overbrace{E_0^\infty + E_c^\infty}^{E_\infty}} + E_k}_{\sim\Energytot} + \EnergyZPE
\end{equation}

Ниже в таблице~\ref{tab:Reagents:Opt} на с.~\pageref{tab:Reagents:Opt}--\pageref{tab:Reagents:Opt:Ends} приводятся результаты оптимизации молекул-\tqt{реагентов} в приближении RI-MP2~\cite{MP:1934,RI:MP2:1997} с использованием расширенного корреляционно-согласованного трёхэкспоненциального базисного набора Даннинга (aug-cc-pVTZ).~\cite{Dunning:1989,Peterson:1993,Woon:1993} Если группа симметрии оптимальной структуры не указана явно, она является тривиальной~(\SymGroup{C}{1}). Все полученные в ходе расчёта собственные частоты молекул действительны.

%\caption{\label{tab:Desmotic:Reagents}Результаты расчёта реагентов для реакций разделения связей}

\begin{longtabu} to \textwidth {rr|SS|X[l]}
\multicolumn{5}{p{\textwidth}}{Таблица\label{tab:Reagents:Opt}~\arabic{table}. Строение молекул-\tqt{реагентов} для реакций разделения связей по данным неэмпирических расчётов} \\ %
\toprule\multicolumn{2}{c|}{Структура / симметрия}  & \(E_{2}\),~\si{\hartree} & \(E_{ZPE}\),~\si{\hartree} & Геометрия \endfirsthead
\multicolumn{5}{p{\textwidth}}{\emph{продолжение таблицы}~\arabic{table}} \\
\toprule\multicolumn{2}{c|}{Структура / симметрия} & \(E_{2}\),~\si{\hartree} & \(E_{ZPE}\),~\si{\hartree} & Геометрия \endhead
  \midrule\multicolumn{5}{l}{\emph{продолжение на следующей странице}} \endfoot
  \bottomrule\multicolumn{5}{l}{} \endlastfoot
\midrule\chemfig{H_3C-[:-30]-[:+30]CH_3} / \(\SymGroup{C}{2v}\) &\cmpd{Propane} & \num{-118.859638} & \num{0.104589} &
\ce{C-C}~\SI{1.523}{\angstrom}; $\angle$\ce{CCC}~\ang{112.3} \\
\midrule \chemfig{H_3C-[:-30]O-[:+30]CH_3} / \(\SymGroup{C}{2v}\) &\cmpd{Me2O} & \num{-154.737526} & \num{0.080693} & 
\ce{C-O}~\SI{1.411}{\angstrom};
$\angle$\ce{COC}~\ang{110.5} \\
\midrule \chemfig{H_3C-[:-30]S-[:+30]CH_3} / \(\SymGroup{C}{2v}\) & \cmpd{Me2S} & \num{-477.351949} & \num{0.076641} & 
\ce{C-S}~\SI{1.804}{\angstrom}; 
$\angle$\ce{CSC}~\ang{97.9} \\
%\midrule\cmpd{Dimethylamine} & \chemfig{H_3C-[:-30]N(-[:-90]H)-[:+30]CH_3} & \(\SymGroup{C}{s}\) & \\
\midrule \chemfig{H_3C-[:-30](-[:-90]CH_3)-[:+30]CH_3} / \(\SymGroup{C}{3v}\) & \cmpd{Isobutane} & \num{-158.087267} & \num{0.132478} & \ce{C-C}~\SI{1.524}{\angstrom}; $\angle$\ce{CCC}~\ang{110.7}\\
\midrule\chemfig{H_3C-[:-30]C(=[:-90]O)-[:+30]CH_3} / \(\SymGroup{C}{2v}\) & \cmpd{Acetone} &  \num{-192.789711} & \num{0.084256} & \ce{C=O}~\SI{1.220}{\angstrom}; 
\ce{C-C}~\SI{1.508}{\angstrom};
$\angle$\ce{OCC}~\ang{122.1};
$\angle$\ce{CCC}~\ang{115.8}\\
\midrule \chemfig{H_3C-[:-30]N(-[:-90]CH_3)-[:+30]CH_3} / \(\SymGroup{C}{3v}\) & \cmpd{Trimethylamine} & \num{-174.100139} & \num{0.121385} & 
\ce{C-N}~\SI{1.451}{\angstrom};
$\angle$\ce{CNC}~\ang{110.2} \\
\midrule\chemfig{H_3C-[:-30]-[:+30]-[:-30]CH_3} / \(\SymGroup{C}{2h}\) &\cmpd{Butane} &  \num{-158.084303} & \num{0.133309} & 
\ce{H_3C-CH_2}~\SI{1.524}{\angstrom}; 
\ce{H_2C-CH_2}~\SI{1.523}{\angstrom}; 
$\angle$\ce{CCC}~\ang{112.3} \\
\midrule\chemfig{H_3C-[:-30]O-[:+30]-[:-30]CH_3} / \(\SymGroup{C}{s}\) & \cmpd{MeOEt} &  \num{-193.966394} & \num{0.108895} & 
\ce{H3C-O}~\SI{1.412}{\angstrom};
\ce{H_2C-O}~\SI{1.417}{\angstrom};
\ce{H2C-CH_3}~\SI{1.509}{\angstrom};
$\angle$\ce{COC}~\ang{111.2};
$\angle$\ce{OCC}~\ang{108.1} \\
\midrule\chemfig{H_3C-[:-30]S-[:+30]-[:-30]CH_3} / \(\SymGroup{C}{s}\) & \cmpd{MeSEt} &  \num{-516.576661} & \num{0.105402} &
\ce{H_3C-S}~\SI{1.805}{\angstrom}; 
\ce{H_2C-S}~\SI{1.809}{\angstrom}; 
\ce{H_2C-CH_3}~\SI{1.519}{\angstrom}; 
$\angle$\ce{CSC}~\ang{98.7}; 
$\angle$\ce{SCC}~\ang{109.3} \\
\midrule\chemfig{H_3C-[:-30]C(=[:-90]O)-[:+30]-[:-30]CH_3} / \(\SymGroup{C}{s}\) & \cmpd{Butanone} & \num{-232.014552} & \num{0.113700} & \ce{C=O}~\SI{1.220}{\angstrom};
\ce{H_3C-CO}~\SI{1.509}{\angstrom}; 
\ce{H_2C-CO}~\SI{1.513}{\angstrom}; 
\ce{H_3C-CH_2}~\SI{1.517}{\angstrom};
$\angle\ce{OCC(H_3)}~\ang{121.9}$;
$\angle\ce{OCC(H_2)}~\ang{122.1}$; 
$\angle\ce{CC(O)C}~\ang{116.0}$; 
$\angle\ce{C(O)CC}~\ang{113.4}$;
%$\tau_{\ce{OCCC}}\approx\ang{0}$ 
\\
\midrule\chemfig{H_3C-[:-30](-[:-90]CH_3)-[:+30]-[:-30]CH_3}  & \cmpd{Isopentane} & \num{-197.311142} & \num{0.161457} & \\
\midrule\chemfig{H_3C-[:-30](-[:-90]CH_3)-[:+30]O-[:-30]CH_3} & \cmpd{Me2CHOMe} & \num{-233.195233} & \num{0.137450} &  \\
\midrule\chemfig{H_3C-[:-30](-[:-90]CH_3)-[:+30]S-[:-30]CH_3} & \cmpd{Me2CHSMe} & & & \\
\midrule \chemfig{H_3C-[:-30]N(-[:-90]CH_3)-[:+30]-[:-30]CH_3} & \cmpd{Me2NEt} & & & \\
%\midrule\cmpd{Isopentanone} &\chemfig{H_3C-[:+30](-[:+90]CH_3)-[:-30]C(=[:-90]O)-[:+30]CH_3} & & & \\
\end{longtabu}\label{tab:Reagents:Opt:Ends}

За исключением простейших структур \cmpd{Propane}--\cmpd{Acetone}, молекулы реагентов обладают конформационной подвижностью. Инверсия атома азота триметиламина~\cmpd{Trimethylamine} взаимодействует со внутренним вращением метильных групп. 
%Внутреннее вращение \emph{н}-бутана~\cmpd{Butane} и его аналогов\dots

\pgfplotstablecreatecol[%
create col/assign/.code={%
\getthisrow{Etot}\entryval%
%\ifnum\pgfplotstablerow>0
\pgfmathsetmacro{\entryval}{min(\entryval,\pgfmathaccuma)}%
%\fi
\edef\pgfmathaccuma{\entryval}
\pgfkeyslet{/pgfplots/table/create col/next content}\entryval
}
]{Emin}\ScanButane %
\pgfplotstablecreatecol[%
create col/assign/.code={%
  \getthisrow{Etot}\entryval%
  \getthisrow{Emin}\refval%
  %\ifnum\pgfplotstablerow>0
  \pgfmathsetmacro{\entryval}{\entryval - \refval}%
  \pgfmathsetmacro{\entryval}{\AUtoKJM * \entryval}%
  %\fi
  \edef\pgfmathaccuma{\entryval}
  \pgfkeyslet{/pgfplots/table/create col/next content}\entryval
}
]{Erel}\ScanButane %
\pgfplotstablesave[columns={Tau,Etot,Erel}]{\ScanButane}{Results/ScanButane.save.dat}

\pgfplotstablecreatecol[%
create col/assign/.code={%
  \getthisrow{Etot}\entry%
  \ifnum\pgfplotstablerow>0
  \pgfmathsetmacro{\entry}{min(\entry,\pgfmathaccuma)}%
  \fi
  \edef\pgfmathaccuma{\entry}
  \pgfkeyslet{/pgfplots/table/create col/next content}\entry
}]{Emin}\ScanMeOEt %
\pgfplotstablecreatecol[%
create col/assign/.code={%
  \getthisrow{Etot}\entryval%
  \getthisrow{Emin}\refval%
  %\ifnum\pgfplotstablerow>0
  \pgfmathsetmacro{\entryval}{\entryval - \refval}%
  \pgfmathsetmacro{\entryval}{\AUtoKJM * \entryval}%
  %\fi
  \edef\pgfmathaccuma{\entryval}
  \pgfkeyslet{/pgfplots/table/create col/next content}\entryval
}
]{Erel}\ScanMeOEt %
\pgfplotstablesave[columns={Tau,Etot,Erel}]{\ScanMeOEt}{Results/ScanMeOEt.save.dat}

\pgfplotstablecreatecol[%
create col/assign/.code={%
  \getthisrow{Etot}\entry%
  \ifnum\pgfplotstablerow>0
  \pgfmathsetmacro{\entry}{min(\entry,\pgfmathaccuma)}%
  \fi
  \edef\pgfmathaccuma{\entry}
  \pgfkeyslet{/pgfplots/table/create col/next content}\entry
}]{Emin}\ScanMeSEt %
\pgfplotstablecreatecol[%
create col/assign/.code={%
  \getthisrow{Etot}\entryval%
  \getthisrow{Emin}\refval%
  %\ifnum\pgfplotstablerow>0
  \pgfmathsetmacro{\entryval}{\entryval - \refval}%
  \pgfmathsetmacro{\entryval}{\AUtoKJM * \entryval}%
  %\fi
  \edef\pgfmathaccuma{\entryval}
  \pgfkeyslet{/pgfplots/table/create col/next content}\entryval
}
]{Erel}\ScanMeSEt %
\pgfplotstablesave[columns={Tau,Etot,Erel}]{\ScanMeSEt}{Results/ScanMeSEt.save.dat}

\pgfplotstablecreatecol[%
create col/assign/.code={%
  \getthisrow{Etot}\entry%
  \ifnum\pgfplotstablerow>0
  \pgfmathsetmacro{\entry}{min(\entry,\pgfmathaccuma)}%
  \fi
  \edef\pgfmathaccuma{\entry}
  \pgfkeyslet{/pgfplots/table/create col/next content}\entry
}]{Emin}\ScanMeCOEt %
\pgfplotstablecreatecol[%
create col/assign/.code={%
  \getthisrow{Etot}\entryval%
  \getthisrow{Emin}\refval%
  %\ifnum\pgfplotstablerow>0
  \pgfmathsetmacro{\entryval}{\entryval - \refval}%
  \pgfmathsetmacro{\entryval}{\AUtoKJM * \entryval}%
  %\fi
  \edef\pgfmathaccuma{\entryval}
  \pgfkeyslet{/pgfplots/table/create col/next content}\entryval
}
]{Erel}\ScanMeCOEt %
\pgfplotstablesave[columns={Tau,Etot,Erel}]{\ScanMeCOEt}{Results/ScanMeCOEt.save.dat}

\pgfplotstablecreatecol[%
create col/assign/.code={%
  \getthisrow{Etot}\entry%
  \ifnum\pgfplotstablerow>0
  \pgfmathsetmacro{\entry}{min(\entry,\pgfmathaccuma)}%
  \fi
  \edef\pgfmathaccuma{\entry}
  \pgfkeyslet{/pgfplots/table/create col/next content}\entry
}]{Emin}\ScanIsoPentane %
\pgfplotstablecreatecol[%
create col/assign/.code={%
  \getthisrow{Etot}\entryval%
  \getthisrow{Emin}\refval%
  %\ifnum\pgfplotstablerow>0
  \pgfmathsetmacro{\entryval}{\entryval - \refval}%
  \pgfmathsetmacro{\entryval}{\AUtoKJM * \entryval}%
  %\fi
  \edef\pgfmathaccuma{\entryval}
  \pgfkeyslet{/pgfplots/table/create col/next content}\entryval
}
]{Erel}\ScanIsoPentane %
\pgfplotstablesave[columns={Tau,Etot,Erel}]{\ScanIsoPentane}{Results/ScanIsoPentane.save.dat}

\pgfplotstablecreatecol[%
create col/assign/.code={%
  \getthisrow{Etot}\entry%
  \ifnum\pgfplotstablerow>0
  \pgfmathsetmacro{\entry}{min(\entry,\pgfmathaccuma)}%
  \fi
  \edef\pgfmathaccuma{\entry}
  \pgfkeyslet{/pgfplots/table/create col/next content}\entry
}]{Emin}\ScanIPrOMe %
\pgfplotstablecreatecol[%
create col/assign/.code={%
  \getthisrow{Etot}\entryval%
  \getthisrow{Emin}\refval%
  %\ifnum\pgfplotstablerow>0
  \pgfmathsetmacro{\entryval}{\entryval - \refval}%
  \pgfmathsetmacro{\entryval}{\AUtoKJM * \entryval}%
  %\fi
  \edef\pgfmathaccuma{\entryval}
  \pgfkeyslet{/pgfplots/table/create col/next content}\entryval
}
]{Erel}\ScanIPrOMe %
\pgfplotstablesave[columns={Tau,Etot,Erel}]
{\ScanIPrOMe}{Results/ScanIPrOMe.save.dat}
%
\pgfplotstablecreatecol[%
create col/assign/.code={%
  \getthisrow{Etot}\entry%
  \ifnum\pgfplotstablerow>0
  \pgfmathsetmacro{\entry}{min(\entry,\pgfmathaccuma)}%
  \fi
  \edef\pgfmathaccuma{\entry}
  \pgfkeyslet{/pgfplots/table/create col/next content}\entry
}]{Emin}\ScanIPrSMe %
\pgfplotstablecreatecol[%
create col/assign/.code={%
  \getthisrow{Etot}\entryval%
  \getthisrow{Emin}\refval%
  %\ifnum\pgfplotstablerow>0
  \pgfmathsetmacro{\entryval}{\entryval - \refval}%
  \pgfmathsetmacro{\entryval}{\AUtoKJM * \entryval}%
  %\fi
  \edef\pgfmathaccuma{\entryval}
  \pgfkeyslet{/pgfplots/table/create col/next content}\entryval
}
]{Erel}\ScanIPrSMe %
\pgfplotstablesave[columns={Tau,Etot,Erel}]
{\ScanIPrSMe}{Results/ScanIPrSMe.save.dat}
%
\pgfplotstablecreatecol[%
create col/assign/.code={%
  \getthisrow{Etot}\entry%
  \ifnum\pgfplotstablerow>0
  \pgfmathsetmacro{\entry}{min(\entry,\pgfmathaccuma)}%
  \fi
  \edef\pgfmathaccuma{\entry}
  \pgfkeyslet{/pgfplots/table/create col/next content}\entry
}]{Emin}\ScanMeMeNEt %
\pgfplotstablecreatecol[%
create col/assign/.code={%
  \getthisrow{Etot}\entryval%
  \getthisrow{Emin}\refval%
  %\ifnum\pgfplotstablerow>0
  \pgfmathsetmacro{\entryval}{\entryval - \refval}%
  \pgfmathsetmacro{\entryval}{\AUtoKJM * \entryval}%
  %\fi
  \edef\pgfmathaccuma{\entryval}
  \pgfkeyslet{/pgfplots/table/create col/next content}\entryval
}
]{Erel}\ScanMeMeNEt %
\pgfplotstablesave[columns={Tau,Etot,Erel}]
{\ScanMeMeNEt}{Results/ScanMeMeNEt.save.dat}

\begin{center}
\caption{Потенциалы внутреннего вращения молекул \cmpd{Butane}--\cmpd{Butanone}\label{fig:Internal:Rotation:Butanes}}

\centerfloat{
\begin{tikzpicture}
  \begin{axis} [%
    %title={\chemfig{H_3C-[:+30]\circ-[:-30]\circ-[:+30]CH_3}},
    %legend pos=outer north east,
    legend style={draw=none},
    xlabel={\chemfig{[,0.75]H_3C-[:+30]\circ-[:-30]\circ-[:+30]CH_3} / $\tau$,~\textdegree},
    xmin=-180, xmax=180,
    xtick={-180,-120,...,180},
    ylabel={$\Delta\Func{E}$,~\si{\kilo\joule\per\mole}},
    ymin=-1, ymax=32]
    \addplot [black,thick,smooth,domain=-180:180] table [x=Tau,y=Erel] 
    {Results/ScanButane.save.dat};\addlegendentry{\chemfig{[,0.75]-[:+30]-[:-30]-[:+30]}~\cmpd{Butane}}
    \addplot [red,thick,smooth,domain=-180:180] table [x=Tau,y=Erel]
    {Results/ScanMeOEt.save.dat};\addlegendentry{\chemfig{[,0.75]-[:+30]-[:-30]O-[:+30]}~\cmpd{MeOEt}}
    \addplot [brown,thick,smooth,domain=-180:180] table [x=Tau,y=Erel] 
    {Results/ScanMeSEt.save.dat};\addlegendentry{\chemfig{[,0.75]-[:+30]-[:-30]S-[:+30]}~\cmpd{MeSEt}}
    \addplot [green,thick,smooth,domain=-180:180] table [x=Tau,y=Erel] 
    {Results/ScanMeCOEt.save.dat}; \addlegendentry{\chemfig{[,0.75]-[:+30]-[:-30]C(=[:-90]O)-[:+30]}~\cmpd{Butanone}}
  \end{axis}
\end{tikzpicture}
}
\end{center}

\begin{center}
\caption{Потенциалы внутреннего вращения молекул \cmpd{Isopentane}--\cmpd{Me2NEt}\label{fig:Internal:Rotation:Isopentanes}}

  \centerfloat{
\begin{tikzpicture}
  \begin{axis} [%
    legend style={draw=none},
    xlabel={\chemfig{[,0.75]H_3C-[:+30]\bullet(-[:+90]CH_3)-[:-30]\circ-[:+30]CH_3}/$\tau$,~\textdegree},
    xmin=-180, xmax=180,
    xtick={-180,-120,...,180},
    ylabel={$\Delta\Func{E}$,~\si{\kilo\joule\per\mole}},
    ymin=-1, ymax=32]
    \addplot [black,thick,smooth,domain=-180:180] table [x=Tau,y=Erel] 
    {Results/ScanIsopentane.save.dat};\addlegendentry{\chemfig{[,0.625]-[:+30](-[:+90])-[:-30]-[:+30]}~\cmpd{Isopentane}}
    \addplot [red,thick,smooth,domain=-180:180] table [x=Tau,y=Erel] 
    {Results/ScanIPrOMe.save.dat};\addlegendentry{\chemfig{[,0.625]-[:+30](-[:+90])-[:-30]O-[:+30]}~\cmpd{Me2CHOMe}}
    \addplot [brown,thick,smooth,domain=-180:180] table [x=Tau,y=Erel] 
    {Results/ScanIPrSMe.save.dat};\addlegendentry{\chemfig{[,0.625]-[:+30](-[:+90])-[:-30]S-[:+30]}~\cmpd{Me2CHSMe}}
    \addplot [blue,thick,smooth,domain=-180:180] table [x=Tau,y=Erel] 
    {Results/ScanMeMeNEt.save.dat};\addlegendentry{\chemfig{[,0.625]-[:+30]N(-[:+90])-[:-30]-[:+30]}~\cmpd{Me2NEt}}
\end{axis}
\end{tikzpicture}%
}
\end{center}

\pgfplotstableset{
%columns/TauAll/.style = {column name = $\tau$,~\textdegree},
%columns/EButane/.style = {column name = \cmpd{Butane}},0
%columns/EMeOEt/.style = {column name = $\Delta\Func{E}$~(\cmpd{MeOEt})},
}

\pgfplotstablenew[
create on use/Tau/.style = {create col/copy column from table={\ScanButane}{Tau}},
create on use/EButane/.style = {create col/copy column from table={\ScanButane}{Erel}},
create on use/EMeOEt/.style = {create col/copy column from table={\ScanMeOEt}{Erel}},
create on use/EMeSEt/.style = {create col/copy column from table={\ScanMeSEt}{Erel}},
create on use/EMeCOEt/.style = {create col/copy column from table={\ScanMeCOEt}{Erel}},
create on use/EIsoPentane/.style = {create col/copy column from table={\ScanIsoPentane}{Erel}},
create on use/EIPrOMe/.style = {create col/copy column from table={\ScanIPrOMe}{Erel}},
create on use/EIPrSMe/.style = {create col/copy column from table={\ScanIPrSMe}{Erel}},
create on use/EDMeNEt/.style = {create col/copy column from table={\ScanMeMeNEt}{Erel}},
columns={Tau,EButane,EMeOEt,EMeSEt,EMeCOEt,EIsoPentane,EIPrOMe,EIPrSMe,EDMeNEt}]{25}\ScanReagents

%\begin{table}
%\label{tab:Rot:Energy:Reagents} 
%\caption{Внутреннее вращение молекул~\cmpd{Butane} -- \cmpd{Me2NEt},~\si{\kilo\joule\per\mole} }
%\centering
%\pgfplotstabletypeset[column type=r,fixed, fixed zerofill, precision=1,
%columns/EButane/.style = {column name = \cmpd{Butane}},
%columns/EMeOEt/.style = {column name = \cmpd{MeOEt}},
%columns/EMeSEt/.style = {column name = \cmpd{MeSEt}},
%columns/EMeCOEt/.style = {column name = \cmpd{Butanone}},
%columns/EIsoPentane/.style = {column name = \cmpd{Isopentane}},
%columns/EIPrOMe/.style = {column name = \cmpd{Me2CHOMe}},
%columns/EIPrSMe/.style = {column name = \cmpd{Me2CHSMe}},
%columns/EDMeNEt/.style = {column name = \cmpd{Me2NEt}},
%]\ScanReagents
%\end{table}

Глобально оптимальными, судя по~рис.~\ref{fig:Internal:Rotation:Butanes}--\ref{fig:Internal:Rotation:Isopentanes}, ожидаемо являются \emph{транс}- или \emph{транс,~гош}-конформеры с $\tau\approx\ang{180}$. В дальнейшем во всех случаях для расчётов энергетических эффектов реакций разделения связей были использованы оптимизированные значения энергии именно для таких структур.


\section{Напряжения свободных шестичленных циклов}

Реакция разделения связей для 1,3- и 1,4-дизамещённых аналогов формально одинаковы:

\begin{center}
  \begin{equation*} \underbrace{\text{\chemfig{[:-30]X*6(---Y---)}} \qquad \text{\chemfig{[:-30]*6(-X----Y-)}}}_{\text{как бы это всё поизяЧнее сделать?}} \end{equation*}
+ \chemfig{H_3C-[:-30]X-[:+30]CH_3} + \chemfig{H_3C-[:-30]Y-[:+30]CH_3} + 4\(\cdot\) \chemfig{H_3C-[:-30]-[:+30]CH_3} \(\longrightarrow\)
  
  \(\longrightarrow\) 
    2\(\cdot\)\chemfig{H_3C-[:+30]X-[:-30]-[:+30]CH_3} +     2\(\cdot\)\chemfig{H_3C-[:+30]-[:-30]-[:+30]CH_3} + 2\(\cdot\)\chemfig{H_3C-[:+30]Y-[:-30]-[:+30]CH_3}
\end{center}

Реакции \emph{позиционного диспропорционирования} дизамещённых аналогов циклогексана

\begin{center}
  \chemfig{*6(-X---Y--)} + \chemfig{*6(------)} \(\longrightarrow\) \chemfig{*6(-X-----)} + \chemfig{*6(----Y--)}
\end{center}
  
B таблице~\ref{tab:Cycle:Six:Opt} на с.~\pageref{tab:Cycle:Six:Opt}--\pageref{tab:Cycle:Six:Opt:Ends} приводятся результаты оптимизации молекул производных циклогексана, включая эндоциклические торсионные углы $\tau$.

\begin{center}
  \chemfig{*6(------)} \quad\chemfig{*6(-X-----)} \quad\chemfig{*6(---Y--X-)} \quad\chemfig{*6(-X---Y--)} \quad\chemfig{*6(-Z--Y--X-)}
\end{center}

\begin{longtabu} to \textwidth {rc|SS|X[l]}
\multicolumn{5}{p{\textwidth}}{Таблица\label{tab:Cycle:Six:Opt}~\arabic{table}. Строение молекул циклогексана и его аналогов по данным неэмпирических расчётов} \\ %  
\toprule\multicolumn{2}{c|}{Структура}  & \(E_{2}\),~\si{\hartree} & \(E_{ZPE}\),~\si{\hartree} & Геометрия \endfirsthead
  \multicolumn{5}{p{\textwidth}}{\emph{продолжение таблицы}~\thetable:} \\
  \toprule\multicolumn{2}{c|}{Структура} & \(E_{2}\),~\si{\hartree} & \(E_{ZPE}\),~\si{\hartree} & Геометрия\endhead
  \midrule\multicolumn{5}{l}{\emph{продолжение на следующей странице}}\endfoot
  \bottomrule\multicolumn{5}{l}{}\endlastfoot
  \midrule\cmpd{Cyclohexane} & \chemfig{?-[:+160]-[:-120]<[:+20]-[:-20,,,,line width=\boldbondwidth]>[:+60]?} / \SymGroup{D}{3d} &  \num{-235.344605} & \num{0.171358}& \ce{C-C}~\SI{1.526}{\angstrom}; $\angle\ce{CCC}~\ang{110.7}$; $\tau=\pm\ang{56.9}$ 
  \\ \bottomrule \multicolumn{5}{l}{Монозамещённые (\emph{пентаметиленовые}) производные~(\SymGroup{C}{s}) } \\
\toprule\cmpd{Pyran} & \chemfig{?-[:+160]-[:-120]O<[:+20]-[:-20,,,,line width=\boldbondwidth]>[:+60]?} &  \num{-271.229641} & \num{0.147833} & $\angle\ce{COC}~\ang{110.9}$; $\angle\ce{OCC}~\ang{111.9}$; $\angle\ce{CC_3C}~\ang{109.9}$; $\angle\ce{CC_4C}~\ang{109.1}$; \\
\midrule\cmpd{Thiopyran} & \chemfig{?-[:+160]-[:-120]S<[:+20]-[:-20,,,,line width=\boldbondwidth]>[:+60]?} & \num{-593.841537}  & \num{0.143971} & \chemfig{[-30,1.5]*6(@{6}-@{1}S-@{2}-@{3}-@{4}-@{5}-)} \\
\midrule\cmpd{Cyclohexanone} & \chemfig{?-[:+160]-[:-120]C(=[:-135,0.875]O)<[:+20]-[:-20,,,,line width=\boldbondwidth]>[:+60]?} & & & \ce{C-C} \\
\midrule\cmpd{MeCyclohexaneEq} & \chemfig{?-[:+160]-[:-120](-[:+155]H_3C)(-[:-90,0.75]H)<[:+20]-[:-20,,,,line width=\boldbondwidth]>[:+60]?} & & & \ce{C-C} \\
\midrule\cmpd{NMePiperidineEq} & \chemfig{?-[:+160]-[:-120]N(-[:+155]H_3C)<[:+20]-[:-20,,,,line width=\boldbondwidth]>[:+60]?} & & & \ce{C-C} \\ \bottomrule
\multicolumn{5}{l}{Симметричные 1,3-дизамещённые производные~(\ce{X}=\ce{Y},~\SymGroup{C}{s}) } \\ 
\toprule\cmpd{Dioxane13} & \chemfig{?-[:+160]-[:-120]O<[:+20]-[:-20,,,,line width=\boldbondwidth]O>[:+60]?} & & & \ce{C-C} \\ 
\midrule\cmpd{Dithiane13} & \chemfig{?-[:+160]-[:-120]S<[:+20]-[:-20,,,,line width=\boldbondwidth]S>[:+60]?} & & & \ce{C-C} \\
\midrule\cmpd{N2Me2Diazine13} & \chemfig{?-[:+160]-[:-120]N(-[:+160]H_3C)<[:+20]-[:-20,,,,line width=\boldbondwidth]N(-[:+20]CH_3)>[:+60]?} & & & \ce{C-C} \\
\bottomrule
\multicolumn{5}{l}{Симметричные 1,4-дизамещённые производные~(\ce{X}=\ce{Y},~\SymGroup{C}{2h}) } \\
\toprule\cmpd{Dioxane14} & \chemfig{?-[:+160]-[:-120]O<[:+20]-[:-20,,,,line width=\boldbondwidth]>[:+60]O?} & & & \ce{C-C} \\
\midrule\cmpd{Dithiane14} & \chemfig{?-[:+160]-[:-120]S<[:+20]-[:-20,,,,line width=\boldbondwidth]>[:+60]S?} & & & \ce{C-C} \\
\midrule\cmpd{Cyclohexandione14} & \chemfig{?-[:-120]C(=[:-135,0.875]O)<[:+20]-[:-20,,,,line width=\boldbondwidth]>[:+60]C(=[:+45,0.875]O)-[:-160]?} & & & \ce{C-C} \\
\midrule\cmpd{Me2CyclohexaneEE} & \chemfig{?-[:-120](-[:+155]H_3C)(-[:-90,0.75]H)<[:+20]-[:-20,,,,line width=\boldbondwidth]>[:+60](-[:+90,0.75]H)(-[:-25]CH_3)-[:-160]?} & & & \ce{C-C} \\
\midrule\cmpd{N2Me2PiperazineEE} & \chemfig{?-[:-120]N(-[:+155]H_3C)<[:+20]-[:-20,,,,line width=\boldbondwidth]>[:+60]N(-[:-25]CH_3)-[:-160]?} & & & \ce{C-C} \\
\bottomrule\multicolumn{5}{l}{Несимметричные 1,4-дизамещённые производные~(\ce{X}$\ne$\ce{Y},~\SymGroup{C}{s}) } \\
\toprule\cmpd{Oxathiane4} & \chemfig{?-[:+160]-[:-120]S<[:+20]-[:-20,,,,line width=\boldbondwidth]>[:+60]O?} & & & \ce{C-C} \\
\midrule \cmpd{Pyranone4} & \chemfig{?-[:+160]-[:-120]C(=[:-135,0.875]O)<[:+20]-[:-20,,,,line width=\boldbondwidth]>[:+60]O?} & & & \ce{C-C} \\
\midrule\cmpd{Me4PyraneEq} & \chemfig{?-[:+160]-[:-120](-[:+155]H_3C)(-[:-90,0.75]H)<[:+20]-[:-20,,,,line width=\boldbondwidth]>[:+60]O?} & & & \ce{C-C} \\
\midrule\cmpd{NMeMorpholineEq} & \chemfig{?-[:+160]-[:-120]N(-[:+155]H_3C)<[:+20]-[:-20,,,,line width=\boldbondwidth]>[:+60]O?} & & & \ce{C-C} \\
\midrule \cmpd{Thiopyranone4} & \chemfig{?-[:+160]-[:-120]C(=[:-135,0.875]O)<[:+20]-[:-20,,,,line width=\boldbondwidth]>[:+60]S?} && & \ce{C-C} \\
\midrule\cmpd{Me4ThiopyraneEq} & \chemfig{?-[:+160]-[:-120](-[:+155]H_3C)(-[:-90,0.75]H)<[:+20]-[:-20,,,,line width=\boldbondwidth]>[:+60]S?}  & & & \ce{C-C} \\
\midrule\cmpd{NMeThiomorpholineEq} & \chemfig{?-[:+160]-[:-120]N(-[:+155]H_3C)<[:+20]-[:-20,,,,line width=\boldbondwidth]>[:+60]S?} & & & \ce{C-C} \\
\midrule\cmpd{Me4CyclohexanoneEq} & \chemfig{?-[:-120]C(=[:-135,0.875]O)<[:+20]-[:-20,,,,line width=\boldbondwidth]>[:+60](-[:+90,0.75]H)(-[:-25]CH_3)-[:-160]?} & & & \ce{C-C} \\
\midrule\cmpd{NMePiperidone4Eq} & \chemfig{?-[:-120]C(=[:-135,0.875]O)<[:+20]-[:-20,,,,line width=\boldbondwidth]>[:+60]N(-[:-25]CH_3)-[:-160]?} & & & \ce{C-C} \\
\midrule\cmpd{4MePiperidineEE} & \chemfig{?-[:-120]N(-[:+155]H_3C)<[:+20]-[:-20,,,,line width=\boldbondwidth]>[:+60](-[:+90,0.75]H)(-[:-25]CH_3)-[:-160]?} & & & \ce{C-C} \\
\bottomrule\multicolumn{5}{l}{Симметричные 1,3,5-тризамещённые производные~(\ce{X}=\ce{Y}=\ce{Z},~\SymGroup{C}{3v}) } \\
\toprule\cmpd{STrioxane} & \chemfig{O?-[:+160]-[:-120]O<[:+20]-[:-20,,,,line width=\boldbondwidth]O>[:+60]?} & \num{} & & \ce{C-O} \\
\midrule\cmpd{STrithiane} & \chemfig{S?-[:+160]-[:-120]S<[:+20]-[:-20,,,,line width=\boldbondwidth]S>[:+60]?} & & & \ce{C-S} \\
\bottomrule\multicolumn{5}{l}{Несимметричные 1,3,5-тризамещённые производные~(\ce{X}=\ce{Y},~\SymGroup{C}{s}) } \\
\toprule\cmpd{Thiodioxane35} & \chemfig{S?-[:+160]-[:-120]O<[:+20]-[:-20,,,,line width=\boldbondwidth]O>[:+60]?} & & & \ce{C-C} \\ 
\midrule\cmpd{Oxa5dithiane13} & \chemfig{O?-[:+160]-[:-120]S<[:+20]-[:-20,,,,line width=\boldbondwidth]S>[:+60]?} & & & \ce{C-C} \\
\midrule\cmpd{N2Me2Oxa5diazine13} & \chemfig{O?-[:+160]-[:-120]N(-[:+160]H_3C)<[:+20]-[:-20,,,,line width=\boldbondwidth]N(-[:+20]CH_3)>[:+60]?} & & & \ce{C-C} \\
\midrule\cmpd{N2Me2Thia5diazine13} & \chemfig{S?-[:+160]-[:-120]N(-[:+160]H_3C)<[:+20]-[:-20,,,,line width=\boldbondwidth]N(-[:+20]CH_3)>[:+60]?} & & & \ce{C-C} \\
\bottomrule
\end{longtabu}\label{tab:Cycle:Six:Opt:Ends}

Длины связей~(\AA) и эндоциклические двугранные углы

\vspace{\medskipamount}
\chemfig{[,1.75]*6(@{6}-@{1}-@{2}-@{3}-@{4}-@{5}-)} %
\namebond{3}{4}{\small\ang{56.9}} %
\namebond{1}{2}{\small\SI{1.526}{\angstrom}} %
\quad
\chemfig{[,1.75]*6(@{6}-@{1}O-@{2}-@{3}-@{4}-@{5}-)} %
\namebond{1}{2}{\small\ang{60.5}} %
\namebond{2}{3}{\small\ang{-57.4}} %
\namebond{3}{4}{\small\ang{53.4}} %
\namebond{5}{4}{\small\SI{1.526}{\angstrom}} % 
\namebond{6}{5}{\small\SI{1.520}{\angstrom}} % 
\namebond{6}{1}{\small\SI{1.422}{\angstrom}} %
\quad 
\chemfig{[,1.75]*6(@{6}-@{1}S-@{2}-@{3}-@{4}-@{5}-)}
\namebond{1}{2}{\small\ang{56.3}} 
\namebond{2}{3}{\small\ang{-62.1}}
\namebond{3}{4}{\small\ang{59.6}}
\namebond{5}{4}{\small\SI{1.525}{\angstrom}}
\namebond{6}{5}{\small\SI{1.521}{\angstrom}}
\namebond{6}{1}{\small\SI{1.814}{\angstrom}}
\quad

%\begin{longtabu} to \textwidth {X} \end{longtabu}

\vspace{\medskipamount}
Энергии напряжения... $\EStrain$

