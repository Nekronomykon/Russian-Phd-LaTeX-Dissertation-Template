% !TeX spellcheck = ru_RU
% !TeX encoding = UTF-8
\chapter{Исследование напряжений в производных циклогексана и бицикло[3.3.1]нонана}\label{ch:Strain:331}

Байеровское напряжение...

Ранние работы по анализу напряжений в связи со структурой алмаза и адамантана\dots~\cite{Mohr:1918}

\section{Реакции разделения связей} 

Наименьшую погрешность обеспечивают \emph{гипергомодесмотические} реакции, сохраняющие не только состав, как гомодесмические... но и число связей различных типов. В частности, для 3,7,9-замещённых аналогов бицикло[3.3.1]нонана выводится уравнение
\begin{center}
\chemfig{[:-30]Z*6(-(-[:180]-[:120]X-[:+60]-[:0]?)--Y---)} + \DrawMeXMe{} + \DrawMeYMe{} + \DrawMeZMe{} +

+ 4~\DrawPropane{} + 4~\DrawIsoButane{}
  \(\longrightarrow\)
  
\(\longrightarrow\) 2~\DrawHeteroButane{}{X}{} + 2~\DrawHeteroButane{}{Y}{} +

+ 4~\DrawIsoPentane{} + 2~\DrawHeteroIsoPentane{}{}{Z}
\end{center}

Для 1,3,5,7-аналогов исследуемой бициклической системы
\begin{center}
  \chemfig{[:-30]*6(-A(-[:180]-[:120]X-[:+60]-[:0]\phantom{Z}?)--Y--Z-)} + \DrawMeXMe{} + \DrawMeYMe{} + 5~\DrawPropane{} + 2~\DrawTrimethyl{}{A} + 2~\DrawTrimethyl{}{Z}
  \(\longrightarrow\)
  
  \(\longrightarrow\) 2~\DrawHeteroButane{}{X}{} + 2~\DrawHeteroButane{}{Y}{} +
  
  +3~\DrawHeteroIsoPentane{}{A}{} + 3~\DrawHeteroIsoPentane{}{Z}{}
\end{center}

Энергетический эффект этих реакций, взятый с обратным знаком, оценивает общее напряжение молекулы относительно ациклических форм, считающихся свободными от напряжения. 

В частности, для 3,7-аналогов 1-азабицикло[3.3.1]нонана~\cmpd{Bicycle331:1N} такое разделение связей описывается уравнением
\begin{center}
  \chemfig{[:-30]*6(-N(-[:180]-[:120]X-[:+60]-[:0]?)--Y---)}
  + \DrawMeXMe{} + \DrawMeYMe{} + 5~\DrawPropane{}
  + 2~\DrawIsoButane{} + 2~\DrawMeNMeMe{}
  \(\longrightarrow\)
  
  \(\longrightarrow\) 
  2~\DrawHeteroButane{}{X}{} + 2~\DrawHeteroButane{}{Y}{} + 3~\DrawIsoPentane{}
  3~\DrawMeNMeEt{}
\end{center}

Для аналогов 1,5-диаазабицикло[3.3.1]нонана~\cmpd{Bicycle331:15N2},~\cmpd{Bicycle331:15N2:37O2},~\cmpd{Bicycle331:15N2:37S2}
\begin{center}
\chemfig{[:-30]*6(-N(-[:180]-[:120]X-[:+60]-[:0]\phantom{N}?)--Y--N-)}
+ \DrawMeXMe{} + \DrawMeYMe{} + 5~\DrawPropane{} + 2~\DrawIsoButane{} + 2~\DrawMeNMeMe{}
  \(\longrightarrow\)
  
\(\longrightarrow\) 2~\DrawHeteroButane{}{X}{} + 2~\DrawHeteroButane{}{Y}{} + 6~\DrawMeNMeEt{}
\end{center}

Для более подробного рассмотрения отдельных взаимодействий необходимо комбинировать эти уравнения с другими.

Для расчётов энергий индивидуальных реагентов применяются неэмпирические (\emph{ab~initio}) методы. Они включают на первом этапе оптимизацию геометрии молекул. Далее расчёт собственных частот колебаний для конформеров. 

Часто несколько методов расчёта комбинируются в виде так называемых \emph{аддитивных схем}. Основу энергии такой схемы~$\Escheme{}$ составляет электронная энергия~$\Energytot$, вычисляемая, как правило, в приближении, учитывающем значительную долю корреляционной энергии в виде предела полного базисного набора~$E_\infty^{(A)}$, а также энергия нулевых колебаний (обычно в гармоническом приближении, где $\EnergyZPE = \frac{\hbar}{2} \sum_a\omega_a$, $\omega_a$ "--- частота $a$-й моды собственных колебаний молекулы), дополняется корреляционной энергией~$E_k(B)$, относящейся к более высокому теоретическому уровню описания.  выполненном .

\begin{equation}
\Escheme = \underbrace{{\overbrace{E_0^\infty + E_c^\infty{(A)}}^{E_\infty^{(A)}}} + E_k(B)}_{\sim\Energytot} + \EnergyZPE
\end{equation}

\section{Строение и конформационное поведение молекул-реагентов для реакций разделения связей}

Ниже в таблице~\ref{tab:Reagents:Opt} на с.~\pageref{tab:Reagents:Opt}--\pageref{tab:Reagents:Opt:Ends} приводятся результаты оптимизации молекул-\tqt{реагентов} в приближении RI-MP2~\cite{MP:1934,RI:MP2:1997} с использованием расширенного диффузными функциями трёхэкспоненциального корреляционно-согласованного базисного набора Даннинга (aug-cc-pVTZ).~\cite{Dunning:1989,Peterson:1993,Woon:1993} Если группа симметрии оптимальной структуры не указана явно, она является тривиальной~(\SymGroup{C}{1}). Все полученные в ходе расчёта собственные частоты молекул действительны.

%\caption{\label{tab:Desmotic:Reagents}Результаты расчёта реагентов для реакций разделения связей

\begin{longtabu} to \textwidth {rcl|SS|X[l]}
\multicolumn{6}{p{\textwidth}}{Таблица~\arabic{table}. Строение минимальных по энергии структур молекул-реагентов для реакций разделения связей, RI-MP2/cc-pVTZ} \label{tab:Reagents:Opt} \\
\toprule\multicolumn{3}{c|}{Структура / симметрия} & \(E_{2}\),~\si{\hartree} & \(E_{ZPE}\),~\si{\hartree}&Геометрия\\\midrule\endfirsthead
\multicolumn{6}{p{\textwidth}}{\emph{продолжение таблицы}~\arabic{table}} \\
\toprule\multicolumn{3}{c|}{Структура / симметрия} & \(E_{2}\),~\si{\hartree} & \(E_{ZPE}\),~\si{\hartree}&Геометрия\\\midrule\endhead
\multicolumn{6}{l}{\emph{продолжение на следующей странице}}\endfoot
\bottomrule\endlastfoot
\cmpd{Propane} & \DrawPropane{} & \(\SymGroup{C}{2v}\) & \num{-118.859638} & \num{0.104589} &
\ce{C-C}~\SI{1.523}{\angstrom}; $\angle$\ce{CCC}~\ang{112.3} \\
\midrule\cmpd{Me2O} & \DrawMeOMe{} & \(\SymGroup{C}{2v}\) &  \num{-154.737526} & \num{0.080693} & 
\ce{C-O}~\SI{1.411}{\angstrom};
$\angle$\ce{COC}~\ang{110.5} \\
\midrule\cmpd{Me2S} & \DrawMeSMe{} & \(\SymGroup{C}{2v}\) & \num{-477.351949} & \num{0.076641} & 
\ce{C-S}~\SI{1.804}{\angstrom}; 
$\angle$\ce{CSC}~\ang{97.9} \\
%\midrule\cmpd{Dimethylamine} & \chemfig{H_3C-[:-30]N(-[:-90]H)-[:+30]CH_3} & \(\SymGroup{C}{s}\) & \\
\midrule\cmpd{Isobutane} & \DrawIsoButane{} & \(\SymGroup{C}{3v}\) & \num{-158.087267} & \num{0.132478} & \ce{C-C}~\SI{1.524}{\angstrom}; $\angle$\ce{CCC}~\ang{110.7}\\
\midrule\cmpd{Acetone} & \DrawMeAc{} & \(\SymGroup{C}{2v}\) & \num{-192.789711} & \num{0.084256} & \ce{C=O}~\SI{1.220}{\angstrom}; 
\ce{C-C}~\SI{1.508}{\angstrom};
$\angle$\ce{OCC}~\ang{122.1};
$\angle$\ce{CCC}~\ang{115.8}\\
\midrule \cmpd{Me3N} & \DrawMeNMeMe{} & \(\SymGroup{C}{3v}\) &  \num{-174.100139} & \num{0.121385} & 
\ce{C-N}~\SI{1.451}{\angstrom};
$\angle$\ce{CNC}~\ang{110.2} \\
\midrule\cmpd{Butane} & \DrawNormButane{} & \(\SymGroup{C}{2h}\) &  \num{-158.084303} & \num{0.133309} & 
\ce{H_3C-CH_2}~\SI{1.524}{\angstrom}; 
\ce{H_2C-CH_2}~\SI{1.523}{\angstrom}; 
$\angle$\ce{CCC}~\ang{112.3} \\
\midrule\cmpd{MeOEt} & \DrawMeOEt{} & \(\SymGroup{C}{s}\) & \num{-193.966394} & \num{0.108895} & 
\ce{H3C-O}~\SI{1.412}{\angstrom};
\ce{H_2C-O}~\SI{1.417}{\angstrom};
\ce{H2C-CH_3}~\SI{1.509}{\angstrom};
$\angle$\ce{COC}~\ang{111.2};
$\angle$\ce{OCC}~\ang{108.1} \\
\midrule\cmpd{MeSEt} & \DrawMeSEt{} & \(\SymGroup{C}{s}\) &   \num{-516.576661} & \num{0.105402} &
\ce{H_3C-S}~\SI{1.805}{\angstrom}; 
\ce{H_2C-S}~\SI{1.809}{\angstrom}; 
\ce{H_2C-CH_3}~\SI{1.519}{\angstrom}; 
$\angle$\ce{CSC}~\ang{98.7}; 
$\angle$\ce{SCC}~\ang{109.3} \\
\midrule\cmpd{Butanone} & \DrawEtAc{} & \(\SymGroup{C}{s}\) & \num{-232.014552} & \num{0.113700} & \ce{C=O}~\SI{1.220}{\angstrom};
\ce{H_3C-CO}~\SI{1.509}{\angstrom}; 
\ce{H_2C-CO}~\SI{1.513}{\angstrom}; 
\ce{H_3C-CH_2}~\SI{1.517}{\angstrom};
$\angle\ce{OCC(H_3)}~\ang{121.9}$;
$\angle\ce{OCC(H_2)}~\ang{122.1}$; 
$\angle\ce{CC(O)C}~\ang{116.0}$; 
$\angle\ce{C(O)CC}~\ang{113.4}$;
%$\tau_{\ce{OCCC}}\approx\ang{0}$ 
\\
\midrule\cmpd{Isopentane} & \DrawIsoPentane{} & \(\SymGroup{C}{1}\) & \num{-197.311142} & \num{0.161457} & \\
\midrule\cmpd{Me2CHOMe} & \DrawMeCHMeOMe{} & \(\SymGroup{C}{1}\) & \num{-233.195233} & \num{0.137450} &  \\
\midrule\cmpd{Me2CHSMe} & \DrawMeCHMeSMe{} & & & \\
\midrule\cmpd{Me2NEt} & \DrawMeNMeEt{} & \(\SymGroup{C}{1}\) & & & \\
%\midrule\cmpd{Isopentanone} &\chemfig{H_3C-[:+30](-[:+90]CH_3)-[:-30]C(=[:-90]O)-[:+30]CH_3} & & & \\
\end{longtabu}\label{tab:Reagents:Opt:Ends}

За исключением простейших структур \cmpd{Propane}--\cmpd{Acetone}, молекулы реагентов сами обладают конформационной подвижностью (хотя карбонильная группа вообще считается жёстко планарной, некоторые из существующих исключений будут рассмотрены ниже). Инверсия атома азота триметиламина~\cmpd{Trimethylamine} слабо взаимодействует со внутренним вращением метильных групп, а (N,N-диметил)этиламина~\cmpd{Me2NEt} "--- преимущественно с внутренним вращением, параметризуемым подструктрурой \ce{H3C-N-CH2-CH3}...
%Внутреннее вращение \emph{н}-бутана~\cmpd{Butane} и его аналогов\dots


\pgfplotstablecreatecol[create col/assign/.code={%
  \getthisrow{Etot}\entryval%
  %\ifnum\pgfplotstablerow>0
  \pgfmathsetmacro{\entryval}{min(\entryval,\pgfmathaccuma)}%
  %\fi
  \edef\pgfmathaccuma{\entryval}
  \pgfkeyslet{/pgfplots/table/create col/next content}\entryval
}
]{Emin}\ScanButane %
\pgfplotstablecreatecol[create col/assign/.code={%
  \getthisrow{Etot}\entryval%
  \getthisrow{Emin}\refval%
  %\ifnum\pgfplotstablerow>0
  \pgfmathsetmacro{\entryval}{\entryval - \refval}%
  \pgfmathsetmacro{\entryval}{\AUtoKJM * \entryval}%
  %\fi
  \edef\pgfmathaccuma{\entryval}
  \pgfkeyslet{/pgfplots/table/create col/next content}\entryval
}
]{Erel}\ScanButane %
\pgfplotstablesave[columns={Tau,Etot,Erel}]{\ScanButane}{Results/ScanButane.save.dat}

\pgfplotstablecreatecol[%
create col/assign/.code={%
  \getthisrow{Etot}\entry%
  %\ifnum\pgfplotstablerow>0
  \pgfmathsetmacro{\entry}{min(\entry,\pgfmathaccuma)}%
  %\fi
  \edef\pgfmathaccuma{\entry}
  \pgfkeyslet{/pgfplots/table/create col/next content}\entry
}]{Emin}\ScanMeOEt %
\pgfplotstablecreatecol[%
create col/assign/.code={%
  \getthisrow{Etot}\entryval%
  \getthisrow{Emin}\refval%
  %\ifnum\pgfplotstablerow>0
  \pgfmathsetmacro{\entryval}{\entryval - \refval}%
  \pgfmathsetmacro{\entryval}{\AUtoKJM * \entryval}%
  %\fi
  \edef\pgfmathaccuma{\entryval}
  \pgfkeyslet{/pgfplots/table/create col/next content}\entryval
}
]{Erel}\ScanMeOEt %
\pgfplotstablesave[columns={Tau,Etot,Erel}]{\ScanMeOEt}{Results/ScanMeOEt.save.dat}

\pgfplotstablecreatecol[create col/assign/.code={%
  \getthisrow{Etot}\entry %
  %\ifnum\pgfplotstablerow>0
  \pgfmathsetmacro{\entry}{min(\entry,\pgfmathaccuma)}%
  %\fi
  \edef\pgfmathaccuma{\entry}
  \pgfkeyslet{/pgfplots/table/create col/next content}\entry
}]{Emin}\ScanMeSEt %
\pgfplotstablecreatecol[create col/assign/.code={ %
  \getthisrow{Etot}\entryval %
  \getthisrow{Emin}\refval %
  %\ifnum\pgfplotstablerow>0
  \pgfmathsetmacro{\entryval}{\entryval - \refval}%
  \pgfmathsetmacro{\entryval}{\AUtoKJM * \entryval}%
  %\fi
  \edef\pgfmathaccuma{\entryval}
  \pgfkeyslet{/pgfplots/table/create col/next content}\entryval
} %
]{Erel}\ScanMeSEt %
\pgfplotstablesave[columns={Tau,Etot,Erel}]{\ScanMeSEt}{Results/ScanMeSEt.save.dat}

\pgfplotstablecreatecol[create col/assign/.code={%
  \getthisrow{Etot}\entry%
  %\ifnum\pgfplotstablerow>0
  \pgfmathsetmacro{\entry}{min(\entry,\pgfmathaccuma)}%
  %\fi
  \edef\pgfmathaccuma{\entry}
  \pgfkeyslet{/pgfplots/table/create col/next content}\entry
} %
]{Emin}\ScanMeCOEt %
\pgfplotstablecreatecol[create col/assign/.code={%
  \getthisrow{Etot}\entryval%
  \getthisrow{Emin}\refval%
  %\ifnum\pgfplotstablerow>0
  \pgfmathsetmacro{\entryval}{\entryval - \refval}%
  \pgfmathsetmacro{\entryval}{\AUtoKJM * \entryval}%
  %\fi
  \edef\pgfmathaccuma{\entryval}
  \pgfkeyslet{/pgfplots/table/create col/next content}\entryval
} %
]{Erel}\ScanMeCOEt %
\pgfplotstablesave[columns={Tau,Etot,Erel}]{\ScanMeCOEt}{Results/ScanMeCOEt.save.dat}

\begin{figure}
  \caption{Потенциалы внутреннего вращения молекул \cmpd{Butane}--\cmpd{Butanone}\label{fig:Internal:Rotation:Butanes}}
  \centerfloat{
    \begin{tikzpicture}
      \begin{axis} [%
        %title={\chemfig{H_3C-[:+30]\circ-[:-30]\circ-[:+30]CH_3}},
        %legend pos=outer north east,
        legend style={draw=none},
        xlabel={\DrawHeteroButane{}{\circ}{\circ} / $\tau$,~\textdegree},
        xmin=-180, xmax=180,
        xtick={-180,-120,...,180},
        ylabel={$\Delta\Func{E}$,~\si{\kilo\joule\per\mole}},
        ymin=-1, ymax=32]
        \addplot [black,thick,smooth,domain=-180:180] table [x=Tau,y=Erel] 
        {Results/ScanButane.save.dat};\addlegendentry{\chemfig{[,0.75]-[:+30]-[:-30]-[:+30]}~\cmpd{Butane}}
        \addplot [red,thick,smooth,domain=-180:180] table [x=Tau,y=Erel]
        {Results/ScanMeOEt.save.dat};\addlegendentry{\chemfig{[,0.75]-[:+30]-[:-30]O-[:+30]}~\cmpd{MeOEt}}
        \addplot [brown,thick,smooth,domain=-180:180] table [x=Tau,y=Erel] 
        {Results/ScanMeSEt.save.dat};\addlegendentry{\chemfig{[,0.75]-[:+30]-[:-30]S-[:+30]}~\cmpd{MeSEt}}
        \addplot [green,thick,smooth,domain=-180:180] table [x=Tau,y=Erel] 
        {Results/ScanMeCOEt.save.dat}; \addlegendentry{\chemfig{[,0.75]-[:+30]-[:-30]C(=[:-90]O)-[:+30]}~\cmpd{Butanone}}
      \end{axis}
    \end{tikzpicture}
  }
\end{figure}


\pgfplotstablecreatecol[create col/assign/.code={%
  \getthisrow{Etot}\entry%
  %\ifnum\pgfplotstablerow>0
  \pgfmathsetmacro{\entry}{min(\entry,\pgfmathaccuma)}%
  %\fi
  \edef\pgfmathaccuma{\entry}
  \pgfkeyslet{/pgfplots/table/create col/next content}\entry
} %
]{Emin}\ScanIsoPentane %
\pgfplotstablecreatecol[create col/assign/.code={%
  \getthisrow{Etot}\entryval%
  \getthisrow{Emin}\refval%
  %\ifnum\pgfplotstablerow>0
  \pgfmathsetmacro{\entryval}{\entryval - \refval}%
  \pgfmathsetmacro{\entryval}{\AUtoKJM * \entryval}%
  %\fi
  \edef\pgfmathaccuma{\entryval}
  \pgfkeyslet{/pgfplots/table/create col/next content}\entryval
} %
]{Erel}\ScanIsoPentane %
\pgfplotstablesave[columns={Tau,Etot,Erel}]{\ScanIsoPentane}{Results/ScanIsoPentane.save.dat}

\pgfplotstablecreatecol[create col/assign/.code={%
  \getthisrow{Etot}\entry%
  %\ifnum\pgfplotstablerow>0
  \pgfmathsetmacro{\entry}{min(\entry,\pgfmathaccuma)}%
  %\fi
  \edef\pgfmathaccuma{\entry}
  \pgfkeyslet{/pgfplots/table/create col/next content}\entry
}]{Emin}\ScanIPrOMe %
\pgfplotstablecreatecol[create col/assign/.code={%
  \getthisrow{Etot}\entryval%
  \getthisrow{Emin}\refval%
  %\ifnum\pgfplotstablerow>0
  \pgfmathsetmacro{\entryval}{\entryval - \refval}%
  \pgfmathsetmacro{\entryval}{\AUtoKJM * \entryval}%
  %\fi
  \edef\pgfmathaccuma{\entryval}
  \pgfkeyslet{/pgfplots/table/create col/next content}\entryval
} %
]{Erel}\ScanIPrOMe %
\pgfplotstablesave[columns={Tau,Etot,Erel}]
{\ScanIPrOMe}{Results/ScanIPrOMe.save.dat}
%
\pgfplotstablecreatecol[%
create col/assign/.code={%
  \getthisrow{Etot}\entry%
  %\ifnum\pgfplotstablerow>0
  \pgfmathsetmacro{\entry}{min(\entry,\pgfmathaccuma)}%
  %\fi
  \edef\pgfmathaccuma{\entry}
  \pgfkeyslet{/pgfplots/table/create col/next content}\entry
} %
]{Emin}\ScanIPrSMe %
\pgfplotstablecreatecol[create col/assign/.code={%
  \getthisrow{Etot}\entryval%
  \getthisrow{Emin}\refval%
  %\ifnum\pgfplotstablerow>0
  \pgfmathsetmacro{\entryval}{\entryval - \refval}%
  \pgfmathsetmacro{\entryval}{\AUtoKJM * \entryval}%
  %\fi
  \edef\pgfmathaccuma{\entryval}
  \pgfkeyslet{/pgfplots/table/create col/next content}\entryval
} %
]{Erel}\ScanIPrSMe %
\pgfplotstablesave[columns={Tau,Etot,Erel}]{\ScanIPrSMe}{Results/ScanIPrSMe.save.dat}
%
\pgfplotstablecreatecol[create col/assign/.code={
  \getthisrow{Etot}\entry %
  %\ifnum\pgfplotstablerow>0
  \pgfmathsetmacro{\entry}{min(\entry,\pgfmathaccuma)}%
  %\fi
  \edef\pgfmathaccuma{\entry}
  \pgfkeyslet{/pgfplots/table/create col/next content}\entry
} %
]{Emin}\ScanMeMeNEt %
\pgfplotstablecreatecol[%
create col/assign/.code={%
  \getthisrow{Etot}\entryval%
  \getthisrow{Emin}\refval%
  %\ifnum\pgfplotstablerow>0
  \pgfmathsetmacro{\entryval}{\entryval - \refval}%
  \pgfmathsetmacro{\entryval}{\AUtoKJM * \entryval}%
  %\fi
  \edef\pgfmathaccuma{\entryval}
  \pgfkeyslet{/pgfplots/table/create col/next content}\entryval
} %
]{Erel}\ScanMeMeNEt %
\pgfplotstablesave[columns={Tau,Etot,Erel}]{\ScanMeMeNEt}{Results/ScanMeMeNEt.save.dat}


\begin{figure}
\caption{Потенциалы внутреннего вращения молекул \cmpd{Isopentane}--\cmpd{Me2NEt}\label{fig:Internal:Rotation:Isopentanes}}
  \centerfloat{
\begin{tikzpicture}
  \begin{axis} [%
    legend style={draw=none},
    xlabel={\DrawHeteroIsoPentane{}{\bullet}{\circ}/$\tau$,~\textdegree},
    xmin=-180, xmax=180,
    xtick={-180,-120,...,180},
    ylabel={$\Delta\Func{E}$,~\si{\kilo\joule\per\mole}},
    ymin=-1, ymax=32]
    \addplot [black,thick,smooth,domain=-180:180] table [x=Tau,y=Erel] 
    {Results/ScanIsopentane.save.dat};\addlegendentry{\chemfig{[,0.675]-[:+30](-[:+90])-[:-30]-[:+30]}~\cmpd{Isopentane}};
    \addplot [red,thick,smooth,domain=-180:180] table [x=Tau,y=Erel] 
    {Results/ScanIPrOMe.save.dat};\addlegendentry{\chemfig{[,0.675]-[:+30](-[:+90])-[:-30]O-[:+30]}~\cmpd{Me2CHOMe}};
    \addplot [brown,thick,smooth,domain=-180:180] table [x=Tau,y=Erel] 
    {Results/ScanIPrSMe.save.dat};\addlegendentry{\chemfig{[,0.675]-[:+30](-[:+90])-[:-30]S-[:+30]}~\cmpd{Me2CHSMe}};
    \addplot [blue,thick,smooth,domain=-180:180] table [x=Tau,y=Erel] 
    {Results/ScanMeMeNEt.save.dat};\addlegendentry{\chemfig{[,0.75]-[:+30]N(-[:+90])-[:-30]-[:+30]}~\cmpd{Me2NEt}};
\end{axis}
\end{tikzpicture}%
}
\end{figure}


%
% Methylacetate
%
\pgfplotstablecreatecol[create col/assign/.code={%
  \getthisrow{Etot}\entryval %
%  \ifnum\pgfplotstablerow>0
  \pgfmathsetmacro{\entryval}{min(\entryval,\pgfmathaccuma)}%
%  \fi
  \edef\pgfmathaccuma{\entryval} %
  \pgfkeyslet{/pgfplots/table/create col/next content}\entryval %
} %
]{Emin}\ScanMeOAc
\pgfplotstablecreatecol[create col/assign/.code={%
  \getthisrow{Etot}\entryval%
  \getthisrow{Emin}\refval%
%  %\ifnum\pgfplotstablerow>0
  \pgfmathsetmacro{\entryval}{\entryval - \refval}%
  \pgfmathsetmacro{\entryval}{\AUtoKJM * \entryval}%
%\fi
  \edef\pgfmathaccuma{\entryval} %
  \pgfkeyslet{/pgfplots/table/create col/next content}\entryval %
} %
]{Erel}\ScanMeOAc
\pgfplotstablesave[columns={Tau,Etot,Erel}]{\ScanMeOAc}{Results/ScanMeOAc.save.dat} 
%
% Methyl thioacetate
%
\pgfplotstablecreatecol[create col/assign/.code={%
  \getthisrow{Etot}\entryval%
  %  \ifnum\pgfplotstablerow>0
  \pgfmathsetmacro{\entryval}{min(\entryval,\pgfmathaccuma)}%
  %  \fi
  \edef\pgfmathaccuma{\entryval} %
  \pgfkeyslet{/pgfplots/table/create col/next content}\entryval %
}]{Emin}\ScanMeSAc %
\pgfplotstablecreatecol[%
create col/assign/.code={%
  \getthisrow{Etot}\entryval%
  \getthisrow{Emin}\refval%
  %  %\ifnum\pgfplotstablerow>0
  \pgfmathsetmacro{\entryval}{\entryval - \refval}%
  \pgfmathsetmacro{\entryval}{\AUtoKJM * \entryval}%
  %\fi
  \edef\pgfmathaccuma{\entryval} %
  \pgfkeyslet{/pgfplots/table/create col/next content}\entryval %
}]{Erel}\ScanMeSAc %
\pgfplotstablesave[columns={Tau,Etot,Erel}]{\ScanMeSAc}{Results/ScanMeSAc.save.dat} 
%
% Methylthionacetate
%
\pgfplotstablecreatecol[create col/assign/.code={%
  \getthisrow{Etot}\entryval %
  %  \ifnum\pgfplotstablerow>0
  \pgfmathsetmacro{\entryval}{min(\entryval,\pgfmathaccuma)}%
  %  \fi
  \edef\pgfmathaccuma{\entryval} %
  \pgfkeyslet{/pgfplots/table/create col/next content}\entryval %
} %
]{Emin}\ScanMeOCSMe
\pgfplotstablecreatecol[create col/assign/.code={%
  \getthisrow{Etot}\entryval%
  \getthisrow{Emin}\refval%
  %  %\ifnum\pgfplotstablerow>0
  \pgfmathsetmacro{\entryval}{\entryval - \refval}%
  \pgfmathsetmacro{\entryval}{\AUtoKJM * \entryval}%
  %\fi
  \edef\pgfmathaccuma{\entryval} %
  \pgfkeyslet{/pgfplots/table/create col/next content}\entryval %
} %
]{Erel}\ScanMeOCSMe
\pgfplotstablesave[columns={Tau,Etot,Erel}]{\ScanMeOCSMe}{Results/ScanMeOCSMe.save.dat} 
%
% Methyldithioacetate
%
\pgfplotstablecreatecol[create col/assign/.code={%
  \getthisrow{Etot}\entryval %
  %  \ifnum\pgfplotstablerow>0
  \pgfmathsetmacro{\entryval}{min(\entryval,\pgfmathaccuma)}%
  %  \fi
  \edef\pgfmathaccuma{\entryval} %
  \pgfkeyslet{/pgfplots/table/create col/next content}\entryval %
} %
]{Emin}\ScanMeSCSMe
\pgfplotstablecreatecol[create col/assign/.code={%
  \getthisrow{Etot}\entryval%
  \getthisrow{Emin}\refval%
  %  %\ifnum\pgfplotstablerow>0
  \pgfmathsetmacro{\entryval}{\entryval - \refval}%
  \pgfmathsetmacro{\entryval}{\AUtoKJM * \entryval}%
  %\fi
  \edef\pgfmathaccuma{\entryval} %
  \pgfkeyslet{/pgfplots/table/create col/next content}\entryval %
} %
]{Erel}\ScanMeSCSMe
\pgfplotstablesave[columns={Tau,Etot,Erel}]{\ScanMeSCSMe}{Results/ScanMeSCSMe.save.dat} 

\begin{figure}
  \caption{Потенциалы внутреннего вращения аналогов сложных эфиров\label{fig:Internal:Rotation:Esters} [переделать координаты, чтобы угол был \ce{H3C-C-Y-CH3}, иначе тут всё слева неправильно]}
  \centerfloat{
    \begin{tikzpicture}
      \begin{axis}[legend style={draw=none},
        xlabel={\chemfig{[,0.875]H_3C-[:+30]C(=[:+90]X)-[:-30]Y-[:+30]CH_3}/$\tau$,~\textdegree},
        xmin=-180, xmax=180,
        xtick={-180,-120,...,180},
        ylabel={$\Delta\Func{E}$,~\si{\kilo\joule\per\mole}},
        ymin=-1, ymax=64]
        \addplot [black,thick,smooth,domain=-180:180] table [x=Tau,y=Erel] 
        {Results/ScanMeOAc.save.dat};\addlegendentry{\chemfig{[,0.675]-[:+30](=[:+90]O)-[:-30]O-[:+30]}~\cmpd{MeOAc}};
        \addplot [blue,thick,smooth,domain=-180:180] table [x=Tau,y=Erel] 
        {Results/ScanMeSAc.save.dat};\addlegendentry{\chemfig{[,0.675]-[:+30](=[:+90]O)-[:-30]S-[:+30]}~\cmpd{MeSAc}};
        \addplot [red,thick,smooth,domain=-180:180] table [x=Tau,y=Erel] 
        {Results/ScanMeOCSMe.save.dat};\addlegendentry{\chemfig{[,0.675]-[:+30](=[:+90]S)-[:-30]O-[:+30]}~\cmpd{MeOCSMe}};
        \addplot [green,thick,smooth,domain=-180:180] table [x=Tau,y=Erel] 
        {Results/ScanMeSCSMe.save.dat};\addlegendentry{\chemfig{[,0.675]-[:+30](=[:+90]S)-[:-30]S-[:+30]}~\cmpd{MeSCSMe}};
      \end{axis}
    \end{tikzpicture}
  }
\end{figure}



Глобально оптимальными, судя по~рис.~\ref{fig:Internal:Rotation:Butanes}--\ref{fig:Internal:Rotation:Isopentanes}, являются \emph{транс}- или \emph{транс,~гош}-конформеры с $\tau\approx+\ang{60}$ и $\tau\approx\ang{180}$. В дальнейшем во всех случаях для расчётов энергетических эффектов реакций разделения связей были использованы оптимизированные значения энергии именно для таких структур.

Введение в рассмотрение сероорганических веществ резко увеличивает число типов связей и, соответственно, количество структур, необходимых для рассмотрения реакции. Это напрямую связано со значительным количеством окислительных состояний, которое может принимать атом серы в органических соединениях: не только сульфиды~\ce{R2S}, но и дисульфиды~\ce{R2S2}, а также сульфоксиды~\ce{R2SO}, сульфоны~\ce{R2SO2}, а также иные серосодержащие функциональные группы\dots

\section{Циклическое и аксиальное напряжение в свободных шестичленных циклах}

Теория молекулярных графов...

Бициклические системы аналогов бицикло[3.3.1]нонана рассматриваются в рамках теории строения молекул как два шестичленных цикла с общей цепью. Логичным выглядит рассмотрение конформационного поведения системы на основе возмущения независимости данных циклов, имея в виду то, что поведение изолированных циклов считается более простым...

Реакция разделения связей для 1,3- и 1,4-дизамещённых аналогов циклогексана формально одинаковы:

\begin{center}
  \begin{equation*} \underbrace{\text{\chemfig{[:-30]X*6(---Y---)}} \qquad \text{\chemfig{[:-30]*6(-X----Y-)}}}_{\text{как бы это всё поизяЧнее сделать?}} \end{equation*}
+ \chemfig{H_3C-[:-30]X-[:+30]CH_3} + \chemfig{H_3C-[:-30]Y-[:+30]CH_3} + 4\(\cdot\) \chemfig{H_3C-[:-30]-[:+30]CH_3} \(\longrightarrow\)
  
  \(\longrightarrow\) 
    2\(\cdot\)\chemfig{H_3C-[:+30]X-[:-30]-[:+30]CH_3} +     2\(\cdot\)\chemfig{H_3C-[:+30]-[:-30]-[:+30]CH_3} + 2\(\cdot\)\chemfig{H_3C-[:+30]Y-[:-30]-[:+30]CH_3}
\end{center}

Для оценки взаимного влияния различных типов замещения в насыщенных шестичленных циклах используются энергетические эффекты реакций \emph{позиционного диспропорционирования} дизамещённых аналогов циклогексана
\begin{center}
  \chemfig{*6(-X---Y--)} + \chemfig{*6(------)} \(\longrightarrow\) \chemfig{*6(-X-----)} + \chemfig{*6(----Y--)}
  
  \chemfig{X*6(--Y----)} + \chemfig{*6(------)} \(\longrightarrow\) \chemfig{X*6(------)} + \chemfig{Y*6(------)}
\end{center}

B таблице~\ref{tab:Cycle:Six:Opt} на с.~\pageref{tab:Cycle:Six:Opt}--\pageref{tab:Cycle:Six:Opt:Ends} приводятся результаты оптимизации молекул производных циклогексана, включая эндоциклические торсионные углы $\tau$.

\begin{center}
  \chemfig{*6(------)} \quad\chemfig{*6(-X-----)} \quad\chemfig{*6(---Y--X-)} \quad\chemfig{*6(-X---Y--)} \quad\chemfig{*6(-Z--Y--X-)}
\end{center}

\begin{longtabu} to \textwidth {rcl|SS|X[l]}
\multicolumn{6}{p{\textwidth}}{Таблица\label{tab:Cycle:Six:Opt}~\arabic{table}. Строение молекул циклогексана и его аналогов по данным неэмпирических расчётов} \\ 
\toprule \multicolumn{3}{c|}{Структура / конформация}  & \(E_{2}\),~\si{\hartree} & \(E_{ZPE}\),~\si{\hartree} & Геометрия \\ \midrule \endfirsthead
\multicolumn{6}{l}{\emph{продолжение таблицы}~\thetable:} \\
\midrule\multicolumn{3}{c|}{Структура / конформация} & \(E_{2}\),~\si{\hartree} & \(E_{ZPE}\),~\si{\hartree} & Геометрия \\ \midrule \endhead
\multicolumn{6}{l}{\emph{продолжение на следующей странице}}\endfoot
\bottomrule\endlastfoot
\cmpd{Cyclohexane} & \chemfig{?-[:+160]-[:-120]<[:+20]-[:-20,,,,line width=\boldbondwidth]>[:+60]?} & \SymGroup{D}{3d}~(\ConfName{C}) &  \num{-235.344605} & \num{0.171358} & \ce{C-C}~\SI{1.526}{\angstrom}; $\angle\ce{CCC}~\ang{110.7}$; $\tau=\pm\ang{56.9}$ \\ 
\bottomrule\multicolumn{6}{l}{Монозамещённые (\emph{пентаметиленовые}) производные \ce{(CH2)5X} }\\\toprule
\cmpd{Pyran} & \chemfig{?-[:+160]-[:-120]O<[:+20]-[:-20,,,,line width=\boldbondwidth]>[:+60]?} & \SymGroup{C}{s}~(\ConfName{C}) &  \num{-271.229641} & \num{0.147833} & $\angle\ce{COC}~\ang{110.9}$; $\angle\ce{OCC}~\ang{111.9}$; $\angle\ce{CC_3C}~\ang{109.9}$; $\angle\ce{CC_4C}~\ang{109.1}$; \\
\midrule\cmpd{Thiopyran} & \chemfig{?-[:+160]-[:-120]S<[:+20]-[:-20,,,,line width=\boldbondwidth]>[:+60]?} & & \num{-593.841537}  & \num{0.143971} & \chemfig{[-30,1.5]*6(@{6}-@{1}S-@{2}-@{3}-@{4}-@{5}-)} \\
\midrule\cmpd{Cyclohexanone} & \chemfig{?-[:+160]-[:-120]C(=[:-135,0.875]O)<[:+20]-[:-20,,,,line width=\boldbondwidth]>[:+60]?} & & & & \ce{C-C} \\
\midrule\cmpd{MeCyclohexaneEq} & \chemfig{?-[:+160]-[:-120](-[:+155]H_3C)(-[:-90,0.75]H)<[:+20]-[:-20,,,,line width=\boldbondwidth]>[:+60]?} & & & & \ce{C-C} \\
\midrule\cmpd{NMePiperidineEq} & \chemfig{?-[:+160]-[:-120]N(-[:+155]H_3C)<[:+20]-[:-20,,,,line width=\boldbondwidth]>[:+60]?} & & & & \ce{C-C} \\ 
\bottomrule\multicolumn{6}{l}{Симметричные 1,3-дизамещённые производные~(\ce{X}=\ce{Y}) }\\\toprule
\cmpd{Dioxane13} & \chemfig{?-[:+160]-[:-120]O<[:+20]-[:-20,,,,line width=\boldbondwidth]O>[:+60]?} & \SymGroup{C}{s}~(\ConfName{C}) & & & \ce{C-C} \\ 
\midrule\cmpd{Dithiane13} & \chemfig{?-[:+160]-[:-120]S<[:+20]-[:-20,,,,line width=\boldbondwidth]S>[:+60]?} & & & & \ce{C-C} \\
\midrule\cmpd{N2Me2Diazine13} & \chemfig{?-[:+160]-[:-120]N(-[:+160]H_3C)<[:+20]-[:-20,,,,line width=\boldbondwidth]N(-[:+20]CH_3)>[:+60]?} & & & & \ce{C-C} \\
\bottomrule\multicolumn{6}{l}{Симметричные 1,4-дизамещённые производные~(\ce{X}=\ce{Y}) } \\\toprule
\cmpd{Dioxane14} & \chemfig{?-[:+160]-[:-120]O<[:+20]-[:-20,,,,line width=\boldbondwidth]>[:+60]O?} & \SymGroup{C}{2h}~(\ConfName{C}) & & & \ce{C-C} \\
\midrule\cmpd{Dithiane14} & \chemfig{?-[:+160]-[:-120]S<[:+20]-[:-20,,,,line width=\boldbondwidth]>[:+60]S?} & & & & \ce{C-C} \\
\midrule\cmpd{Cyclohexandione14} & \chemfig{?-[:-120]C(=[:-135,0.875]O)<[:+20]-[:-20,,,,line width=\boldbondwidth]>[:+60]C(=[:+45,0.875]O)-[:-160]?} & & & & \ce{C-C} \\
\midrule\cmpd{Me2CyclohexaneEE} & \chemfig{?-[:-120](-[:+155]H_3C)(-[:-90,0.75]H)<[:+20]-[:-20,,,,line width=\boldbondwidth]>[:+60](-[:+90,0.75]H)(-[:-25]CH_3)-[:-160]?} & & & & \ce{C-C} \\
\midrule\cmpd{N2Me2PiperazineEE} & \chemfig{?-[:-120]N(-[:+155]H_3C)<[:+20]-[:-20,,,,line width=\boldbondwidth]>[:+60]N(-[:-25]CH_3)-[:-160]?} & & & & \ce{C-C} \\
\bottomrule\multicolumn{6}{l}{Несимметричные 1,4-дизамещённые производные~(\ce{X}$\ne$\ce{Y}) } \\
\toprule\cmpd{Oxathiane4} & \chemfig{?-[:+160]-[:-120]S<[:+20]-[:-20,,,,line width=\boldbondwidth]>[:+60]O?} & \SymGroup{C}{s}~(\ConfName{C}) & & & \ce{C-C} \\
\midrule \cmpd{Pyranone4} & \chemfig{?-[:+160]-[:-120]C(=[:-135,0.875]O)<[:+20]-[:-20,,,,line width=\boldbondwidth]>[:+60]O?} & & & & \ce{C-C} \\
\midrule\cmpd{Me4PyraneEq} & \chemfig{?-[:+160]-[:-120](-[:+155]H_3C)(-[:-90,0.75]H)<[:+20]-[:-20,,,,line width=\boldbondwidth]>[:+60]O?} & & & & \ce{C-C} \\
\midrule\cmpd{NMeMorpholineEq} & \chemfig{?-[:+160]-[:-120]N(-[:+155]H_3C)<[:+20]-[:-20,,,,line width=\boldbondwidth]>[:+60]O?} & & & & \ce{C-C} \\
\midrule \cmpd{Thiopyranone4} & \chemfig{?-[:+160]-[:-120]C(=[:-135,0.875]O)<[:+20]-[:-20,,,,line width=\boldbondwidth]>[:+60]S?} & & & & \ce{C-C} \\
\midrule\cmpd{Me4ThiopyraneEq} & \chemfig{?-[:+160]-[:-120](-[:+155]H_3C)(-[:-90,0.75]H)<[:+20]-[:-20,,,,line width=\boldbondwidth]>[:+60]S?} & & & & \ce{C-C} \\
\midrule\cmpd{NMeThiomorpholineEq} & \chemfig{?-[:+160]-[:-120]N(-[:+155]H_3C)<[:+20]-[:-20,,,,line width=\boldbondwidth]>[:+60]S?} & & & & \ce{C-C} \\
\midrule\cmpd{Me4CyclohexanoneEq} & \chemfig{?-[:-120]C(=[:-135,0.875]O)<[:+20]-[:-20,,,,line width=\boldbondwidth]>[:+60](-[:+90,0.75]H)(-[:-25]CH_3)-[:-160]?} & & & & \ce{C-C} \\
\midrule\cmpd{NMePiperidone4Eq} & \chemfig{?-[:-120]C(=[:-135,0.875]O)<[:+20]-[:-20,,,,line width=\boldbondwidth]>[:+60]N(-[:-25]CH_3)-[:-160]?} & & & & \ce{C-C} \\
\midrule\cmpd{4MePiperidineEE} & \chemfig{?-[:-120]N(-[:+155]H_3C)<[:+20]-[:-20,,,,line width=\boldbondwidth]>[:+60](-[:+90,0.75]H)(-[:-25]CH_3)-[:-160]?} & & & & \ce{C-C} \\
\bottomrule\multicolumn{6}{l}{Симметричные 1,3,5-тризамещённые производные~(\ce{X}=\ce{Y}=\ce{Z}) } \\
\toprule\cmpd{STrioxane} & \chemfig{O?-[:+160]-[:-120]O<[:+20]-[:-20,,,,line width=\boldbondwidth]O>[:+60]?} & \SymGroup{C}{3v}~(\ConfName{C}) & \num{} & & \ce{C-O} \\
\midrule\cmpd{STrithiane} & \chemfig{S?-[:+160]-[:-120]S<[:+20]-[:-20,,,,line width=\boldbondwidth]S>[:+60]?} & & & & \ce{C-S} \\
\bottomrule\multicolumn{6}{l}{Несимметричные 1,3,5-тризамещённые производные~(\ce{X}=\ce{Y}) } \\
\toprule\cmpd{Thiodioxane35} & \chemfig{S?-[:+160]-[:-120]O<[:+20]-[:-20,,,,line width=\boldbondwidth]O>[:+60]?} & \SymGroup{C}{s}~(\ConfName{C}) & & & \ce{C-C} \\ 
\midrule\cmpd{Oxa5dithiane13} & \chemfig{O?-[:+160]-[:-120]S<[:+20]-[:-20,,,,line width=\boldbondwidth]S>[:+60]?} & & & & \ce{C-C} \\
\midrule\cmpd{N2Me2Oxa5diazine13} & \chemfig{O?-[:+160]-[:-120]N(-[:+160]H_3C)<[:+20]-[:-20,,,,line width=\boldbondwidth]N(-[:+20]CH_3)>[:+60]?} & & & & \ce{C-C} \\
\midrule\cmpd{N2Me2Thia5diazine13} & \chemfig{S?-[:+160]-[:-120]N(-[:+160]H_3C)<[:+20]-[:-20,,,,line width=\boldbondwidth]N(-[:+20]CH_3)>[:+60]?} & & & & \ce{C-C} \\
\end{longtabu}\label{tab:Cycle:Six:Opt:Ends}

Длины связей~(\AA) и эндоциклические двугранные углы

\vspace{\medskipamount}

\chemfig{[,1.75]*6(@{6}-@{1}-@{2}-@{3}-@{4}-@{5}-)} %
\namebond{3}{4}{\small\ang{56.9}}  %
\namebond{1}{2}{\small\num{1.526}} %
\quad
\chemfig{[,1.75]*6(@{6}-@{1}O-@{2}-@{3}-@{4}-@{5}-)} %
\namebond{1}{2}{\small\ang{60.5}}  %
\namebond{2}{3}{\small\ang{-57.4}} %
\namebond{3}{4}{\small\ang{53.4}}  %
\namebond{5}{4}{\small\num{1.526}} % 
\namebond{6}{5}{\small\num{1.520}} % 
\namebond{6}{1}{\small\num{1.422}} %
\quad
\chemfig{[,1.75]*6(@{6}-@{1}S-@{2}-@{3}-@{4}-@{5}-)}
\namebond{1}{2}{\small\ang{56.3}} 
\namebond{2}{3}{\small\ang{-62.1}}
\namebond{3}{4}{\small\ang{59.6}}
\namebond{5}{4}{\small\num{1.525}}
\namebond{6}{5}{\small\num{1.521}}
\namebond{6}{1}{\small\num{1.814}}
\quad

%\begin{longtabu} to \textwidth {X} \end{longtabu}

\vspace{\medskipamount}
Энергии напряжения... $\EStrain$

