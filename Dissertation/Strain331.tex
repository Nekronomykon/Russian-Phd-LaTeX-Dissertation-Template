% !TeX spellcheck = ru_RU
% !TeX encoding = UTF-8
\chapter{Исследование напряжений в производных бицикло[3.3.1]нонана}

Реакции разделения связей. Наименьшую погрешность обеспечивают \emph{гипергомодесмотические} реакции, сохраняющие не только состав, как гомодесмические... но и число связей различных типов. В частности, для 3,7,9-замещённых аналогов бицикло[3.3.1]нонана выводится уравнение
\begin{center}
  \chemfig{[:-30]Z*6(-(-[:180]-[:120]X-[:+60]-[:0]?)--Y---)}
  + \chemfig{H_3C-[:-30]X-[:+30]CH_3} + \chemfig{H_3C-[:-30]Y-[:+30]CH_3} + \chemfig{H_3C-[:-30]Z-[:+30]CH_3} + 4\(\cdot\) \chemfig{H_3C-[:-30]-[:+30]CH_3}
  + 4\(\cdot\)\chemfig{H_3C-[:+30](-[:+90]CH_3)-[:-30]CH_3}
  \(\longrightarrow\)
  
  \(\longrightarrow\) 
  2\(\cdot\)\chemfig{H_3C-[:+30]X-[:-30]-[:+30]CH_3} + 2\(\cdot\)\chemfig{H_3C-[:+30]Y-[:-30]-[:+30]CH_3} + 4\(\cdot\)\chemfig{H_3C-[:+30](-[:+90]CH_3)-[:-30]-[:+30]CH_3} + 
  2\(\cdot\)\chemfig{H_3C-[:+30](-[:+90]CH_3)-[:-30]Z-[:+30]CH_3}
\end{center}

Для 1,3,5,7-аналогов исследуемой бициклической системы
\begin{center}
  \chemfig{[:-30]*6(-A(-[:180]-[:120]X-[:+60]-[:0]\phantom{Z}?)--Y--Z-)}
  + \chemfig{H_3C-[:-30]X-[:+30]CH_3} + \chemfig{H_3C-[:-30]Y-[:+30]CH_3} + 5\(\cdot\) \chemfig{H_3C-[:-30]-[:+30]CH_3}
  + 2\(\cdot\)\chemfig{H_3C-[:+30]A(-[:+90]CH_3)-[:-30]CH_3}
  + 2\(\cdot\)\chemfig{H_3C-[:+30]Z(-[:+90]CH_3)-[:-30]CH_3}
  \(\longrightarrow\)
  
  \(\longrightarrow\) 
  2\(\cdot\)\chemfig{H_3C-[:+30]X-[:-30]-[:+30]CH_3} + 2\(\cdot\)\chemfig{H_3C-[:+30]Y-[:-30]-[:+30]CH_3} + 3\(\cdot\)\chemfig{H_3C-[:+30]A(-[:+90]CH_3)-[:-30]-[:+30]CH_3} + 3\(\cdot\)\chemfig{H_3C-[:+30]Z(-[:+90]CH_3)-[:-30]-[:+30]CH_3}
\end{center}

Энергетический эффект этих реакций, взятый с обратным знаком, оценивает общее напряжение молекулы относительно ациклических форм, считающихся свободными от напряжения. 

Для более подробного рассмотрения отдельных взаимодействий необходимо комбинировать эти уравнения с другими.

Расчёты 

\begin{equation}
\Func{E} = \underbrace{E_\infty  + E_k}_{\sim E_{el}} + E_{ZPE}
\end{equation}

Ниже в таблице~\ref{tab:Reagents:Opt} на с.~\pageref{tab:Reagents:Opt:Begin}--\pageref{tab:Reagents:Opt:End} приводятся результаты оптимизации молекул-\tqt{реагентов} в приближении MP2/aug-cc-pVTZ. Если группа симметрии оптимальной структуры не указана явно, она является тривиальной~(\SymGroup{C}{1}). Все полученные в ходе расчёта собственные частоты молекул действительны.

%\caption{\label{tab:Desmotic:Reagents}Результаты расчёта реагентов для реакций разделения связей}
\label{tab:Reagents:Opt:Begin}
\begin{longtabu} to \textwidth {rr|SS|X[l]}
%\caption{Оптимизированные энергии и структурные параметры молекул}
%\caption{Оптимизированные (MP2 / aug-cc-pVTZ) энергии и геометрические параметры молекул}
%\small
\toprule\multicolumn{2}{c|}{\label{tab:Reagents:Opt} Структура / симметрия}  & \(E_{2}\),~\si{\hartree} & \(E_{ZPE}\),~\si{\hartree} & Геометрия \endfirsthead
\multicolumn{5}{p{\textwidth}}{\emph{продолжение таблицы}~\arabic{table}} \\
\toprule\multicolumn{2}{c|}{Структура / симметрия} & \(E_{2}\),~\si{\hartree} & \(E_{ZPE}\),~\si{\hartree} & Геометрия \endhead
  \bottomrule\multicolumn{5}{l}{\emph{продолжение на следующей странице}} \endfoot
  \bottomrule\multicolumn{5}{l}{} \endlastfoot
\midrule\chemfig{H_3C-[:-30]-[:+30]CH_3} / \(\SymGroup{C}{2v}\) &\cmpd{Propane} & \num{-118.859638} & \num{0.104589} &
\ce{C-C}~\SI{1.523}{\angstrom}; $\angle$\ce{CCC}~\ang{112.3} \\
\midrule \chemfig{H_3C-[:-30]O-[:+30]CH_3} / \(\SymGroup{C}{2v}\) &\cmpd{Dimethylether} & \num{-154.737526} & \num{0.080693} & 
\ce{C-O}~\SI{1.411}{\angstrom};
$\angle$\ce{COC}~\ang{110.5} \\
\midrule \chemfig{H_3C-[:-30]S-[:+30]CH_3} / \(\SymGroup{C}{2v}\) & \cmpd{Dimethylsulfide} & \num{-477.351949} & \num{0.076641} & 
\ce{C-S}~\SI{1.804}{\angstrom}; 
$\angle$\ce{CSC}~\ang{97.9} \\
%\midrule\cmpd{Dimethylamine} & \chemfig{H_3C-[:-30]N(-[:-90]H)-[:+30]CH_3} & \(\SymGroup{C}{s}\) & \\
\midrule \chemfig{H_3C-[:-30](-[:-90]CH_3)-[:+30]CH_3} / \(\SymGroup{C}{3v}\) & \cmpd{Isobutane} & \num{-158.087267} & \num{0.132478} & \ce{C-C}~\SI{1.524}{\angstrom}; $\angle$\ce{CCC}~\ang{110.7}\\
\midrule \chemfig{H_3C-[:-30]N(-[:-90]CH_3)-[:+30]CH_3} / \(\SymGroup{C}{3v}\) & \cmpd{Trimethylamine} & \num{-174.100139} & \num{0.121385} & 
\ce{C-N}~\SI{1.451}{\angstrom};
$\angle$\ce{CNC}~\ang{110.2} \\
\midrule\chemfig{H_3C-[:-30]C(=[:-90]O)-[:+30]CH_3} / \(\SymGroup{C}{2v}\) & \cmpd{Acetone} &  \num{-192.789711} & \num{0.084256} & \ce{C=O}~\SI{1.220}{\angstrom}; 
\ce{C-C}~\SI{1.508}{\angstrom};
$\angle$\ce{OCC}~\ang{122.1};
$\angle$\ce{CCC}~\ang{115.8}\\
%\midrule
\midrule\chemfig{H_3C-[:-30]-[:+30]-[:-30]CH_3} / \(\SymGroup{C}{2h}\) &\cmpd{Butane} &  \num{-158.084303} & \num{0.133309} & 
\ce{H_3C-CH_2}~\SI{1.524}{\angstrom}; 
\ce{H_2C-CH_2}~\SI{1.523}{\angstrom}; 
$\angle$\ce{CCC}~\ang{112.3} \\
\midrule\chemfig{H_3C-[:-30]O-[:+30]-[:-30]CH_3} / \(\SymGroup{C}{s}\) & \cmpd{Methylethylether} &  \num{-193.966394} & \num{0.108895} & 
\ce{H3C-O}~\SI{1.412}{\angstrom};
\ce{H_2C-O}~\SI{1.417}{\angstrom};
\ce{H2C-CH_3}~\SI{1.509}{\angstrom};
$\angle$\ce{COC}~\ang{111.2};
$\angle$\ce{OCC}~\ang{108.1} \\
\midrule\chemfig{H_3C-[:-30]S-[:+30]-[:-30]CH_3} / \(\SymGroup{C}{s}\) & \cmpd{Methylethylthioether} &  \num{-516.576661} & \num{0.105402} &
\ce{H_3C-S}~\SI{1.805}{\angstrom}; 
\ce{H_2C-S}~\SI{1.809}{\angstrom}; 
\ce{H_2C-CH_3}~\SI{1.519}{\angstrom}; 
$\angle$\ce{CSC}~\ang{98.7}; 
$\angle$\ce{SCC}~\ang{109.3} \\
\midrule\chemfig{H_3C-[:-30]C(=[:-90]O)-[:+30]-[:-30]CH_3} / \(\SymGroup{C}{s}\) & \cmpd{Butanone} & \num{-232.014552} & \num{0.113700} & \ce{C=O}~\SI{1.220}{\angstrom};
\ce{H_3C-CO}~\SI{1.509}{\angstrom}; 
\ce{H_2C-CO}~\SI{1.513}{\angstrom}; 
\ce{H_3C-CH_2}~\SI{1.517}{\angstrom};
$\angle$\ce{OCC(H_3)}~\ang{121.9};
$\angle$\ce{OCC(H_2)}~\ang{122.1}; 
$\angle$\ce{CC(O)C}~\ang{116.0}; 
$\angle$\ce{C(O)CC}~\ang{113.4} \\
\midrule\chemfig{H_3C-[:-30](-[:-90]CH_3)-[:+30]-[:-30]CH_3}  & \cmpd{Isopentane} & \num{-197.311142} & \num{0.161457} & \\
\midrule \chemfig{H_3C-[:-30]N(-[:-90]CH_3)-[:+30]-[:-30]CH_3} & \cmpd{Dimethylethylamine} & & & \\
\midrule\chemfig{H_3C-[:-30](-[:-90]CH_3)-[:+30]O-[:-30]CH_3} & \cmpd{Methylisopropylether} & \num{-233.195233} & \num{0.137450} &  \\
\midrule\chemfig{H_3C-[:-30](-[:-90]CH_3)-[:+30]S-[:-30]CH_3} & \cmpd{Methylisopropylthioether} & & & \\
%\midrule\cmpd{Isopentanone} &\chemfig{H_3C-[:+30](-[:+90]CH_3)-[:-30]C(=[:-90]O)-[:+30]CH_3} & & & \\
\end{longtabu}\label{tab:Reagents:Opt:End}

В частности, для 3,7-аналогов 1-азабицикло[3.3.1]нонана
\begin{center}
  \chemfig{[:-30]*6(-N(-[:180]-[:120]X-[:+60]-[:0]?)--Y---)} +
  
  + \chemfig{H_3C-[:-30]X-[:+30]CH_3} + \chemfig{H_3C-[:-30]Y-[:+30]CH_3}
  + 5\(\cdot\)\chemfig{H_3C-[:-30]-[:+30]CH_3}
  + 2\(\cdot\)\chemfig{H_3C-[:-30](-[:-90]CH_3)-[:+30]CH_3}
  + 2\(\cdot\)\chemfig{H_3C-[:-30]N(-[:-90]CH_3)-[:+30]CH_3}
  \(\longrightarrow\)
  
  \(\longrightarrow\) 
  2\(\cdot\)\chemfig{H_3C-[:+30]X-[:-30]-[:+30]CH_3} + 2\(\cdot\)\chemfig{H_3C-[:+30]Y-[:-30]-[:+30]CH_3} + 3\(\cdot\)\chemfig{H_3C-[:+30](-[:+90]CH_3)-[:-30]-[:+30]CH_3} + 
  3\(\cdot\)\chemfig{H_3C-[:+30]N(-[:+90]CH_3)-[:-30]-[:+30]CH_3}
\end{center}

и 1,5-диаазабицикло[3.3.1]нонана:
\begin{center}
  \chemfig{[:-30]*6(-N(-[:180]-[:120]X-[:+60]-[:0]\phantom{N}?)--Y--N-)}
  
  + \chemfig{H_3C-[:-30]X-[:+30]CH_3} + \chemfig{H_3C-[:-30]Y-[:+30]CH_3} + 5\(\cdot\) \chemfig{H_3C-[:-30]-[:+30]CH_3}
  + 2\(\cdot\)\chemfig{H_3C-[:+30](-[:+90]CH_3)-[:-30]CH_3}
  + 2\(\cdot\)\chemfig{H_3C-[:+30]N(-[:+90]CH_3)-[:-30]CH_3}
  \(\longrightarrow\)
  
  \(\longrightarrow\) 
  2\(\cdot\)\chemfig{H_3C-[:+30]X-[:-30]-[:+30]CH_3} + 2\(\cdot\)\chemfig{H_3C-[:+30]Y-[:-30]-[:+30]CH_3} + 
  6\(\cdot\)\chemfig{H_3C-[:+30]N(-[:+90]CH_3)-[:-30]-[:+30]CH_3}
\end{center}

\section{Напряжения свободных шестичленных циклов}

Реакция разделения связей для 1,3- и 1,4-дизамещённых аналогов формально одинаковы:

\begin{center}
  \begin{equation*} \underbrace{\text{\chemfig{[:-30]X*6(---Y---)}} \qquad \text{\chemfig{[:-30]*6(-X----Y-)}}}_{\text{как бы это всё поизяЧнее сделать?}} \end{equation*}
+ \chemfig{H_3C-[:-30]X-[:+30]CH_3} + \chemfig{H_3C-[:-30]Y-[:+30]CH_3} + 4\(\cdot\) \chemfig{H_3C-[:-30]-[:+30]CH_3} \(\longrightarrow\)
  
  \(\longrightarrow\) 
    2\(\cdot\)\chemfig{H_3C-[:+30]X-[:-30]-[:+30]CH_3} +     2\(\cdot\)\chemfig{H_3C-[:+30]-[:-30]-[:+30]CH_3} + 2\(\cdot\)\chemfig{H_3C-[:+30]Y-[:-30]-[:+30]CH_3}
\end{center}

Реакции \emph{позиционного диспропорционирования} дизамещённых аналогов циклогексана

\begin{center}
  \chemfig{*6(-X---Y--)} + \chemfig{*6(------)} \(\longrightarrow\) \chemfig{*6(-X-----)} + \chemfig{*6(----Y--)}
\end{center}
  
B таблице~\ref{tab:Cycle:Six:Opt} на с.~\pageref{tab:Cycle:Six:Opt:Begin}--\pageref{tab:Cycle:Six:Opt:End} приводятся результаты оптимизации молекул производных циклогексана, включая эндоциклические торсионные углы $\tau$.

\label{tab:Cycle:Six:Opt:Begin}\begin{longtabu} to \textwidth {cr|SS|X[l]}
  %\caption{Оптимизированные энергии и структурные параметры молекул}
  %\caption{Оптимизированные (MP2 / aug-cc-pVTZ) энергии и геометрические параметры молекул}
  %\small
  \toprule
  \multicolumn{2}{c|}{\label{tab:Cycle:Six:Opt} Структура / симметрия}  & \(E_{2}\),~\si{\hartree} & \(E_{ZPE}\),~\si{\hartree} & Геометрия \endfirsthead
  \multicolumn{5}{p{\textwidth}}{\emph{продолжение таблицы}~\arabic{table}} \\
  \toprule\multicolumn{2}{c|}{Структура / симметрия} & \(E_{2}\),~\si{\hartree} & \(E_{ZPE}\),~\si{\hartree} & Геометрия \endhead
  \midrule\multicolumn{5}{l}{\emph{продолжение на следующей странице}} \endfoot
  \bottomrule\multicolumn{5}{l}{} \endlastfoot
  \midrule\chemfig{?-[:+160]-[:-120]<[:+20]-[:-20,,,,line width=\boldbondwidth]>[:+60]?} / \SymGroup{D}{3d} & \cmpd{Cyclohexane} & \num{-235.344605} & & \ce{C-C}~\SI{1.526}{\angstrom}; $\angle$\ce{CCC}~\ang{110.7}; $\tau=\pm\ang{56.9}$ \\
  \midrule\chemfig{?-[:+160]-[:-120]O<[:+20]-[:-20,,,,line width=\boldbondwidth]>[:+60]O?} / \SymGroup{C}{2h} & \cmpd{Dioxane14} & & & \ce{C-C} \\
  \midrule\chemfig{?-[:+160]-[:-120]S<[:+20]-[:-20,,,,line width=\boldbondwidth]>[:+60]S?} / \SymGroup{C}{2h} & \cmpd{Dithiane} & & & \ce{C-C} \\
  \midrule\chemfig{?-[:-120]C(=[:-135]O)<[:+20]-[:-20,,,,line width=\boldbondwidth]>[:+60]C(=[:+45]O)-[:-160]?} / \SymGroup{C}{2h} & \cmpd{Cyclohexandione14} & & & \ce{C-C} \\
  \midrule\chemfig{?-[:-120](-[:+155]H_3C)(-[:-90,0.75]H)<[:+20]-[:-20,,,,line width=\boldbondwidth]>[:+60](-[:+90,0.75]H)(-[:-25]CH_3)-[:-160]?} / \SymGroup{C}{2h} & \cmpd{Me2CyclohexaneEE} & & & \ce{C-C} \\
  \midrule\chemfig{?-[:-120]N(-[:+155]H_3C)<[:+20]-[:-20,,,,line width=\boldbondwidth]>[:+60]N(-[:-25]CH_3)-[:-160]?} / \SymGroup{C}{2h} & \cmpd{N2Me2PiperazineEE} & & & \ce{C-C} \\
  \midrule\chemfig{?-[:+160]-[:-120]O<[:+20]-[:-20,,,,line width=\boldbondwidth]>[:+60]?} / \SymGroup{C}{s} & \cmpd{Pyran} & & & \ce{C-C} \\
  \midrule\chemfig{?-[:+160]-[:-120]S<[:+20]-[:-20,,,,line width=\boldbondwidth]>[:+60]?} / \SymGroup{C}{s} & \cmpd{Thiopyran} & & & \ce{C-C} \\
  \midrule\chemfig{?-[:+160]-[:-120]C(=[:-135]O)<[:+20]-[:-20,,,,line width=\boldbondwidth]>[:+60]?} / \SymGroup{C}{s} & \cmpd{Cyclohexanone} & & & \ce{C-C} \\
  \midrule\chemfig{?-[:+160]-[:-120](-[:+155]H_3C)(-[:-90,0.75]H)<[:+20]-[:-20,,,,line width=\boldbondwidth]>[:+60]?} / \SymGroup{C}{s} & \cmpd{NMeCyclohexaneEq} & & & \ce{C-C} \\
  \midrule\chemfig{?-[:+160]-[:-120]N(-[:+155]H_3C)<[:+20]-[:-20,,,,line width=\boldbondwidth]>[:+60]?} / \SymGroup{C}{s} & \cmpd{NMePiperidineEq} & & & \ce{C-C} \\
\end{longtabu}\label{tab:Cycle:Six:Opt:End}

Энергии напряжения... \EStrain