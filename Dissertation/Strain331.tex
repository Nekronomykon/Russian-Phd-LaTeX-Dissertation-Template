\chapter{Исследование напряжений в производных бицикло[3.3.1]нонана}

Реакции разделения связей. Наименьшую погрешность обеспечивают \emph{гипергомодесмотические} реакции, сохраняющие не только состав, как гомодесмические... но и число связей различных типов. В частности, для 3,7,9-замещённых аналогов бицикло[3.3.1]нонана выводится уравнение
\begin{center}
  \chemfig{[:-30]Z*6(-(-[:180]-[:120]X-[:+60]-[:0]?)--Y---)}
  + \chemfig{H_3C-[:-30]X-[:+30]CH_3} + \chemfig{H_3C-[:-30]Y-[:+30]CH_3} + \chemfig{H_3C-[:-30]Z-[:+30]CH_3} + 4\(\cdot\) \chemfig{H_3C-[:-30]-[:+30]CH_3}
  + 4\(\cdot\)\chemfig{H_3C-[:+30](-[:+90]CH_3)-[:-30]CH_3}
  \(\longrightarrow\)
  
  \(\longrightarrow\) 
  2\(\cdot\)\chemfig{H_3C-[:+30]X-[:-30]-[:+30]CH_3} + 2\(\cdot\)\chemfig{H_3C-[:+30]Y-[:-30]-[:+30]CH_3} + 4\(\cdot\)\chemfig{H_3C-[:+30](-[:+90]CH_3)-[:-30]-[:+30]CH_3} + 
  2\(\cdot\)\chemfig{H_3C-[:+30](-[:+90]CH_3)-[:-30]Z-[:+30]CH_3}
\end{center}

Для 1,3,5,7-аналогов исследуемой бициклической системы
\begin{center}
  \chemfig{[:-30]*6(-A(-[:180]-[:120]X-[:+60]-[:0]\phantom{Z}?)--Y--Z-)}
  + \chemfig{H_3C-[:-30]X-[:+30]CH_3} + \chemfig{H_3C-[:-30]Y-[:+30]CH_3} + 5\(\cdot\) \chemfig{H_3C-[:-30]-[:+30]CH_3}
  + 2\(\cdot\)\chemfig{H_3C-[:+30]A(-[:+90]CH_3)-[:-30]CH_3}
  + 2\(\cdot\)\chemfig{H_3C-[:+30]Z(-[:+90]CH_3)-[:-30]CH_3}
  \(\longrightarrow\)
  
  \(\longrightarrow\) 
  2\(\cdot\)\chemfig{H_3C-[:+30]X-[:-30]-[:+30]CH_3} + 2\(\cdot\)\chemfig{H_3C-[:+30]Y-[:-30]-[:+30]CH_3} + 3\(\cdot\)\chemfig{H_3C-[:+30]A(-[:+90]CH_3)-[:-30]-[:+30]CH_3} + 3\(\cdot\)\chemfig{H_3C-[:+30]Z(-[:+90]CH_3)-[:-30]-[:+30]CH_3}
\end{center}

Энергетический эффект этих реакций, взятый с обратным знаком, оценивает общее напряжение молекулы относительно ациклических форм, считающихся свободными от напряжения. 

Для более подробного рассмотрения отдельных взаимодействий необходимо комбинировать эти уравнения с другими.

Расчёты 

\begin{equation}
E = E_A^\infty + E_{ZPE} + E_k
\end{equation}

Результаты расчёта реагентов

%\caption{\label{tab:Desmotic:Reagents}Результаты расчёта реагентов для реакций разделения связей}
\begin{longtabu} to \textwidth {rlrr X}
\toprule
\# & Молекула / Симм. & \(E_{tot}\)(A),~\si{\hartree} & \(E_{ZPE}\),~\si{\hartree} & Геометрия \\
\midrule\cmpd{Propane} & \chemfig{H_3C-[:-30]-[:+30]CH_3} / \(\SymGroup{C}{2v}\) & & & \\
\midrule\cmpd{Dimethylether} & \chemfig{H_3C-[:-30]O-[:+30]CH_3} / \(\SymGroup{C}{2v}\) & & & \\
\midrule\cmpd{Dimethylsulfide} & \chemfig{H_3C-[:-30]S-[:+30]CH_3} / \(\SymGroup{C}{2v}\) & & & \\
%\midrule\cmpd{Dimethylamine} & \chemfig{H_3C-[:-30]N(-[:-90]H)-[:+30]CH_3} & \(\SymGroup{C}{s}\) & \\
\midrule\cmpd{Isobutane} & \chemfig{H_3C-[:-30](-[:-90]CH_3)-[:+30]CH_3} / \(\SymGroup{C}{3v}\) & & & \\
\midrule\cmpd{Trimethylamine} & \chemfig{H_3C-[:-30]N(-[:-90]CH_3)-[:+30]CH_3} / \(\SymGroup{C}{3v}\) & & & \\
\midrule\cmpd{Acetone} & \chemfig{H_3C-[:-30]C(=[:-90]O)-[:+30]CH_3} / \(\SymGroup{C}{2v}\) & & & \\
%\midrule
\midrule\cmpd{Butane} & \chemfig{H_3C-[:+30]-[:-30]-[:+30]CH_3} / \(\SymGroup{C}{2h}\) & & & \\
\midrule\cmpd{Methylethylether} & \chemfig{H_3C-[:+30]O-[:-30]-[:+30]CH_3} / \(\SymGroup{C}{s}\) & & & \\
\midrule\cmpd{Methylethylthioether} & \chemfig{H_3C-[:+30]S-[:-30]-[:+30]CH_3} / \(\SymGroup{C}{s}\) & & & \\
\midrule\cmpd{Butanone} & \chemfig{H_3C-[:+30]-[:-30]C(=[:-90]O)-[:+30]CH_3} / \(\SymGroup{C}{s}\) & & & \\
\midrule\cmpd{Isopentane} &\chemfig{H_3C-[:+30](-[:+90]CH_3)-[:-30]-[:+30]CH_3} / \(\SymGroup{C}{1}\) & & & \\
\midrule\cmpd{Isopentanone} &\chemfig{H_3C-[:+30](-[:+90]CH_3)-[:-30]C(=[:-90]O)-[:+30]CH_3} / \(\SymGroup{C}{1}\) & & & \\
\midrule\cmpd{Methylisopropylether} &\chemfig{H_3C-[:+30](-[:+90]CH_3)-[:-30]O-[:+30]CH_3} / \(\SymGroup{C}{1}\) & & & \\
\midrule\cmpd{Methylisopropylthioether} &\chemfig{H_3C-[:+30](-[:+90]CH_3)-[:-30]S-[:+30]CH_3} / \(\SymGroup{C}{1}\) & & & \\
\midrule\cmpd{Isopentane} &\chemfig{H_3C-[:+30](-[:+90]CH3)-[:-30]-[:+30]CH_3} / \(\SymGroup{C}{1}\) & & & \\
\midrule\cmpd{Dimethylethylamine} & \chemfig{H_3C-[:+30]N(-[:+90]CH_3)-[:-30]-[:+30]CH_3} / \(\SymGroup{C}{1}\) & & & \\
\bottomrule
\end{longtabu}

В частности, для 3,7-аналогов 1-азабицикло[3.3.1]нонана
\begin{center}
  \chemfig{[:-30]*6(-N(-[:180]-[:120]X-[:+60]-[:0]?)--Y---)}
  + \chemfig{H_3C-[:-30]X-[:+30]CH_3} + \chemfig{H_3C-[:-30]Y-[:+30]CH_3} + 5\(\cdot\) \chemfig{H_3C-[:-30]-[:+30]CH_3}
  + 2\(\cdot\)\chemfig{H_3C-[:-30](-[:-90]CH_3)-[:+30]CH_3}
  + 2\(\cdot\)\chemfig{H_3C-[:-30]N(-[:-90]CH_3)-[:+30]CH_3}
  \(\longrightarrow\)
  
  \(\longrightarrow\) 
  2\(\cdot\)\chemfig{H_3C-[:+30]X-[:-30]-[:+30]CH_3} + 2\(\cdot\)\chemfig{H_3C-[:+30]Y-[:-30]-[:+30]CH_3} + 3\(\cdot\)\chemfig{H_3C-[:+30](-[:+90]CH_3)-[:-30]-[:+30]CH_3} + 
  3\(\cdot\)\chemfig{H_3C-[:+30]N(-[:+90]CH_3)-[:-30]-[:+30]CH_3}
\end{center}

и 1,5-диаазабицикло[3.3.1]нонана:
\begin{center}
  \chemfig{[:-30]*6(-N(-[:180]-[:120]X-[:+60]-[:0]\phantom{N}?)--Y--N-)}
  + \chemfig{H_3C-[:-30]X-[:+30]CH_3} + \chemfig{H_3C-[:-30]Y-[:+30]CH_3} + 5\(\cdot\) \chemfig{H_3C-[:-30]-[:+30]CH_3}
  + 2\(\cdot\)\chemfig{H_3C-[:+30](-[:+90]CH_3)-[:-30]CH_3}
  + 2\(\cdot\)\chemfig{H_3C-[:+30]N(-[:+90]CH_3)-[:-30]CH_3}
  \(\longrightarrow\)
  
  \(\longrightarrow\) 
  2\(\cdot\)\chemfig{H_3C-[:+30]X-[:-30]-[:+30]CH_3} + 2\(\cdot\)\chemfig{H_3C-[:+30]Y-[:-30]-[:+30]CH_3} + 
  6\(\cdot\)\chemfig{H_3C-[:+30]N(-[:+90]CH_3)-[:-30]-[:+30]CH_3}
\end{center}

Шестичленные циклы 

\begin{center}
  \chemfig{[:-30]X*6(---Y---)} + \chemfig{H_3C-[:-30]X-[:+30]CH_3} + \chemfig{H_3C-[:-30]Y-[:+30]CH_3} + 4\(\cdot\) \chemfig{H_3C-[:-30]-[:+30]CH_3} \(\longrightarrow\)
  
  \(\longrightarrow\) 
    2\(\cdot\)\chemfig{H_3C-[:+30]X-[:-30]-[:+30]CH_3} +     2\(\cdot\)\chemfig{H_3C-[:+30]-[:-30]-[:+30]CH_3} + 2\(\cdot\)\chemfig{H_3C-[:+30]Y-[:-30]-[:+30]CH_3}
\end{center}

