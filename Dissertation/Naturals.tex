\section{Природные соединения, содержащие скелет бицикло[$3.3.1$]нонана}

\subsubsection{Хинолизидиновые алкалоиды}

Одним из представителей трициклических алкалоидов
является цитизин~\cmpd{Cytisine}. Родственные ему ромбифолин~\cmpd{Rhombifoline}. Тетрациклические: анагирин~\cmpd{Anagyrine}, термопсин~\cmpd{Thermopsine}, баптифолин~\cmpd{Baptifoline} и эпибаптифолин~\cmpd{Epibaptifoline}.~\cite{Goller:2019} 

\begin{center}
\begin{tabular}{ccc}
\chemfig{*6(-C(=[,0.75]O)-N(*6(--(-[:+30]-[:+90]NH-[:+150]?[a])<:>:?[a]-))-=-=)} & 
\chemfig{*6(-C(=[,0.75]O)-N(*6(--(-[:+30]-[:+90]N (-[:+30]-[:-30]-[:-90]=[:-150,0.75]H_2C)-[:+150]?[a])<:>:?[a]-))-=-=)} & \\
\cmpd{Cytisine} &
\cmpd{Rhombifoline} & 
\\
\end{tabular}
\end{center}

\begin{center}
\begin{tabular}{ccc}
    \chemfig{*6(-C(=[,0.75]O)-N(*6(--(-[:+30](*6(-----N)) (<:[:-90,0.75]H)-[:+90]\phantom{N}-[:+150]?[a])<>?[a]-))-=-=)} & 
    \chemfig{*6(-C(=[,0.75]O)-N(*6(--(-[:+30](*6(-----N)) (<[:-90,0.75]H)-[:+90]\phantom{N}-[:+150]?[a])<>?[a]-))-=-=)} & \\
\cmpd{Anagyrine} & \cmpd{Thermopsine} & \\
  \end{tabular}
\end{center}

\begin{center}
\begin{tabular}{ccc}
  \chemfig{*6(-C(=[,0.75]O)-N(*6(--(-[:+30](*6(--(<:[:-30,0.875]OH)---N)) (<:[:-90,0.75]H)-[:+90]\phantom{N}-[:+150]?[a])<>?[a]-))-=-=)} &
  \chemfig{*6(-C(=[,0.75]O)-N(*6(--(-[:+30](*6(--(<[:-30,0.875]OH)---N)) (<:[:-90,0.75]H)-[:+90]\phantom{N}-[:+150]?[a])<>?[a]-))-=-=)} & \\
  \cmpd{Baptifoline} & \cmpd{Epibaptifoline} & \\
  \end{tabular}
\end{center}

Пахикарпин или (-)-спартеин~\cmpd{Pachycarpine}, (+)-спартеин~\cmpd{Sparteine}.

\begin{center}
  \begin{tabular}{ccc}
    \chemfig{*6(--N(*6(--(-[:+30](*6(-----N)) (<:[:-90,0.75]H)-[:+90]\phantom{N}-[:+150]?[a])<>?[a]-))-(<[:+90,0.75]H)---)} & 
    \chemfig{*6(--N(*6(--(-[:+30](*6(-----N)) (<[:-90,0.75]H)-[:+90]\phantom{N}-[:+150]?[a])<:>:?[a]-))-(<:[:+90,0.75]H)---)} & \\
    \cmpd{Pachycarpine} & \cmpd{Sparteine} & \\
  \end{tabular}
\end{center}

Трициклический алкалоид ангустифолин~\cmpd{Angustifoline} по данным исследования ЯЭО и других экспериментов ЯМР представляет собой структуру с бициклом [$3.3.1$] в конформации~\CC{}, которая стабилизируется внутримолекулярным мостиком \ce{N\bond{...}H-N}.~\cite{Wysocka:1994} Аналогичная картина наблюдается в кристаллической структуре гидроиодида $\alpha$-изолупанина~\ce{{\cmpd{IsolupanineA}}.HI}~\cite{Koziol:1986}
 
\begin{center}
\begin{tabular}{ccc}
\chemfig{*6(-C(=[,0.75]O)-N(*6(--(-[:+30] (-[:-30]-[:+30,0.75]=[:+90,0.75]CH_2)
  (<:[:-90,0.75]H)-[:+90]NH-[:+150]?[a])<>?[a]-))-(<[:+90,0.75]H)---)} & 
\chemfig{*6(-C(=[,0.75]O)-N(*6(--(-[:+30] (*6(-----N))  (<[:-90,0.75]H)-[:+90]\phantom{N}-[:+150]?[a])<>?[a]-))-(<[:+90,0.75]H)---)} &
\\
\cmpd{Angustifoline} & \cmpd{IsolupanineA} &  \\
  \end{tabular}
\end{center}

\subsubsection{Изохинолиновые алкалоиды}

Производные морфинана содержат, как правило, фрагмент 2-азабицикло$[3.3.1]$нонана.

\subsubsection{Терпеноиды и терпеновые алкалоиды}

Структурное разнообразие изопреноидных метаболитов...

Ряд клована "--- карбобицикл.

Ряд пачуленола и другие. Конформационные затруднения.

Атидан~\cmpd{Atidane} и атизин~\cmpd{Atisine} "--- 3-азабицикло[3.3.1]нонан.  

%Атидан\index{атидан}~\cmpd{Atidane}
\begin{center}
  \chemfig{*6(-\chembelow{N}{H}--?[b] (-[:+90]-[:+150]?[a])-(<[:-60]-[:+20,,,,line width=\boldbondwidth]>[:-30](-[:-90]?[d])(-[:+30,1.5]-[:+0,1.125]?[c] (<[:-45,0.75]H)-[:+90]CH_3)-[:+120](?[b]) (-[:+90,0.75]H)<:[:-0]-[:-45,1.5]?[c](-[:+0,0.75]H)-[:-120]?[d])-(<[:+160,0.75]H) (-[:+90]?[a])-)}

\cmpd{Atidane}
\end{center}

Атизин\index{атизин}~\cmpd{Atisine}
\begin{center}
  \chemfig{*6(-N(*5(---O-))-(<:[:-90,0.75]H)-?[b] (-[:+90]-[:+150]?[a])-(<[:-60]-[:+20,,,,line width=\boldbondwidth]>[:-30](-[:-90]?[d]) (-[:+30,1.5](-[:+90]OH)(<[:-45,0.75]H)-[:+0,1.125]?[c]=[:+15]CH_2)-[:+120](?[b]) (-[:+90,0.75]H)<:[:-0]-[:-45,1.5]?[c](-[:+0,0.75]H)-[:-120]?[d])-(<[:+160]H_3C) (-[:+90]?[a])-)}

\cmpd{Atisine}
\end{center}?

Аконитановые алкалоиды, производные~\cmpd{Aconitane}, помимо системы 3-азабицикло[3.3.1]нонан, содержат ещё и карбобицикл; обе эти подструктуры не имеют общих атомов.

\begin{center}
  \chemfig{*6(-\chembelow{N}{H}-(<:[:-90,0.75]H)(*6(-?[c](<:[:-90,0.75]H)-(<[:-90,0.75]H) (*6(---(<[:+0,0.75]H) (-[:+120]?[b])--))-(<[:-30,0.75]H)-(<[:+90,0.75]H)(-[:+30]?[b])-))-(-[:+90]-[:+150]?[a])-(<[:-60]?[c,{<}])-(<[:+160,0.75]H) (-[:+90]?[a])-)}
  
  \cmpd{Aconitane}
\end{center}

