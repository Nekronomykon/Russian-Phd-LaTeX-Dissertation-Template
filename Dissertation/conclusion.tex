\chapter*{Заключение}                       % Заголовок
\addcontentsline{toc}{chapter}{Заключение}  % Добавляем его в оглавление

%% Согласно ГОСТ Р 7.0.11-2011:
%% 5.3.3 В заключении диссертации излагают итоги выполненного исследования, рекомендации, перспективы дальнейшей разработки темы.
%% 9.2.3 В заключении автореферата диссертации излагают итоги данного исследования, рекомендации и перспективы дальнейшей разработки темы.
%% Поэтому имеет смысл сделать эту часть общей и загрузить из одного файла в автореферат и в диссертацию:

Основные результаты работы заключаются в следующем.
%% Согласно ГОСТ Р 7.0.11-2011:
%% 5.3.3 В заключении диссертации излагают итоги выполненного исследования, рекомендации, перспективы дальнейшей разработки темы.
%% 9.2.3 В заключении автореферата диссертации излагают итоги данного исследования, рекомендации и перспективы дальнейшей разработки темы.
\begin{enumerate}
  \item Неэмпирическими расчётными методами было исследовано конформационное поведение бицикло$[3.3.1]$нонана, его замещённых производных и насыщенных гетероциклических аналогов.
  \item Для анализа закономерностей конформационных эффектов замещения аппарат гипергомодесмотических реакций, разработанный Уилером для каркасных и полиненасыщенных углеводородов, был распространён на насыщенные гетероциклы и гетеробициклические соединения.
  \item На основе анализа \ldots
  \item Численные исследования показали, что \ldots
  \item Математическое моделирование показало \ldots
  \item Для выполнения поставленных задач был создан \ldots
\end{enumerate}

И какая-нибудь заключающая фраза.

Последний параграф может включать благодарности.  В заключение автор выражает благодарность и большую признательность научному руководителю Иванову~И.\:И. за~поддержку, помощь, обсуждение результатов и~научное руководство. Также автор благодарит Сидорова~А.\:А. и~Петрова~Б.\:Б. за~помощь в~работе с~образцами, Рабиновича~В.\:В. за предоставленные образцы и~обсуждение результатов, Занудятину~Г.\:Г. и авторов шаблона \tqt{Russian-Phd-LaTeX-Dissertation-Template} для работы в системе \LaTeX{} за~помощь в оформлении диссертации. Автор также благодарит много разных людей и~всех, кто сделал настоящую работу автора возможной. Соответствующие алфавиты следует поблагодарить за предоставленные буквы, шрифты "--- за начертания, а древних кхмеров "--- за цифры, включая \tqt{ноль}~(0).
