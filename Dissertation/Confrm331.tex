% !TeX spellcheck = ru_RU
% !TeX encoding = UTF-8
\chapter{Конформационное поведение бицикло[3.3.1]нонанов и их аналогов}\label{ch:Conformation:331}

\section{Конформационые энергии, эффекты и строение конформеров}

Корреляционные диаграммы конформационной энергии аналогов бицикло[3.3.1]нонана\dots~\cite{Pisarev:2019}

Применение QTAIM к анализу~\cite{Bushmarinov:UX:2009,Bushmarinov:2011}

\newcommand{\ScaleCorrStr}{0.625}

\begin{figure*}
  \caption{Конформационная энергия для некоторых вариантов симметричного гетероаналогичного замещения в бицикло[3.3.1]нонане (MP4(2)/aug-cc-pVDZ + ZPE,~ккал/моль)~\cite{Pisarev:2013:rus,Pisarev:2013}}\label{fig:Conform331:MP2}
  \centering
  \begin{tikzpicture}[xscale=3.75,yscale=1.75] %
    \DrawCorrFrame{-2}{6};
\draw(1.50,6.50) node [anchor=north,draw=black,fill=white]{\schemestart %
\chemfig{[0,\ScaleCorrStr]?[a]-[:+30,,,,line width=\boldbondwidth] (>[:+120]-[:-120](-[:-150]-[:+105,0.5]?[a,{>}])(-[:-30]-[:-60]?[b]))-[:-30,,,,line width=\boldbondwidth]?[b,{<}]} %
\arrow{->} %
\chemfig{?[a]-[:+30,,,,line width=\boldbondwidth](>[:+120]-[:-120](-[:-150]-[:-60]?[a,{>}])(-[:-30]-[:-60]?[b]))-[:-30,,,,line width=\boldbondwidth]?[b,{<}]} % 
\schemestop %
};
\draw(1.50,-1.00) node [anchor=north,draw=black,fill=white]{\schemestart %
\chemfig{?[a]-[:+30,,,,line width=\boldbondwidth] (>[:+120]-[:-120](-[:-150]-[:+105,0.5]?[a,{>}])(-[:-30]-[:-60]?[b]))-[:-+30,,,,line width=\boldbondwidth]?[b,{<}]} %
\arrow{<-} %
\chemfig{[0,\ScaleCorrStr]?[a]-[:+30,,,,line width=\boldbondwidth](>[:+120]-[:-120](-[:-150]-[:-60]?[a,{>}])(-[:-30]-[:-60]?[b]))-[:-30,,,,line width=\boldbondwidth]?[b,{<}]} % 
  \schemestop %
};
%
    % [O x 2O]
    \begin{scope}[red,dashed]
      \DrawQuadCorrelation{2.21}{1.11}{1.21}{3.23}
    \end{scope}
    % [O x 2S]
    \begin{scope}[red,dotted]
      \DrawQuadCorrelation{2.21}{1.11}{-1.44}{-1.53}
    \end{scope}
    % [S x 2O]
    \begin{scope}[black,dashed]
      \DrawQuadCorrelation{2.21}{1.26}{1.21}{1.55}
    \end{scope}
    % [S x 2S]
    \begin{scope}[black,dotted]
      \DrawQuadCorrelation{2.21}{1.26}{-1.44}{-0.47}
    \end{scope}
    % [2N x 2O]
    \begin{scope}[blue,dashed]
      \DrawQuadCorrelation{2.21}{4.06}{1.21}{5.38}
    \end{scope}
    % [2N x 2S]
    \begin{scope}[blue,dotted]
      \DrawQuadCorrelation{2.21}{4.06}{-1.44}{3.01}
    \end{scope}
    %structures:
    % bicyclo[3.3.1]nonane
\draw(1.00, 2.21) node [anchor=south] {\chemfig{[:-30,\ScaleCorrStr]*6(-(-[:180]-[:120]-[:+60]-[:0]?)-----)}};
\draw(0.85, 2.21) node [anchor=east]  {\cmpd{Bicycle331}};
\draw(1.00, 2.21) node [anchor=north] {\num{2.21}};
    % 1,5-diazabicyclo[3.3.1]nonane
\draw(2.00, 4.06) node [anchor=north] {\chemfig{[:-30,\ScaleCorrStr]*6(-N(-[:180]-[:120]-[:+60]-[:0]\phantom{N}?)----N-)}};
\draw(2.00, 4.06) node [anchor=south] {\num{4.06}};
\draw(2.15, 4.06) node [anchor=west]  {\cmpd{Bicycle331:15N2}};
    % 
\draw(2.00, 1.35) node [anchor=south] {\chemfig{[:-30,\ScaleCorrStr]S*6(-(-[:180]-[:120]-[:+60]-[:0]?)-----)}};
\draw(1.85, 1.35)node[anchor=east]{\num{1.26}};
\draw(2.15, 1.35) node [anchor=west]  {\cmpd{Bicycle331:9S}};
    %
\draw(2.00, 1.00) node [anchor=north] {\chemfig{[:-30,\ScaleCorrStr]O*6(-(-[:180]-[:120]-[:+60]-[:0]?)-----)}};
\draw(1.85, 1.00)node[anchor=east]{\num{1.11}};
\draw(2.15, 1.00) node [anchor=west]  {\cmpd{Bicycle331:9O}};
    %
\draw(3.00, 1.21) node [anchor=north] {\chemfig{[:-30,\ScaleCorrStr]*6(-(-[:180]-[:120]O-[:+60]-[:0]?)--O---)}};
\draw(3.00, 1.21)node[anchor=south]{\num{1.21}};
    %
\draw(3.00,-1.44) node [anchor=south] {\chemfig{[:-30,\ScaleCorrStr]*6(-(-[:180]-[:120]S-[:+60]-[:0]?)--S---)}};
\draw(3.00,-1.44) node [anchor=north] {\num{-1.44}};
\draw(3.15,-1.44) node [anchor=west]  {\cmpd{Bicyclo331:37S2}};
    %
\draw(4.00, 5.38) node [anchor=south] {\chemfig{[:-30,\ScaleCorrStr]*6(-N(-[:180]-[:120]O-[:+60]-[:0]\phantom{N}?)--O--N-)}};
\draw(4.00, 5.38)node[anchor=north] {\num{5.38}};
\draw(3.85, 5.38) node [anchor=east]  {\cmpd{Bicyclo331:37O2:15N2}};
    %
\draw(4.00, 3.30) node [anchor=south] {\chemfig{[:-30,\ScaleCorrStr]O*6(-(-[:180]-[:120]O-[:+60]-[:0]?)--O---)}};
\draw(4.15, 3.30) node [anchor=west] {\num{3.23}};
\draw(3.85, 3.30) node [anchor=east] {\cmpd{Bicyclo331:379O3}};
    %
\draw(4.00, 2.95) node [anchor=north] {\chemfig{[:-30,\ScaleCorrStr]*6(-N(-[:180]-[:120]S-[:+60]-[:0]\phantom{N}?)--S--N-)}};
\draw(4.15, 2.95)node[anchor=west]{\num{3.01}};
\draw(3.85, 2.95) node [anchor=east]  {\cmpd{Bicyclo331:37S2:15N2}};
    %
\draw(4.00, 1.55) node [anchor=north] {\chemfig{[:-30,\ScaleCorrStr]S*6(-(-[:180]-[:120]O-[:+60]-[:0]?)--O---)}};
\draw(4.00, 1.55) node [anchor=south] {\num{1.55}};
    %
\draw(4.00,-0.47) node [anchor=south] {\chemfig{[:-30,\ScaleCorrStr]S*6(-(-[:180]-[:120]S-[:+60]-[:0]?)--S---)}};
\draw(4.00,-0.47) node [anchor=north] {\num{-0.47}};
\draw(3.85,-0.47) node [anchor=east]  {\cmpd{Bicyclo331:379S3}};
    %
\draw(4.00,-1.53) node [anchor=north] {\chemfig{[:-30,\ScaleCorrStr]O*6(-(-[:180]-[:120]S-[:+60]-[:0]?)--S---)}};
\draw(4.00,-1.53)node[anchor=south]{\num{-1.53}};
    %
  \end{tikzpicture}
\end{figure*}

\begin{longtabu} to \textwidth {c|c|SS|S|X[l]}
  \multicolumn{6}{p{\textwidth}}{Таблица\label{tab:Biycle:331:Opt}~\arabic{table}. Расчётное строение аналогов бицикло$[3.3.1]$нонана (RI-MP2 / aug-cc-pVTZ)} \\ %  
  \toprule\midrule
  \multicolumn{2}{c|}{Структура}  & \(E_{2}\),~\si{\hartree} & \(E_{ZPE}\),~\si{\hartree} & \(\mu\) & Геометрия \endfirsthead
  \multicolumn{6}{p{\textwidth}}{\emph{продолжение таблицы}~\thetable:} \\
  \toprule\multicolumn{2}{c|}{Структура} & \(E_{2}\),~\si{\hartree} & \(E_{ZPE}\),~\si{\hartree} & \(\mu\) & Геометрия \endhead
  \midrule\multicolumn{6}{l}{\emph{продолжение на следующей странице}}\endfoot
  \bottomrule\multicolumn{6}{l}{}\endlastfoot
  \midrule
  \cmpd{Bicycle331} & \BC{}~\(\SymGroup{C}{s}\) & -351.821711 & 0.237460 & 0.117 & \\
  \chemfig{[:-30,0.75]*6(-(-[:180]-[:120]-[:+60]-[:0]?)-----)} & \CC{}~\(\SymGroup{C}{2}\) & -351.826668 & & 0.024 & \\
  & \TT{}~\(\SymGroup{C}{2}\) & & & & \\
  \midrule
\cmpd{Bicycle331:9O} & \BC{}~\(\SymGroup{C}{s}\) & & & & \\
\chemfig{[:-30,0.75]O*6(-(-[:180]-[:120]-[:+60]-[:0]?)-----)} & \CC{}~\(\SymGroup{C}{2}\) & & & & \\
& \TT{}~\(\SymGroup{C}{2}\) & & & & \\
  \midrule
\cmpd{Bicycle331:9S} & \BC{}~\(\SymGroup{C}{s}\) & & & & \\
\chemfig{[:-30,0.75]S*6(-(-[:180]-[:120]-[:+60]-[:0]?)-----)} & \CC{}~\(\SymGroup{C}{2}\) & & & & \\
& \TT{}~\(\SymGroup{C}{2}\) & & & & \\
\bottomrule
\end{longtabu}\label{tab:Biycle:331:Opt:Ends}

\section*{*~*~*}

\begin{table}
  \caption{Ещё одна иллюстрация конформационного поведения аналогов бицикло[3.3.1]нонана}\label{fig:Conformational:Behavior:331:XYZ}
  \centerfloat{
    \begin{tabu} to \textwidth {|X[c]|X[c]|X[c]|}
      \toprule
      \BC{} & \CC{} & \CB{} \\
      \midrule
      \multicolumn{3}{c}{
        \schemestart
        %\ChemPicture
        \chemfig{X?[a]<[:-30,1.25]-[:+30,,,,line width=\boldbondwidth](-[:+45,0.75,,,line width=\boldbondwidth]R')(>[:+120]Z-[:-120](-[:+135,0.75]R)(-[:-150]?[a])(-[:-30]-[:-60]Y?[b]))-[:-30,,,,line width=\boldbondwidth]?[b,{<}]
        }
        %\ChemPicture
        \arrow{<=>}
        % \ChemPicture
        \chemfig{X?[a]<[:+60]-[:+30,,,,line width=\boldbondwidth](-[:+45,0.75,,,line width=\boldbondwidth]R')(>[:+120]Z-[:-120](-[:+135,0.75]R)(-[:-150]?[a])(-[:-30]-[:-60]Y?[b]))-[:-+30,,,,line width=\boldbondwidth]?[b,{<}]} % 
        \arrow{<=>}
        %\ChemPicture
        \chemfig{X?[a]<[:+60]-[:+30,,,,line width=\boldbondwidth](-[:+45,0.75,,,line width=\boldbondwidth]R')(>[:+120]Z-[:-120](-[:+135,0.75]R)(-[:-150]?[a])(-[:-30]-[:+30,1.25]Y?[b]))-[:-+30,,,,line width=\boldbondwidth]?[b,{<}]}
        \schemestop
      } 
      \\
      \midrule
      \multicolumn{3}{c}{
        \schemestart
        %\ChemPicture
        \chemfig{X?[a]<[:-30,1.25]-[:+30,,,,line width=\boldbondwidth]N(>[:+120]-[:-120](-[:+135,0.75]R)(-[:-150]?[a])(-[:-30]-[:-60]Y?[b]))-[:-30,,,,line width=\boldbondwidth]?[b,{<}]
        }
        %\ChemPicture
        \arrow{<=>}
        % \ChemPicture
        \chemfig{X?[a]<[:+60]-[:+30,,,,line width=\boldbondwidth]N(>[:+120]-[:-120](-[:+135,0.75]R)(-[:-150]?[a])(-[:-30]-[:-60]Y?[b]))-[:-30,,,,line width=\boldbondwidth]?[b,{<}]} % 
        \arrow{<=>}
        %\ChemPicture
        \chemfig{X?[a]<[:+60]-[:+30,,,,line width=\boldbondwidth]N(>[:+120]-[:-120](-[:+135,0.75]R)(-[:-150]?[a])(-[:-30]-[:+30,1.25]Y?[b]))-[:-30,,,,line width=\boldbondwidth]?[b,{<}]}
        \schemestop
      } 
      \\      \midrule
      \multicolumn{3}{c}{
        \schemestart
        %\ChemPicture
        \chemfig{X?[a]<[:-30,1.25]-[:+30,,,,line width=\boldbondwidth]N(>[:+120]-[:-120]N(-[:-150]?[a])(-[:-30]-[:-60]Y?[b]))-[:-30,,,,line width=\boldbondwidth]?[b,{<}]
        }
        %\ChemPicture
        \arrow{<=>}
        % \ChemPicture
        \chemfig{X?[a]<[:+60]-[:+30,,,,line width=\boldbondwidth]N(>[:+120]-[:-120]N(-[:-150]?[a])(-[:-30]-[:-60]Y?[b]))-[:-30,,,,line width=\boldbondwidth]?[b,{<}]} % 
        \arrow{<=>}
        %\ChemPicture
        \chemfig{X?[a]<[:+60]-[:+30,,,,line width=\boldbondwidth]N(>[:+120]-[:-120]N(-[:-150]?[a])(-[:-30]-[:+30,1.25]Y?[b]))-[:-30,,,,line width=\boldbondwidth]?[b,{<}]}
        \schemestop
      } 
      \\ \bottomrule
    \end{tabu}
  }
\end{table}

