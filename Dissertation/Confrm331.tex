\chapter{Конформационное поведение бицикло[3.3.1]нонанов и их аналогов}



\begin{figure}
  \centerfloat{
    \begin{tabu} to \textwidth {|X[c]|X[c]|X[c]|}
      \toprule
      \BC{} & \CC{} & \CB{} \\
      \midrule
      \multicolumn{3}{c}{
      \schemestart
      %\ChemPicture
      \chemfig{X?[a]<[:-30,1.25]-[:+30,,,,line width=\boldbondwidth](-[:+45,0.75,,,line width=\boldbondwidth]R')(>[:+120]Z-[:-120](-[:+135,0.75]R)(-[:-150]?[a])(-[:-30]-[:-60]Y?[b]))-[:-+30,,,,line width=\boldbondwidth]?[b,{<}]
      }
      %\ChemPicture
      \arrow{<=>}
      % \ChemPicture
      \chemfig{X?[a]<[:+60]-[:+30,,,,line width=\boldbondwidth](-[:+45,0.75,,,line width=\boldbondwidth]R')(>[:+120]Z-[:-120](-[:+135,0.75]R)(-[:-150]?[a])(-[:-30]-[:-60]Y?[b]))-[:-+30,,,,line width=\boldbondwidth]?[b,{<}]} % 
      \arrow{<=>}
      %\ChemPicture
      \chemfig{X?[a]<[:+60]-[:+30,,,,line width=\boldbondwidth](-[:+45,0.75,,,line width=\boldbondwidth]R')(>[:+120]Z-[:-120](-[:+135,0.75]R)(-[:-150]?[a])(-[:-30]-[:+30,1.25]Y?[b]))-[:-+30,,,,line width=\boldbondwidth]?[b,{<}]}
      \schemestop
      } 
    \\      \midrule
    \multicolumn{3}{c}{
      \schemestart
      %\ChemPicture
      \chemfig{X?[a]<[:-30,1.25]-[:+30,,,,line width=\boldbondwidth]N(>[:+120]-[:-120](-[:+135,0.75]R)(-[:-150]?[a])(-[:-30]-[:-60]Y?[b]))-[:-+30,,,,line width=\boldbondwidth]?[b,{<}]
      }
      %\ChemPicture
      \arrow{<=>}
      % \ChemPicture
      \chemfig{X?[a]<[:+60]-[:+30,,,,line width=\boldbondwidth]N(>[:+120]-[:-120](-[:+135,0.75]R)(-[:-150]?[a])(-[:-30]-[:-60]Y?[b]))-[:-+30,,,,line width=\boldbondwidth]?[b,{<}]} % 
      \arrow{<=>}
      %\ChemPicture
      \chemfig{X?[a]<[:+60]-[:+30,,,,line width=\boldbondwidth]N(>[:+120]-[:-120](-[:+135,0.75]R)(-[:-150]?[a])(-[:-30]-[:+30,1.25]Y?[b]))-[:-+30,,,,line width=\boldbondwidth]?[b,{<}]}
      \schemestop
    } 
    \\      \midrule
    \multicolumn{3}{c}{
      \schemestart
      %\ChemPicture
      \chemfig{X?[a]<[:-30,1.25]-[:+30,,,,line width=\boldbondwidth]N(>[:+120]-[:-120]N(-[:-150]?[a])(-[:-30]-[:-60]Y?[b]))-[:-+30,,,,line width=\boldbondwidth]?[b,{<}]
      }
      %\ChemPicture
      \arrow{<=>}
      % \ChemPicture
      \chemfig{X?[a]<[:+60]-[:+30,,,,line width=\boldbondwidth]N(>[:+120]-[:-120]N(-[:-150]?[a])(-[:-30]-[:-60]Y?[b]))-[:-+30,,,,line width=\boldbondwidth]?[b,{<}]} % 
      \arrow{<=>}
      %\ChemPicture
      \chemfig{X?[a]<[:+60]-[:+30,,,,line width=\boldbondwidth]N(>[:+120]-[:-120]N(-[:-150]?[a])(-[:-30]-[:+30,1.25]Y?[b]))-[:-+30,,,,line width=\boldbondwidth]?[b,{<}]}
      \schemestop
    } 
    \\ \bottomrule
    \end{tabu}
  }
\caption{\label{fig:Conformational:Behavior:331:XYZ}Конформационное поведение аналогов бицикло[3.3.1]нонана}
\end{figure}

В настоящее время большинство физико-химических методов исследования структуры использует результаты неэмпирического моделирования строения молекул по крайней мере на этапе формирования структурных гипотез. Такая методология подразумевает выбор в пользу той или иной гипотезы на основании экспериментальных измерений.

\section{Конформационые энергии, эффекты и строение конформеров}