\chapter{Конформационное поведение производных и гетероаналогов бицикло[3.3.1]нонана (обзор литературы)}

\epigraph{Quia igitur huius mundi machina sphaerica est, dicendum est in primis quid sit sphaera?}{De sphaera Lincolniensem Secundum tractatus, cap. I}

Под термином \tqt{\emph{циклическая система}} здесь и в дальнейшем мы подразумеваем циклы и полициклические каркасы в общем, рассматривая прежде всего их остовы, состоящие обычно из тяжёлых (неводородных) атомов.

\begin{figure}
  \centering
  %\vspace{\bigskipamount}
  \vspace{\bigskipamount}
  Двухкоординированнные~(тип~I)\qquad\chemfig{>[:+30]CH_2-[:+150]}

  \vspace{\medskipamount}
  \begin{tabular}{cc|cc|}
    \chemfig{>[:+30]B(-H)-[:+150]} & 
    \chemfig{>[:+30]B(-R)-[:+150]} &
    \chemfig{>[:+30]C(-[:+55]R)(-[:-55]R{'})-[:+150]} &
    \chemfig{>[:+30]C(=X)-[:+150]} \\
  \end{tabular}
  ~
  \begin{tabular}{cc|c|}
    \chemfig{>[:+30]N(-H)-[:+150]} & 
    \chemfig{>[:+30]N(-R)-[:+150]} & 
    \chemfig{>[:+30]O-[:+150]} \\
  \end{tabular}
  ~
  \begin{tabular}{c|ccc|}
    \chemfig{>[:+30]Si(-[:+55]R)(-[:-55]R{'})-[:+150]} &
    \chemfig{>[:+30]P(-H)-[:+150]} & \chemfig{>[:+30]P(-R)-[:+150]} & \chemfig{>[:+30]P(-[:+55]R)(=[:-55]O)-[:+150]} \\
  \end{tabular}
  ~
  \begin{tabular}{ccc|}
    \chemfig{>[:+30]S-[:+150]} & \chemfig{>[:+30]S(=O)-[:+150]} &
    \chemfig{>[:+30]S(=[:+55]O)(=[:-55]O)-[:+150]} \\
  \end{tabular}
  
  \vspace{\bigskipamount}
  Трёхкоординированные~(тип~II)\qquad\chemfig{C(<[:-120])(<:[:-60])(-[:+150]H)-[:+30]}

  \vspace{\medskipamount}
  \begin{tabular}{c|c|cc|c|cc}
    \chemfig{B(<[:-120])(<:[:-60])-[:+90]} & \chemfig{C(<[:-120])(<:[:-60])(-[:+150]R)-[:+30]} &   \chemfig{N(<[:-120])(<:[:-60])-[:+30]} & \chemfig{N^+(<[:-120])(<:[:-60])(-[:+150]R)-[:+30]} & \chemfig{Si(<[:-120])(<:[:-60])(-[:+150]R)-[:+30]} &  \chemfig{P(<[:-120])(<:[:-60])-[:+30]} & \chemfig{P(<[:-120])(<:[:-60])(=[:+150]O)-[:+30]}
  \end{tabular}
  \vspace{\medskipamount}
  \caption{\label{fig:Atom:Types}Стереохимическая типизация атомов в циклических системах}
\end{figure}

Мы используем широкое понятие о \tqt{замещении} в циклических системах; то есть, если дополнительно не оговаривается иное, мы имеем в виду под замещением не только введение или изменение заместителя (\emph{замещение} в~узком смысле слова), но и \emph{гетероаналогию} или \emph{гетероаналогичное замещение} атома \ce{C} соответствующего циклического углеводорода на~гетероатом того же стереохимического типа~(рис.~\ref{fig:Atom:Types}). 

Конформационный анализ молекулярных структур часто включает их разбиение на~подструктуры с дальнейшим выражением конформации структуры в~целом как суперпозиции конформаций подструктур. Система бицикло[3.3.1]нонана~\cmpd{Bicycle331} допускает несколько вариантов подобных разбиений~(рис.~\ref{fig:Decomposition:331}):

\begin{enumerate}
\item\label{item:331:Decomposition:6:6} Два 1,3-конденсированных шестичленных цикла (здесь и~в~дальнейшем циклы будем обозначать вместе с~направлением обхода): $\text{\CycleFirst} = \left(5\rightarrow 9\rightarrow 1\rightarrow 2\rightarrow 3\rightarrow 4\right)$ и~$\text{\CycleSecond} = \left(1\rightarrow 9\rightarrow 5\rightarrow 6\rightarrow 7\rightarrow 8\right)$;
\item\label{item:331:Decomposition:8:1} Восьмичленный цикл~$\text{\CycleThird} = \left(8\rightarrow 7\rightarrow 6\rightarrow 5\rightarrow 4\rightarrow 3\rightarrow 2\rightarrow 1\right)$ с одноатомным мостиком $-9-$ между положениями 1 и 5;
\item\label{item:331:Decomposition:2x5:3} Два пятичленных \emph{псевдоцикла} (группы из атомов, трактуемых как цикл, но не обязательно соединённых связями в соответствующем порядке) $\text{\CyclePseudoFirst}=(1\rightarrow 2\rightarrow 3\rightarrow 4\rightarrow 5)$ и $\text{\CyclePseudoSecond}=(5\rightarrow 6\rightarrow 7\rightarrow 8\rightarrow 1)$, примыкающих к трёхчленному псевдоциклу $(1\rightarrow 9\rightarrow 5)$;
\item\label{item:331:Decomposition:2x2:2} Три связанных трёхатомных фрагмента: $2-3-4$, $8-7-6$ (т.\,н. \tqt{крылья}) и $1-9-5$;
\end{enumerate}

\begin{figure}
\centerfloat{
\begin{tabular}{|c|c|c|c|}
%\toprule
\multicolumn{4}{c}{\chemfig{[:-30,1.25]9*6(-1(-[:180]2-[:120]3-[:+60]4-[:0]\phantom{5}?)-8-7-6-5-)}} \\
\multicolumn{4}{c}{\cmpd{Bicycle331} } \\
\midrule
\ref{item:331:Decomposition:6:6} & \ref{item:331:Decomposition:8:1} &
\ref{item:331:Decomposition:2x5:3} & \ref{item:331:Decomposition:2x2:2} \\
\chemfig{[:-30]3?-[:-60]2-[:0]1-[:+120]9-[:+60]5-[:+180]4?}\chemfig{[:-30]9*6(-1-8-7-6-5-)} &
\chemfig{[:-30]9*6(-[,,,,dash pattern=on 1pt off 1pt]1(-[:180]2-[:120]3-[:+60]4-[:0]\phantom{5}?)-8-7-6-5-[,,,,dash pattern=on 1pt off 1pt])} & 
\chemfig{[:-30]9*6(-1?[a](-[:180]2-[:120]3-[:+60]4-[:0]\phantom{5}?[a,1,{dash pattern=on 1pt off 1pt}])-8-7-6-5-)} &
\chemfig{[:-30]9*6(-1(-[:180,,,,dash pattern=on 1pt off 1pt]2-[:120]3-[:+60]4-[:0,,,,dash pattern=on 1pt off 1pt]\phantom{5}?)-[,,,,dash pattern=on 1pt off 1pt]8-7-6-[,,,,dash pattern=on 1pt off 1pt]5-)} \\
$\leftarrow$\qquad$\longleftarrow$ & $\longrightarrow$ & $\leftarrow$\quad$\rightarrow$ & \\
\CycleFirst \qquad \CycleSecond & \CycleThird & \CyclePseudoFirst\quad\CyclePseudoSecond & \\ 
\bottomrule
\end{tabular}
\vspace{\bigskipamount}
}
\caption{\label{fig:Decomposition:331} Номенклатура положений и структурные декомпозиции системы бицикло[3.3.1]нонана. Для циклов {\CycleFirst}--{\CyclePseudoSecond} указано направление обхода }
\end{figure}

Конформация бициклического скелета~\cmpd{Bicycle331} традиционно определяется согласно разбиению~\ref{item:331:Decomposition:6:6}, т.\,е. из двух составляющих его конформаций шестичленных циклов. Поэтому нам кажется закономерным вначале кратко сформулировать хорошо известные основные положения конформационного анализа шестичленных циклов.

\begin{figure}
\centerfloat{
\begin{tabular}{c|cc|c}
\toprule\cmpd{Bicycle331:XYZ}: & 
\multicolumn{3}{c}{\chemfig{[:-30]Z*6(-(-[:180]-[:120]X-[:+60]-[:0]?)--Y---)}} \\
\midrule & \ConfName{К} & \ConfName{Т} & \ConfName{В} \\
\midrule\CycleFirst: & 
\chemfig{X?<[:-60]-[:+20,,,,line width=\boldbondwidth]>[:-20]Z-[:+120]-[:-160]?} & 
\chemfig{X?-[:-150,0.75]<[:-30,0.75]-[:+30,1.5,,,line width=\boldbondwidth]Z>[:-30,0.75]-[:-150,0.75]?} &
\chemfig{X(-[:-35]?)<[:-60]-[:+0,1.5,,,line width=\boldbondwidth]>[:+60]Z-[:-145]?}  \\
\chemfig{[:-30]X*6(---Z---)} &
&
\chemfig{?-[:-150,0.75]X<[:-30,0.75]-[:+30,1.5,,,line width=\boldbondwidth]>[:-30,0.75]Z-[:-150,0.75]?} &
\chemfig{(-[:-35]?)<[:-60]X-[:+0,1.5,,,line width=\boldbondwidth]>[:+60]-[:-145]Z?} \\
\midrule\CycleSecond: & 
\chemfig{Y?<[:-60]-[:+20,,,,line width=\boldbondwidth]>[:-20]Z-[:+120]-[:-160]?}&
\chemfig{Y?-[:-150,0.75]<[:-30,0.75]-[:+30,1.5,,,line width=\boldbondwidth]Z>[:-30,0.75]-[:-150,0.75]?} &
\chemfig{Y(-[:-35]?)<[:-60]-[:+0,1.5,,,line width=\boldbondwidth]>[:+60]Z-[:-145]?}  \\
\chemfig{[:-30]Y*6(---Z---)} & 
&
\chemfig{?-[:-150,0.75]Y<[:-30,0.75]-[:+30,1.5,,,line width=\boldbondwidth]>[:-30,0.75]Z-[:-150,0.75]?} &
\chemfig{(-[:-35]?)<[:-60]Y-[:+0,1.5,,,line width=\boldbondwidth]>[:+60]-[:-145]Z?} \\
\midrule\cmpd{Cyclohexane}: &
\ChemPicture{?<[:-60]-[:+20,,,,line width=\boldbondwidth]>[:-20]-[:+120]-[:-160]?} & 
\ChemPicture{?-[:-150,0.75]<[:-30,0.75]-[:+30,1.5,,,line width=\boldbondwidth]>[:-30,0.75]-[:-150,0.75]?} &
\ChemPicture{(-[:-35]?)<[:-60]-[:+0,1.5,,,line width=\boldbondwidth]>[:+60]-[:-145]?} \\
\ce{X}=\ce{Y}=\ce{CH2} & ($\SymGroup{D}{3d}$) & ($\SymGroup{D}{2}$) & ($\SymGroup{C}{2v}$) \\
\bottomrule
\end{tabular}
\vspace{\medskipamount}
}
\caption{\label{fig:Conformations:Six}Базисные конформации шестичленных циклов для 1,4-дизамещённых аналогов циклогексана~\cmpd{Cyclohexane}}
\end{figure}

Для пространственных форм насыщенных шестичленных циклов выделяются три класса так называемых \emph{базисных} конформаций: \tqt{кресло}~(\ConfName{К}), \tqt{твист}~(\ConfName{T}) и \tqt{ванна}~(\ConfName{В})~(рис.~\ref{fig:Conformations:Six}); остальные формы являются промежуточными и могут быть выражены как \emph{выпуклые} комбинации этих трёх. Оптимальной является структура \ConfName{К}; \ConfName{В} обычно является переходным состоянием процесса взаимопревращения двух конформеров со~структурой \ConfName{T}; соответствующий процесс называется \emph{пседовращением} шестичленного цикла. Причиной такой нестабильности \tqt{ванны} признано сильное повышение энергии структур с ординарными связями в заслонённой конформации (так называемое \emph{питцеровское напряжение}). 

\begin{figure}
  \centerfloat{
    \begin{tabular}{cc}
      \multicolumn{2}{c}{\ChemPicture{?[a]<[:-30,1.5]-[:+30,,,,line width=\boldbondwidth](>[:+120]-[:-120](-[:-150]?[a])(-[:-30]-[:+30,1.25]?[b]))-[:-+30,,,,line width=\boldbondwidth]?[b,{<}]}} \\
      \multicolumn{2}{c}{\BB{}~(\TT{})} \\
      \midrule\ChemPicture{?[a]<[:-30,1.25]-[:+30,,,,line width=\boldbondwidth](>[:+120]-[:-120](-[:-150]?[a])(-[:-30]-[:-60]?[b]))-[:-+30,,,,line width=\boldbondwidth]?[b,{<}]} & 
      \ChemPicture{?[a]<[:+60]-[:+30,,,,line width=\boldbondwidth](>[:+120]-[:-120](-[:-150]?[a])(-[:-30]-[:+30,1.25]?[b]))-[:-+30,,,,line width=\boldbondwidth]?[b,{<}]} \\
      \multicolumn{2}{c}{\BC{} \(\equiv\) \CB{}} \\
      \midrule\multicolumn{2}{c}{\ChemPicture{?[a]<[:+60]-[:+30,,,,line width=\boldbondwidth](>[:+120]-[:-120](-[:-150]?[a])(-[:-30]-[:-60]?[b]))-[:-+30,,,,line width=\boldbondwidth]?[b,{<}]} } \\
      \multicolumn{2}{c}{\CC{}} \\
    \end{tabular}
  }
  \vspace{\medskipamount}
  \caption{\label{fig:System331:Conf}Основные конформации бицикло[3.3.1]нонана~\cmpd{Bicycle331}}
\end{figure}

Для насыщенных производных~\cmpd{Bicycle331} выделяются конформации \tqt{двойное кресло}~(\CC{}), \tqt{ванна-кресло}~(\BC{}), \tqt{кресло-ванна} (\CB{}), \tqt{кресло-ванна}~(\CB{}) и~\tqt{двойной ванны} (\BB{})~(рис.~\ref{fig:System331:379XYZ:Conf}). В случае структурной эквивалентности \tqt{крыльев} (например, для исходного углеводорода~\cmpd{Bicycle331}) наблюдается \emph{конформационное вырождение} "--- формы \CB{} и~\BC{} становятся структурно и~энергетически неразличимыми~(рис.~\ref{fig:System331:Conf}).

\begin{figure}
\centerfloat{
\begin{tabular}{c|c}
\multicolumn{2}{c}{
\ChemPicture{X?[a]<[:-30,1.5]-[:+30,,,,line width=\boldbondwidth](>[:+120]Z-[:-120](-[:-150]?[a])(-[:-30]-[:+30,1.25]Y?[b]))-[:-+30,,,,line width=\boldbondwidth]?[b,{<}]} 
}
\\
\multicolumn{2}{c}{\BB{}~(\TT{})} \\
\midrule
\ChemPicture{X?[a]<[:-30,1.25]-[:+30,,,,line width=\boldbondwidth](>[:+120]Z-[:-120](-[:-150]?[a])(-[:-30]-[:-60]Y?[b]))-[:-+30,,,,line width=\boldbondwidth]?[b,{<}]
}
& 
\ChemPicture{X?[a]<[:+60]-[:+30,,,,line width=\boldbondwidth](>[:+120]Z-[:-120](-[:-150]?[a])(-[:-30]-[:+30,1.25]Y?[b]))-[:-+30,,,,line width=\boldbondwidth]?[b,{<}]}
\\
\BC{} & \CB{}
\\
\midrule
\multicolumn{2}{c}{ %
\ChemPicture{X?[a]<[:+60]-[:+30,,,,line width=\boldbondwidth](>[:+120]Z-[:-120](-[:-150]?[a])(-[:-30]-[:-60]Y?[b]))-[:-+30,,,,line width=\boldbondwidth]?[b,{<}]} % 
}
\\
\multicolumn{2}{c}{\CC{}}
\\
\end{tabular}
}
\vspace{\medskipamount}
\caption{\label{fig:System331:379XYZ:Conf}Основные конформации 3X,~7Y,~9Z-аналогов бицикло[3.3.1]нонана}
\end{figure}

Форма \BB{} у~производных бицикло[3.3.1]нонана из-за питцеровских напряжений встречается очень редко (характерна для сильно затруднённых и напряжённых молекул), и в этих случаях обычно наблюдается экспериментально в виде формы «двойного твиста» (\TT{}).

Торсионные напряжения также являются фактором устойчивости и для формы \CB{}, содержащей подструктуру \tqt{ванны}, в которой, однако, заслонённые конформации связей незначительно отклоняются от идеальной формы \ConfName{В} с эндоциклическими двугранными углами $\tau_{9123}\simeq 0$ и $\tau_{9543}\simeq 0$. Часто эти отклонения симметричны, т. е., $\tau_{9123} = - \tau_{9543}$. Всё это является, в частности, показателем стереохимической (конформационной) жёсткости формы \tqt{кресла} насыщенного шестичленного цикла. Такая жёсткость образуется из сочетания низкой энергии и высоких собственных частот колебания структуры. Жёсткие конформации одной из подструктур молекулы могут стабилизировать менее энергетически выгодные формы других как термодинамически, так и кинетически.

\begin{figure}
\centerfloat{
}
\caption{\label{fig:Interactions:37}3,7-взаимодействия в молекулах аналогов бицикло[3.3.1]нонана}
\end{figure}

Конформация \CC{} состоит из кресловидных шестичленных циклов, почти свободных от питцеровского напряжения. Основным фактором устойчивости для этой конформации оказывается невалентное взаимодействие между \tqt{крыльями} бициклической системы. Особенно значима в~этом смысле роль положений 3 и 7~(рис.~\ref{fig:Interactions:37}). Дисперсионное или электростатическое отталкивание между этими положениями или \emph{эндо}-заместителями в~них приводит к~повышению энергии и~дестабилизации \CC{} относительно других форм, тогда как притяжение стабилизирует соответствующую структуру.

\begin{figure}
\centerfloat{}
\caption{\label{fig:Interactions:2468}2,4~(6,8)-взаимодействия в молекулах аналогов бицикло[3.3.1]нонана}
\end{figure}

Дополнительным фактором, влияющим на относительные энергии конформеров \BC{}/\CB{} и~\CC{} являются взаимодействия положений 2 и~4 (6 и~8) внутри \tqt{крыльев}~(рис.~\ref{fig:Interactions:2468}).). Они частично сходны с 1,3-диаксиальными взаимодействиями в соответствующих конформерах 1,3-\emph{цис}-дизамещённых аналогов циклогексана. Отталкивание приводит к уплощению циклов и сближению двух упомянутых форм.

Максимальная симметрия бициклического скелета~\cmpd{Bicycle331}, порождённая из~симметрической группы $\SymGroup{S}{9}$ как группа автоморфизмов соответствующего молекулярного графа, порождается системой орбит (эквивалентных вершин) вида $\left\langle(2\,4\,6\,8)(1\,5)(3\,7)(9)\right\rangle\subset\SymGroup{S}{9}$. В эвклидовом трёхмерном (\tqt{мировом}) пространстве $\SymGroup{E}{3}\left(\AGroup{R}\right)\simeq\AGroup{R}^3$ эта группа (изоморфно реализуется) представлена (?)  в~виде точечной группы симметрии \(\SymGroup{C}{2v}\) для~\CC{} и~\BB{}. Для других конформаций симметрия снижается до~подгрупп \(\SymGroup{C}{s}\) у~\BC{}/\CB{} и~$\SymGroup{C}{2}$ у~\TT{}.

Конформационное поведение производных бицикло[3.3.1]нонана в общем случае определяется взаимопревращением конформеров, в которых бициклический скелет приобретает либо конформации \CC{}, либо \BC{}/\CB{}.

