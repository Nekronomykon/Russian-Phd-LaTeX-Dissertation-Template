% !TeX spellcheck = ru_RU
% !TeX encoding = UTF-8
\chapter{Конформационное поведение производных и гетероаналогов бицикло[3.3.1]нонана (обзор литературы)}\label{ch:Review:Basics}

\epigraph{Quia igitur huius mundi machina sphaerica est, dicendum est in primis quid sit sphaera?}{De sphaera Lincolniensem Secundum tractatus, cap. I}

Под термином \tqt{\emph{циклическая система}} здесь и в дальнейшем мы подразумеваем молекулярные циклы и полициклические каркасы, рассматриваемые как часть молекулы, и прежде всего "--- их остовы, состоящие обычно из тяжёлых (неводородных) атомов.

\begin{figure}
  \caption{Стереохимическая типизация атомов в циклических системах\label{fig:Atom:Types}}

  \centerfloat{
  Двухкоординированнные~(тип~I)

  \ChemPicture{[,1.25]>[:+30]\circ-[:+150]}

  \begin{tabular}{cc|cc|}
    \chemfig{>[:+30]B(-H)-[:+150]} & 
    \chemfig{>[:+30]B(-R)-[:+150]} &
    \chemfig{>[:+30]C(-[:+55]R)(-[:-55]R{'})-[:+150]} &
    \chemfig{>[:+30]C(=[:0,0.875]X)-[:+150]} \\
  \end{tabular}
  ~
  \begin{tabular}{cc|c|}
    \chemfig{>[:+30]N(-H)-[:+150]} & 
    \chemfig{>[:+30]N(-R)-[:+150]} & 
    \chemfig{>[:+30]O-[:+150]} \\
  \end{tabular}

  \begin{tabular}{c|ccc|}
    \chemfig{>[:+30]Si(-[:+55]R)(-[:-55]R{'})-[:+150]} &
    \chemfig{>[:+30]P(-H)-[:+150]} & \chemfig{>[:+30]P(-R)-[:+150]} & \chemfig{>[:+30]P(-[:+55]R)(=[:-55,0.875]O)-[:+150]} \\
  \end{tabular}
  ~
  \begin{tabular}{ccc|}
    \chemfig{>[:+30]S-[:+150]} & \chemfig{>[:+30]S(=[:0,0.875]O)-[:+150]} &
    \chemfig{>[:+30]S(=[:+55,0.875]O)(=[:-55,0.875]O)-[:+150]} \\
  \end{tabular}
  
  \vspace{\bigskipamount}

  Трёх- и четырёхкоординированные~(тип~II)

  \ChemPicture{[,1.25]\bullet(<[:-120])(<:[:-60])(-[:+150])-[:+30]}
  
  \begin{tabular}{c|c|cc|c|cc}
    \chemfig{B(<[:-120])(<:[:-60])-[:+90]} & \chemfig{C(<[:-120])(<:[:-60])(-[:+150])-[:+30]} &   \chemfig{N(<[:-120])(<:[:-60])-[:+30]} & \chemfig{N^+(<[:-120])(<:[:-60])(-[:+150])-[:+30]} & \chemfig{Si(<[:-120])(<:[:-60])(-[:+150])-[:+30]} &  \chemfig{P(<[:-120])(<:[:-60])-[:+30]} & \chemfig{P(<[:-120])(<:[:-60])(=[:+150,0.875]O)-[:+30]}
  \end{tabular}
}
% \vspace{\medskipamount}
\end{figure}

Мы используем широкое понятие о \tqt{замещении} в циклических системах; то есть, если дополнительно не оговаривается иное, мы имеем в виду под замещением не только введение или изменение заместителя (\emph{замещение} в~узком смысле слова), но и \emph{гетероаналогию} или \emph{гетероаналогичное замещение} атома \ce{C} соответствующего циклического углеводорода на~гетероатом того же стереохимического типа~(рис.~\ref{fig:Atom:Types}). 

Различие между типами состоит в~том, что простые, свободные циклы состоят только из~атомов типа~I, а~би- и~полициклические каркасы обнаруживают в~своём составе не~менее двух атомов типа~II.

Исследования конформационного поведения аналогов бицикло[3.3.1]нонана до~конца 1980-х годов обобщены в~\cite{Zefirov:1991a}. В последнее время появились работы

Конформационный анализ молекулярных структур часто включает их разбиение на~подструктуры с~дальнейшим выражением конформации структуры в~целом как~суперпозиции конформаций подструктур. Система бицикло[3.3.1]нонана~\cmpd{Bicycle331} допускает несколько вариантов подобных разбиений~(рис.~\ref{fig:Decomposition:331}):

\begin{enumerate}
\item\label{item:331:Decomposition:IUPAC} Два атома 1 и~5 c тремя соединяющими их мостиками $1-2-3-4-5$, $1-8-7-6-5$ и~$1-9-5$ (такое разбиение соответствует международному обозначению бицикла [3.3.1]);
\item\label{item:331:Decomposition:6:6} Два 1,3-конденсированных шестичленных цикла (здесь и~в~дальнейшем циклы будем обозначать вместе с~направлением обхода): $\text{\CycleFirst} = \left(5\rightarrow 9\rightarrow 1\rightarrow 2\rightarrow 3\rightarrow 4\right)$ и~$\text{\CycleSecond} = \left(1\rightarrow 9\rightarrow 5\rightarrow 6\rightarrow 7\rightarrow 8\right)$;
\item\label{item:331:Decomposition:8:1} Восьмичленный цикл~$\text{\CycleThird} = \left(8\rightarrow 7\rightarrow 6\rightarrow 5\rightarrow 4\rightarrow 3\rightarrow 2\rightarrow 1\right)$ с одноатомным мостиком $-9-$ между положениями 1 и 5;
\item\label{item:331:Decomposition:2x5:3} Два пятичленных \emph{псевдоцикла} (группы из~атомов, трактуемых как~цикл, но~не~обязательно соединённых связями в~соответствующем порядке) $\text{\CyclePseudoFirst}=(1\rightarrow 2\rightarrow 3\rightarrow 4\rightarrow 5)$ и $\text{\CyclePseudoSecond}=(5\rightarrow 6\rightarrow 7\rightarrow 8\rightarrow 1)$, примыкающих к трёхчленному псевдоциклу $(1\rightarrow 9\rightarrow 5)$;
\item\label{item:331:Decomposition:2x2:2} Три связанных трёхатомных фрагмента: $2-3-4$, $8-7-6$ (т.\,н.~\tqt{крылья}) и \tqt{мостик} $1-9-5$;
\end{enumerate}

\begin{figure}
\caption{Номенклатура положений и структурные декомпозиции системы бицикло[3.3.1]нонана\label{fig:Decomposition:331} %. Для циклов {\CycleFirst--\CycleThird}, {\CyclePseudoFirst} и~{\CyclePseudoSecond} указано направление обхода
}

%\centering
\centerfloat{
\begin{tabular}{|c|c|c|c|}
%\toprule
\multicolumn{4}{c}{\chemfig{[:-30,1.25]\circ*6(-\bullet(-[:180]\circ-[:120]\circ-[:+60]\circ-[:0]\phantom{\bullet}?)-\circ-\circ-\circ-\bullet-)} \qquad = \qquad \chemfig{[:-30,1.25]9*6(-1(-[:180]2-[:120]3-[:+60]4-[:0]\phantom{5}?)-8-7-6-5-)}} \\
\multicolumn{4}{c}{ \ref{item:331:Decomposition:IUPAC} } \\
\multicolumn{4}{c}{} \\
%\midrule
\ref{item:331:Decomposition:6:6} & \ref{item:331:Decomposition:8:1} &
\ref{item:331:Decomposition:2x5:3} & \ref{item:331:Decomposition:2x2:2} \\
\chemfig{[:-30,0.875]3?-[:-60]2-[:0]1-[:+120]9-[:+60]5-[:+180]4?}\chemfig{[:-30,0.875]9*6(-1-8-7-6-5-)} &
\chemfig{[:-30,0.875]9*6(-[,,,,dash pattern=on 1pt off 1pt]1(-[:180]2-[:120]3-[:+60]4-[:0]\phantom{5}?)-8-7-6-5-[,,,,dash pattern=on 1pt off 1pt])} & 
\chemfig{[:-30,0.875]9*6(-1?[a](-[:180]2-[:120]3-[:+60]4-[:0]\phantom{5}?[a,1,{dash pattern=on 1pt off 1pt}])-8-7-6-5-)} &
\chemfig{[:-30,0.875]9*6(-1(-[:180,,,,dash pattern=on 1pt off 1pt]2-[:120]3-[:+60]4-[:0,,,,dash pattern=on 1pt off 1pt]\phantom{5}?)-[,,,,dash pattern=on 1pt off 1pt]8-7-6-[,,,,dash pattern=on 1pt off 1pt]5-)} \\
$\leftarrow$ \qquad $\leftarrow$ & $\longrightarrow$ & $\leftarrow$\quad$\rightarrow$ & \\
\CycleFirst \qquad \CycleSecond & \CycleThird & \CyclePseudoFirst\quad\CyclePseudoSecond & \\ 
%\bottomrule
\end{tabular}
% \vspace{\bigskipamount}
}
\end{figure}

Максимальная симметрия бициклического скелета~\cmpd{Bicycle331}, как группа автоморфизмов соответствующего молекулярного графа, порождается из группы перестановок всех вершин $\SymGroup{S}{9}$ системой нетривиальных орбит (классов эквивалентности вершин) вида $\left\langle(2\,4\,6\,8)(1\,5)(3\,7)\right\rangle$. В эвклидовом трёхмерном (\tqt{мировом}) пространстве $\SymGroup{E}{3}\left(\AGroup{R}\right)\simeq\AGroup{R}^3$ эта группа (изоморфно реализуется?) представлена (?)  в~виде точечной группы симметрии \(\SymGroup{C}{2v}\) для~\CC{} и~\BB{}. Для других конформаций симметрия снижается до~подгрупп \(\SymGroup{C}{s}\) у~\BC{}/\CB{} и~$\SymGroup{C}{2}$ у~\TT{}.

Конформация бициклического скелета~\cmpd{Bicycle331} традиционно определяется согласно разбиению~\ref{item:331:Decomposition:6:6}, т.\,е. из двух составляющих его конформаций шестичленных циклов. Поэтому нам кажется закономерным вначале кратко сформулировать хорошо известные основные положения конформационного анализа шестичленных циклов~(рис.~\ref{fig:Conformations:Six}).

\begin{figure}
\caption{Базисные конформации циклов для циклогексана~\cmpd{Cyclohexane} и~его аналогов\label{fig:Conformations:Six}}

\centerfloat{
\begin{tabular}{c|cc|c}
\toprule
%\cmpd{Bicycle331:XYZ}: & \multicolumn{3}{c}{\chemfig{[:-30]Z*6(-(-[:180]-[:120]X-[:+60]-[:0]?)--Y---)}} \\ \midrule
 & \ConfName{К} & \ConfName{Т} & \ConfName{В} \\ & ($\SymGroup{D}{3d}$) & ($\SymGroup{D}{2}$) & ($\SymGroup{C}{2v}$) \\ \midrule
\chemfig{[:-90,0.75]*6(------)} &
\chemfig{?<[:-60]-[:+20,,,,line width=\boldbondwidth]>[:-20]-[:+120]-[:-160]?} & 
\chemfig{?-[:-150,0.75]<[:-30,0.75]-[:+30,1.5,,,line width=\boldbondwidth]>[:-30,0.75]-[:-150,0.75]?} &
\chemfig{(-[:-35]?)<[:-60]-[:+0,1.5,,,line width=\boldbondwidth]>[:+60]-[:-145]?} \\
\cmpd{Cyclohexane} & & & \\ \midrule
\chemfig{[:-90,0.75]*6(-X-----)} &
\chemfig{X?<[:-60]-[:+20,,,,line width=\boldbondwidth]>[:-20]-[:+120]-[:-160]?} & 
\chemfig{X?-[:-150,0.75]<[:-30,0.75]-[:+30,1.5,,,line width=\boldbondwidth]>[:-30,0.75]-[:-150,0.75]?} &
\chemfig{X(-[:-35]?)<[:-60]-[:+0,1.5,,,line width=\boldbondwidth]>[:+60]-[:-145]?} \\ 
& & \chemfig{?-[:-150,0.75]X<[:-30,0.75]-[:+30,1.5,,,line width=\boldbondwidth]>[:-30,0.75]-[:-150,0.75]?} &
\chemfig{(-[:-35]?)<[:-60]X-[:+0,1.5,,,line width=\boldbondwidth]>[:+60]-[:-145]?} \\ \midrule
\chemfig{[:-90,0.75]*6(-X---Y--)} & 
\chemfig{X?<[:-60]-[:+20,,,,line width=\boldbondwidth]>[:-20]Y-[:+120]-[:-160]?} & 
\chemfig{X?-[:-150,0.75]<[:-30,0.75]-[:+30,1.5,,,line width=\boldbondwidth]Y>[:-30,0.75]-[:-150,0.75]?} &
\chemfig{X(-[:-35]?)<[:-60]-[:+0,1.5,,,line width=\boldbondwidth]>[:+60]Y-[:-145]?}  \\
& & \chemfig{?-[:-150,0.75]X<[:-30,0.75]-[:+30,1.5,,,line width=\boldbondwidth]>[:-30,0.75]Y-[:-150,0.75]?} &
\chemfig{(-[:-35]?)<[:-60]X-[:+0,1.5,,,line width=\boldbondwidth]>[:+60]-[:-145]Y?} \\ \midrule
\chemfig{[:-90,0.75]*6(---X--Y-)} & 
\chemfig{?<[:-60]-[:+20,,,,line width=\boldbondwidth]X>[:-20]-[:+120]Y-[:-160]?}&
\chemfig{?-[:-150,0.75]<[:-30,0.75]-[:+30,1.5,,,line width=\boldbondwidth]X>[:-30,0.75]-[:-150,0.75]Y?} &
\chemfig{Y(-[:-35]?)<[:-60]-[:+0,1.5,,,line width=\boldbondwidth]X>[:+60]-[:-145]?}  \\
 & & \chemfig{?-[:-150,0.75]<[:-30,0.75]Y-[:+30,1.5,,,line width=\boldbondwidth]>[:-30,0.75]X-[:-150,0.75]?} &
\chemfig{(-[:-35]?)<[:-60]-[:+0,1.5,,,line width=\boldbondwidth]X>[:+60]-[:-145]Y?} \\ \midrule
\chemfig{[:-90,0.75]*6(-X--Y--Z-)} &
\chemfig{X?<[:-60]-[:+20,,,,line width=\boldbondwidth]Y>[:-20]-[:+120]Z-[:-160]?} & 
\chemfig{?-[:-150,0.75]X<[:-30,0.75]-[:+30,1.5,,,line width=\boldbondwidth]Y>[:-30,0.75]-[:-150,0.75]Z?} &
\chemfig{X(-[:-35]?)<[:-60]-[:+0,1.5,,,line width=\boldbondwidth]Y>[:+60]-[:-145]Z?} \\
\bottomrule
\end{tabular}
%\vspace{\medskipamount}
}
\end{figure}

Для пространственных форм насыщенных шестичленных циклов выделяются три класса так называемых \emph{базисных} конформаций: \tqt{кресло}~(\ConfName{К}), \tqt{твист}~(\ConfName{T}) и \tqt{ванна}~(\ConfName{В})~(рис.~\ref{fig:Conformations:Six}); остальные формы являются промежуточными и могут быть выражены как \emph{выпуклые} комбинации этих трёх. Оптимальной является структура \ConfName{К}; \ConfName{В} обычно является переходным состоянием процесса взаимопревращения двух конформеров со~структурой \ConfName{T}; соответствующий процесс называется \emph{пседовращением} шестичленного цикла. Причиной такой нестабильности \tqt{ванны} признано сильное повышение энергии структур с ординарными связями в заслонённой конформации (так называемое \emph{питцеровское напряжение}). 

Для насыщенных производных~\cmpd{Bicycle331} выделяются конформации \tqt{двойное кресло}~(\CC{}), \tqt{ванна-кресло}~(\BC{}), \tqt{кресло-ванна} (\CB{}), \tqt{кресло-ванна}~(\CB{}) и~\tqt{двойной ванны} (\BB{})~(рис.~\ref{fig:System331:379XYZ:Conf}). В случае структурной эквивалентности \tqt{крыльев} (например, для исходного углеводорода~\cmpd{Bicycle331}) наблюдается \emph{конформационное вырождение} "--- формы \CB{} и~\BC{} становятся структурно и~энергетически неразличимыми.%~(рис.~\ref{fig:System331:Conf}).


\begin{figure}
\caption{Основные конформации  бицикло[3.3.1]нонана и его 3X,~7Y,~9Z-аналогов\label{fig:System331:379XYZ:Conf}}

  \centerfloat{
\begin{tabular}{c|c}
\multicolumn{2}{c}{
\ChemPicture{X?[a]<[:-30,1.5]-[:+30,,,,line width=\boldbondwidth](>[:+120]Z-[:-120](-[:-150]?[a])(-[:-30]-[:+30,1.25]Y?[b]))-[:-+30,,,,line width=\boldbondwidth]?[b,{<}]} 
}
\\
\multicolumn{2}{c}{\BB{}~(\TT{})} \\
\midrule
\ChemPicture{X?[a]<[:-30,1.25]-[:+30,,,,line width=\boldbondwidth](>[:+120]Z-[:-120](-[:-150]?[a]) (-[:-30]-[:-60]Y?[b]))-[:-+30,,,,line width=\boldbondwidth]?[b,{<}]}
& 
\ChemPicture{X?[a]<[:+60]-[:+30,,,,line width=\boldbondwidth](>[:+120]Z-[:-120](-[:-150]?[a])(-[:-30]-[:+30,1.25]Y?[b]))-[:-+30,,,,line width=\boldbondwidth]?[b,{<}]}
\\
\BC{} & \CB{}
\\
\midrule
\multicolumn{2}{c}{ %
\ChemPicture{X?[a]<[:+60]-[:+30,,,,line width=\boldbondwidth](>[:+120]Z-[:-120](-[:-150]?[a]) (-[:-30]-[:-60]Y?[b]))-[:-+30,,,,line width=\boldbondwidth]?[b,{<}]} % 
}
\\
\multicolumn{2}{c}{\CC{}}
\\
\end{tabular}
~
\begin{tabular}{cc}
  \multicolumn{2}{c}{\ChemPicture{?[a]<[:-30,1.5]-[:+30,,,,line width=\boldbondwidth](>[:+120]-[:-120](-[:-150]?[a])(-[:-30]-[:+30,1.25]?[b]))-[:-+30,,,,line width=\boldbondwidth]?[b,{<}]}} \\
  \multicolumn{2}{c}{\BB{}~(\TT{})} \\
  \midrule\ChemPicture{?[a]<[:-30,1.25]-[:+30,,,,line width=\boldbondwidth](>[:+120]-[:-120](-[:-150]?[a])(-[:-30]-[:-60]?[b]))-[:-+30,,,,line width=\boldbondwidth]?[b,{<}]} & 
  \ChemPicture{?[a]<[:+60]-[:+30,,,,line width=\boldbondwidth](>[:+120]-[:-120](-[:-150]?[a])(-[:-30]-[:+30,1.25]?[b]))-[:-+30,,,,line width=\boldbondwidth]?[b,{<}]} \\
  \multicolumn{2}{c}{\BC{} \(\equiv\) \CB{}} \\
  \midrule\multicolumn{2}{c}{\ChemPicture{?[a]<[:+60]-[:+30,,,,line width=\boldbondwidth](>[:+120]-[:-120](-[:-150]?[a])(-[:-30]-[:-60]?[b]))-[:-+30,,,,line width=\boldbondwidth]?[b,{<}]} } \\
  \multicolumn{2}{c}{\CC{}} \\
\end{tabular}
}
%\vspace{\medskipamount}
\end{figure}

Антибредтовские структуры "--- соединения со связью повышенной кратности в голове мостика: бицикло[3.3.1]нонан-$\Delta^{1(2)}$, 1-азабицикло\-[3.3.1]нонанон-2. Цикл, содержащий связь повышенной кратности, обычно предпочитает находиться в конформации близкой к \tqt{ванне}\dots

\begin{center}
  \chemfig{[:-30]*6(-(-[:180]-[:120]-[:+60]-[:0]?)---=-)} \quad
  \chemfig{[:-30]*6(-N(-[:180]C(=[:-120]O)-[:120]-[:+60]-[:0]?)-----)} \quad 
\end{center}


Форма \BB{} экспериментально наблюдается очень редко. Из-за значительных напряжений она характерна для сильно затруднённых молекул. В этих случаях она обычно обнаруживается в виде формы «двойного твиста» (\TT{}).

Дважды антибредтовские структуры на основе бицикло[3.3.1]нонана-$\Delta^{1(2),5(6)}$  и\,т.\,п.

\begin{center}
  \chemfig{[:-30]*6(-(=_[:180]-[:120]-[:+60]-[:0]?)---=-)} \quad
  \chemfig{[:-30]*6(-N(-[:180]C(=[:-120]O)-[:120]-[:+60]-[:0]?)---=-)} \quad \chemfig{[:-30]*6(-N(-[:180]C(=[:-120]O)-[:120]-[:+60]-[:0]?)--C(=[:0]O)-=-)}
\end{center}

Торсионные напряжения также являются фактором устойчивости и для формы \CB{}, содержащей подструктуру \tqt{ванны}, в которой, однако, заслонённые конформации связей незначительно отклоняются от идеальной формы \ConfName{В} с эндоциклическими двугранными углами $\tau_{9123}\simeq 0$ и $\tau_{9543}\simeq 0$. Часто эти отклонения симметричны, т. е., $\tau_{9123} = - \tau_{9543}$. Всё это является, в частности, показателем стереохимической (конформационной) жёсткости формы \tqt{кресла} насыщенного шестичленного цикла. Такая жёсткость образуется из сочетания низкой энергии и высоких собственных частот колебания структуры. Жёсткие конформации одной из подструктур молекулы могут стабилизировать менее энергетически выгодные формы других как термодинамически, так и кинетически.

\begin{figure}
\caption{3,7-взаимодействия в молекулах аналогов бицикло[3.3.1]нонана\label{fig:Interactions:37}}
\centerfloat{}
\end{figure}

Конформация \CC{} состоит из кресловидных шестичленных циклов, почти свободных от питцеровского напряжения. Основным фактором устойчивости для этой конформации оказывается невалентное взаимодействие между \tqt{крыльями} бициклической системы. Особенно значима в~этом смысле роль положений 3 и 7~(рис.~\ref{fig:Interactions:37}). Дисперсионное или электростатическое отталкивание между этими положениями или \emph{эндо}-заместителями в~них приводит к~повышению энергии и~дестабилизации \CC{} относительно других форм, тогда как притяжение стабилизирует соответствующую структуру.

\begin{figure}
\caption{2,4~(6,8)-взаимодействия в молекулах аналогов бицикло[3.3.1]нонана\label{fig:Interactions:2468}}
\centerfloat{}
\end{figure}

Дополнительным фактором, влияющим на относительные энергии конформеров \BC{}/\CB{} и~\CC{} являются взаимодействия положений 2 и~4 (6 и~8) внутри \tqt{крыльев}~(рис.~\ref{fig:Interactions:2468}). Они частично сходны с 1,3-диаксиальными взаимодействиями в соответствующих конформерах 1,3-\emph{цис}-дизамещённых аналогов циклогексана. Отталкивание приводит к уплощению циклов и сближению двух упомянутых форм.

Хотя рентгеноструктурные данные показывают, что скелет 1,5-диметил-3,7-дитиабицикло[3.3.1]нонан-9-селона~\cmpd{Dithia37Selone9} принимает в кристаллической структуре конформацию \CC{},~\cite{Brooks:1991} бициклическая основа других 9-замещённых 1,5-диметил-3,7-дитиабицикло[3.3.1]нонаны "--- \cmpd{Dithia37NNPPh39}~\cite{Brooks:1993}, а также сам исходный кетон \cmpd{Dithia37Ketone9}~\cite{Brooks:1995}"--- обладают конформацией~\BC{}.

\begin{center}
  \begin{tabular}{cccc}
\ChemPicture{S?[a]<[:+60]-[:+30,,,,line width=\boldbondwidth](-[:+45,,,,line width=\boldbondwidth]CH_3)(>[:+120]C(=[:+90]Se)-[:-120] (-[:+135]H_3C)(-[:-150]?[a])(-[:-30]-[:-60]S?[b]))-[:-+30,,,,line width=\boldbondwidth]?[b,{<}]} &
\ChemPicture{S?[a]<[:-30,1.25]-[:+30,,,,line width=\boldbondwidth](-[:+45,,,,line width=\boldbondwidth]CH_3) (>[:+120]C(=[:+90,0.875]N-[:+30,0.75]NPPh_3)-[:-120] (-[:+135]H_3C) (-[:-150]?[a]) (-[:-30]-[:-60]S?[b]))-[:-+30,,,,line width=\boldbondwidth]?[b,{<}]} & 
\ChemPicture{S?[a]<[:-30,1.25]-[:+30,,,,line width=\boldbondwidth](-[:+45,,,,line width=\boldbondwidth]CH_3) (>[:+120]C(=[:+90,0.875]O)-[:-120] (-[:+135]H_3C) (-[:-150]?[a]) (-[:-30]-[:-60]S?[b]))-[:-+30,,,,line width=\boldbondwidth]?[b,{<}]} & 
\\
\cmpd{Dithia37Selone9} & \cmpd{Dithia37NNPPh39} & \cmpd{Dithia37Ketone9} & \\
\end{tabular}
\end{center}

Для незамещённого 3,7-дитиабицикло[3.3.1]нонана расчётные данные также предсказывают оптимальность формы~\BC{}\dots

Скелет скопина, представляющий собой 9-метил-3-окса-9-аза\-три\-цикло\-[$3.3.1.0^{2,4}$]\-нонан\-ол-7~\cmpd{Scopine}~\cite{Ecija:2016} формально относится к семейству аналогов [$3.3.1$]. Его скелет находится в конформации \BC{}. 
Псевдопеллетьерин (9-метил-9-аза\-би\-цикло\-[3.3.1]нонан\-он-3)~\cmpd{Pseudopelletierine} 
\cite{VallejoLopez:2017}, напротив, предпочитает \CC{}-формы

\begin{center}
\begin{tabular}{ccc}
\chemfig{[:-30]N*6((-[:0,0.65]CH_3)>?[a]-[:-15]?[b]<O>(?[b])-[:-15](-[:180]-[:-120](<:[:180,0.75]HO)-[:-60]?[a])<)} & & \\
&& \\
\chemfig{HO>:*6(--(-[:+30]?)<N(-[:180,0.75]H_3C)>(-[:-30]?)--)} & 
\cmpd{Scopine} & 
\\
\end{tabular}

\begin{tabular}{c}\ChemPicture{[:-30]N*6((-[:0,0.75]CH_3)-?----(-[:180]-[:-120]C(=[:180,0.75]O)-[:-60]?)-)} \\ \cmpd{Pseudopelletierine} \\ \end{tabular}

\end{center}

[3.3.1]-пропелланы "--- производные трицикло[3.3.1.$0^{1,5}$]нонана общей формулы~\cmpd{Propellanes331}. В~частности, скелет
3,7-дитиа-[3.3.1]-пропелланов~\cmpd{Propellanes331S37} (производные 3,7-дитиатрицикло[3.3.1.$0^{1,5}$]нонана) в кристаллической фазе принимает конформацию \TT{} в достаточно широком ряду соединений: исходный~\cmpd{Propellanes331S37}, дисульфоксид~\cmpd{Propellanes331SO37}, комплексы с \ce{I2} и целым рядом хлоридов металлов.~\cite{Herbstein:1986,Marsh:1988}

\begin{center}
  \begin{tabular}{cccc}
  %\ChemPicture{[:-30]Y*6(-?[a](-[:180]-[:120]X-[:+60]-[:0]?[a])--X---)} 
  \chemfig{*5(--Y--(*5(--X--))(*3(-Z-))-)} & \chemfig{*5(--S--(*5(--S--))(*3(--))-)} & \chemfig{*5(--S(=[:0]O)--(*5(--S(=[:180]O)--))(*3(--))-)} & \\
  \cmpd{Propellanes331} & \cmpd{Propellanes331S37} & \cmpd{Propellanes331SO37}\\
\end{tabular}
\end{center}

Такое конформационное предпочтение трициклических систем, производных от [$3.3.1$], вероятно, связано с конформационным поведением шестичленного цикла в составе аналогов бицикло[$3.1.0$]гексана, где он обычно находится в виде \tqt{ванны}\dots

Примером неорганической структуры с соответствующим скелетом является тетраборат-анион \ce{B4O9^{4-}}~(\cmpd{Tetraborate}):
\begin{center}
  \begin{tabular}{cc}
  \chemfig{O-B*6(-O-B(-[:-90]O)(-[:0]O?)-O-B(-[:+90]O) (-[:0]O-[:-60]B?(-[:0]O))-O-)} & \\
  \cmpd{Tetraborate} & \\
  \end{tabular}
\end{center}



\section{Биспидины (производные и гетероаналоги 3,7-диазабицикло[3.3.1]нонана)}

Катион протонированного 1,5-диметил-3,7-диазабицикло[3.3.1]нонанона-9~\cmpd{BSP:O9:15Me2:HPlus} в кристаллах своих солей обнаруживает различную конформацию атома водорода связанного с нейтральным атомом азота в зависимости от способности аниона \ce{X-} к образованию водородных связей.

\begin{center}
  \begin{tabular}{ccc}
    \chemfig{N?[a](-[:-150]H)<[:+60]-[:+30,,,,line width=\boldbondwidth](-[:+45,,,,line width=\boldbondwidth]CH_3)(>[:+120]C(=[:+90,0.75]O)-[:-120] (-[:+135]H_3C) (-[:-150]?[a]) (-[:-30]-[:-60]{N}^+?[b](-[:-135]H)(-[:-45]H)))-[:-+30,,,,line width=\boldbondwidth]?[b,{<}]} & 
    \chemfig{N?[a](-[:-75]H)<[:+60]-[:+30,,,,line width=\boldbondwidth](-[:+45,,,,line width=\boldbondwidth]CH_3)(>[:+120]C(=[:+90,0.75]O)-[:-120] (-[:+135]H_3C) (-[:-150]?[a]) (-[:-30]-[:-60]{N}^+?[b](-[:-105]H)(-[:-30]H)))-[:-+30,,,,line width=\boldbondwidth]?[b,{<}]} & \\
    \texttt{BOZGAA}~\cmpd{CCDC:BOZGAA} &
    \texttt{BOZFUT}~\cmpd{CCDC:BOZFUT} & \\
    (\ce{\cmpd{BSP:O9:15Me2:HPlus} + Cl^-.H2O}) & (\ce{\cmpd{BSP:O9:15Me2:HPlus} + CH3COO^-}) & \\ 
  \end{tabular}
\end{center}

3,7-диметил-9,9-диоксо-9-тиа-3,7-диазабицикло[3.3.1]нонан~\cmpd{Bispidine9SO2} по данным РСА~(\texttt{AFUFUE}) находится в конформации~\CC{}. Структурные данные согласуются с результатом оптимизации соответствующей формы молекулы в приближении B3LYP / 6-31+G[dp].~\cite{Vlasova:2013:rus,Vlasova:2013}
\begin{center}
  \begin{tabular}{cc}
\ChemPicture{N?[a](-[:-150]H_3C)<[:+60]-[:+30,,,,line width=\boldbondwidth](>[:+120]S(=[:+135,0.875]O)(=[:+45,0.875]O)-[:-120](-[:-150]?[a]) (-[:-30]-[:-60]N?[b](-[:-30]CH_3)))-[:-+30,,,,line width=\boldbondwidth]?[b,{<}]} & \\
\cmpd{Bispidine9SO2} & \\
\end{tabular}
\end{center}

\subsection{Природные соединения, содержащие биспидиновый скелет}

Хинолизидиновые алкалоиды. 

Одним из представителей трициклических алкалоидов
является цитизин~\cmpd{Cytisine}. Родственные ему ромбифолин~\cmpd{Rhombifoline} и другие, анагирин~\cmpd{Anagyrine}, термопсин~\cmpd{Thermopsine}, баптифолин~\cmpd{Baptifoline} и эпибаптифолин~\cmpd{Epibaptifoline}.~\cite{Goller:2019} 

\begin{center}
\begin{tabular}{ccc}
\chemfig{*6(-C(=[,0.75]O)-N(*6(--(-[:+30]-[:+90]NH-[:+150]?[a])<:>:?[a]-))-=-=)} & 
\chemfig{*6(-C(=[,0.75]O)-N(*6(--(-[:+30]-[:+90]N (-[:+30]-[:-30]-[:-90]=[:-150,0.75]H_2C)-[:+150]?[a])<:>:?[a]-))-=-=)} & \\
\cmpd{Cytisine} &
\cmpd{Rhombifoline} & 
\\
\end{tabular}
\end{center}

\begin{center}
\begin{tabular}{ccc}
    \chemfig{*6(-C(=[,0.75]O)-N(*6(--(-[:+30](*6(-----N)) (<:[:-90,0.75]H)-[:+90]\phantom{N}-[:+150]?[a])<>?[a]-))-=-=)} & 
    \chemfig{*6(-C(=[,0.75]O)-N(*6(--(-[:+30](*6(-----N)) (<[:-90,0.75]H)-[:+90]\phantom{N}-[:+150]?[a])<>?[a]-))-=-=)} & \\
\cmpd{Anagyrine} & \cmpd{Thermopsine} & \\
  \end{tabular}
\end{center}

\begin{center}
\begin{tabular}{ccc}
  \chemfig{*6(-C(=[,0.75]O)-N(*6(--(-[:+30](*6(--(<:[:-30,0.875]OH)---N)) (<:[:-90,0.75]H)-[:+90]\phantom{N}-[:+150]?[a])<>?[a]-))-=-=)} &
  \chemfig{*6(-C(=[,0.75]O)-N(*6(--(-[:+30](*6(--(<[:-30,0.875]OH)---N)) (<:[:-90,0.75]H)-[:+90]\phantom{N}-[:+150]?[a])<>?[a]-))-=-=)} & \\
  \cmpd{Baptifoline} & \cmpd{Epibaptifoline} & \\
  \end{tabular}
\end{center}

Пахикарпин или (-)-спартеин~\cmpd{Pachycarpine}, (+)-спартеин~\cmpd{Sparteine}.

\begin{center}
  \begin{tabular}{ccc}
    \chemfig{*6(--N(*6(--(-[:+30](*6(-----N)) (<:[:-90,0.75]H)-[:+90]\phantom{N}-[:+150]?[a])<>?[a]-))-(<[:+90,0.75]H)---)} & 
    \chemfig{*6(--N(*6(--(-[:+30](*6(-----N)) (<[:-90,0.75]H)-[:+90]\phantom{N}-[:+150]?[a])<:>:?[a]-))-(<:[:+90,0.75]H)---)} & \\
    \cmpd{Pachycarpine} & \cmpd{Sparteine} & \\
  \end{tabular}
\end{center}

Трициклический алкалоид ангустифолин~\cmpd{Angustifoline} по данным исследования ЯЭО и других экспериментов ЯМР представляет собой структуру с бициклом [3.3.1] в конформации~\CC{}, которая стабилизируется внутримолекулярным мостиком \ce{N\bond{...}H-N}.~\cite{Wysocka:1994} Аналогичная картина наблюдается в кристаллической структуре гидроиодида $\alpha$-изолупанина~\ce{{\cmpd{IsolupanineA}}.HI}~\cite{Koziol:1986}
 
\begin{center}
\begin{tabular}{ccc}
\chemfig{*6(-C(=[,0.75]O)-N(*6(--(-[:+30] (-[:-30]-[:+30,0.875]=[:+90,0.75]CH_2)
  (<:[:-90,0.75]H)-[:+90]NH-[:+150]?[a])<>?[a]-))-(<[:+90,0.75]H)---)} & 
\chemfig{*6(-C(=[,0.75]O)-N(*6(--(-[:+30] (*6(-----N))  (<[:-90,0.75]H)-[:+90]\phantom{N}-[:+150]?[a])<>?[a]-))-(<[:+90,0.75]H)---)} &
\\
\cmpd{Angustifoline} & \cmpd{IsolupanineA} &  \\
  \end{tabular}
\end{center}


\section*{* * *}

Таким образом, конформационное поведение производных и гетероаналогов бицикло[3.3.1]нонана в наибольшей степени определяется взаимопревращением тех конформеров, в которых бициклический скелет приобретает либо конформацию \CC{}, либо формы \BC{}/\CB{}.

Большинство современных физико-химических методов исследования структуры и~динамики молекул используют результаты неэмпирического моделирования строения по~крайней мере на этапе формирования структурных гипотез. Такая методология подразумевает выбор в пользу той или иной гипотезы, являющейся результатом математического моделирования \tqt{из~первых принципов}, на основании экспериментальных измерений.