%% Согласно ГОСТ Р 7.0.11-2011:
%% 5.3.3 В заключении диссертации излагают итоги выполненного исследования, рекомендации, перспективы дальнейшей разработки темы.
%% 9.2.3 В заключении автореферата диссертации излагают итоги данного исследования, рекомендации и перспективы дальнейшей разработки темы.
\begin{enumerate}
  \item Неэмпирическими расчётными методами было исследовано конформационное поведение бицикло$[3.3.1]$нонана, его замещённых производных и насыщенных гетероциклических аналогов.
  \item Для анализа закономерностей конформационных эффектов замещения аппарат гипергомодесмотических реакций, разработанный Уилером для каркасных и полиненасыщенных углеводородов, был распространён на насыщенные гетероциклы и гетеробициклические соединения.
  \item На основе анализа \ldots
  \item Численные исследования показали, что \ldots
  \item Математическое моделирование показало \ldots
  \item Для выполнения поставленных задач был создан \ldots
\end{enumerate}
